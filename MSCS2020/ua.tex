\section{Universal Algebra}
\label{sec:univ-alg}

In this section we present our formalization in Agda of the core
concepts of heterogeneous universal algebra; in the next two sections
we focus respectively on equational logic and signature morphisms.
Meinke' and Tucker's chapter~\cite{meinke-tucker-1992} is our
reference for heterogeneous universal algebra; we will recall some
definitions and state all the results we formalized. Bove et
al.~\cite{agda-intro} offer a gentle introduction to Agda; we expect
the reader to be familiar with Haskell or some other functional
language.

\subsection{Signature, algebra, and homomorphism}

\paragraph*{Signature}

A \emph{signature} is a pair of sets $(S,F)$, called \textit{sorts}
and \textit{operations} (or \textit{function symbols}) respectively;
each operation is a triple $(f,[s_1,\ldots,s_n],s)$ consisting of a
\textit{name}, its \textit{arity}, and the \textit{target sort} (we
also use the notation $f \colon [s_1,...,s_n] \Rightarrow s$).

In Agda we use dependent records to represent signatures; in dependent
records the type of some field may depend on the value of a previous
one or parameters of the record. Type-theoretically one can take
operations (of a signature) as a family of sets indexed by the arity and target sort
(an indexed family of sets can also be thought as predicates over the
index set, an index satisfies the predicate if its family is
inhabited):
\begin{code}
\>[0]\AgdaKeyword{record}\AgdaSpace{}%
\AgdaRecord{Signature}\AgdaSpace{}%
\AgdaSymbol{:}\AgdaSpace{}%
\AgdaPrimitiveType{Set₁}\AgdaSpace{}%
\AgdaKeyword{where}\<%
\\
\>[0][@{}l@{\AgdaIndent{0}}]%
\>[2]\AgdaKeyword{field}\<%
\\
\>[2][@{}l@{\AgdaIndent{0}}]%
\>[4]\AgdaField{sorts}%
\>[11]\AgdaSymbol{:}\AgdaSpace{}%
\AgdaPrimitiveType{Set}\<%
\\
%
\>[4]\AgdaField{ops}%
\>[11]\AgdaSymbol{:}\AgdaSpace{}%
\AgdaSymbol{(}\AgdaDatatype{List}\AgdaSpace{}%
\AgdaField{sorts}\AgdaSymbol{)}\AgdaSpace{}%
\AgdaOperator{\AgdaFunction{×}}\AgdaSpace{}%
\AgdaField{sorts}\AgdaSpace{}%
\AgdaSymbol{→}\AgdaSpace{}%
\AgdaPrimitiveType{Set}\<%
\end{code}

\noindent \AgdaBound{A}\AgdaSymbol{×}\AgdaBound{B} corresponds to the non-dependent cartesian product
of \AgdaBound{A} and \AgdaBound{B}.

In order to declare a concrete signature one first declares
the set of sorts and the set of operations, which are then bundled
together in a record.  For example, the mono-sorted signature of monoids has a
unique sort, so we use the unit type |⊤| with its sole constructor
|tt|. We define a family indexed on |List ⊤ x ⊤|, with two constructors,
corresponding with the operations: a 0-ary operation |e|, and a binary
operation |∙| (note that constructors can
start with a lower-case letter or any symbol):

\begin{code}
\>[0]\AgdaKeyword{data}\AgdaSpace{}%
\AgdaDatatype{monoid{-}op}\AgdaSpace{}%
\AgdaSymbol{:}\AgdaSpace{}%
\AgdaDatatype{List}\AgdaSpace{}%
\AgdaRecord{⊤}\AgdaSpace{}%
\AgdaOperator{\AgdaFunction{×}}\AgdaSpace{}%
\AgdaRecord{⊤}\AgdaSpace{}%
\AgdaSymbol{→}\AgdaSpace{}%
\AgdaPrimitiveType{Set}\AgdaSpace{}%
\AgdaKeyword{where}\<%
\\
\>[0][@{}l@{\AgdaIndent{0}}]%
\>[3]\AgdaInductiveConstructor{e}\AgdaSpace{}%
\AgdaSymbol{:}\AgdaSpace{}%
\AgdaDatatype{monoid{-}op}\AgdaSpace{}%
\AgdaSymbol{(}\AgdaInductiveConstructor{[]}\AgdaSpace{}%
\AgdaOperator{\AgdaInductiveConstructor{,}}\AgdaSpace{}%
\AgdaInductiveConstructor{tt}\AgdaSymbol{)}\<%
\\
%
\>[3]\AgdaInductiveConstructor{∙}\AgdaSpace{}%
\AgdaSymbol{:}\AgdaSpace{}%
\AgdaDatatype{monoid{-}op}\AgdaSpace{}%
\AgdaSymbol{(}\AgdaInductiveConstructor{tt}\AgdaSpace{}%
\AgdaOperator{\AgdaInductiveConstructor{∷}}\AgdaSpace{}%
\AgdaOperator{\AgdaFunction{[}}\AgdaSpace{}%
\AgdaInductiveConstructor{tt}\AgdaSpace{}%
\AgdaOperator{\AgdaFunction{]}}\AgdaSpace{}%
\AgdaOperator{\AgdaInductiveConstructor{,}}\AgdaSpace{}%
\AgdaInductiveConstructor{tt}\AgdaSymbol{)}\<%
\\
\\
\>[0]\AgdaFunction{monoid{-}sig}\AgdaSpace{}%
\AgdaSymbol{:}\AgdaSpace{}%
\AgdaRecord{Signature}\<%
\\
\>[0]\AgdaFunction{monoid{-}sig}\AgdaSpace{}%
\AgdaSymbol{=}\AgdaSpace{}%
\AgdaKeyword{record}\AgdaSpace{}%
\AgdaSymbol{\{}\AgdaSpace{}%
\AgdaField{sorts}\AgdaSpace{}%
\AgdaSymbol{=}\AgdaSpace{}%
\AgdaRecord{⊤}\AgdaSpace{}%
\AgdaSymbol{;}\AgdaSpace{}%
\AgdaField{ops}\AgdaSpace{}%
\AgdaSymbol{=}\AgdaSpace{}%
\AgdaDatatype{monoid{-}op}\AgdaSpace{}%
\AgdaSymbol{\}}\<%
\end{code}

\noindent The signature of monoid actions has two sorts, one for the
monoid and the other for the set on which the monoid acts.

\begin{code}
\\
\>[0]\AgdaKeyword{data}\AgdaSpace{}%
\AgdaDatatype{actMonₛ}\AgdaSpace{}%
\AgdaSymbol{:}\AgdaSpace{}%
\AgdaPrimitiveType{Set}\AgdaSpace{}%
\AgdaKeyword{where}\<%
\\
\>[0][@{}l@{\AgdaIndent{0}}]%
\>[2]\AgdaInductiveConstructor{mon}\AgdaSpace{}%
\AgdaSymbol{:}\AgdaSpace{}%
\AgdaDatatype{actMonₛ}\<%
\\
%
\>[2]\AgdaInductiveConstructor{set}\AgdaSpace{}%
\AgdaSymbol{:}\AgdaSpace{}%
\AgdaDatatype{actMonₛ}\<%
\\
\\
\>[0]\AgdaKeyword{data}\AgdaSpace{}%
\AgdaDatatype{actMonₒ}\AgdaSpace{}%
\AgdaSymbol{:}\AgdaSpace{}%
\AgdaDatatype{List}\AgdaSpace{}%
\AgdaDatatype{actMonₛ}\AgdaSpace{}%
\AgdaOperator{\AgdaFunction{×}}\AgdaSpace{}%
\AgdaDatatype{actMonₛ}\AgdaSpace{}%
\AgdaSymbol{→}\AgdaSpace{}%
\AgdaPrimitiveType{Set}\AgdaSpace{}%
\AgdaKeyword{where}\<%
\\
\>[0][@{}l@{\AgdaIndent{0}}]%
\>[2]\AgdaInductiveConstructor{e}%
\>[5]\AgdaSymbol{:}\AgdaSpace{}%
\AgdaDatatype{actMonₒ}\AgdaSpace{}%
\AgdaSymbol{(}\AgdaInductiveConstructor{[]}\AgdaSpace{}%
\AgdaOperator{\AgdaInductiveConstructor{,}}\AgdaSpace{}%
\AgdaInductiveConstructor{mon}\AgdaSymbol{)}\<%
\\
%
\>[2]\AgdaInductiveConstructor{∙}%
\>[5]\AgdaSymbol{:}\AgdaSpace{}%
\AgdaDatatype{actMonₒ}\AgdaSpace{}%
\AgdaSymbol{(}\AgdaSpace{}%
\AgdaInductiveConstructor{mon}\AgdaSpace{}%
\AgdaOperator{\AgdaInductiveConstructor{∷}}\AgdaSpace{}%
\AgdaOperator{\AgdaFunction{[}}\AgdaSpace{}%
\AgdaInductiveConstructor{mon}\AgdaSpace{}%
\AgdaOperator{\AgdaFunction{]}}\AgdaSpace{}%
\AgdaOperator{\AgdaInductiveConstructor{,}}\AgdaSpace{}%
\AgdaInductiveConstructor{mon}\AgdaSymbol{)}\<%
\\
%
\>[2]\AgdaInductiveConstructor{*}%
\>[5]\AgdaSymbol{:}\AgdaSpace{}%
\AgdaDatatype{actMonₒ}\AgdaSpace{}%
\AgdaSymbol{(}\AgdaSpace{}%
\AgdaInductiveConstructor{mon}\AgdaSpace{}%
\AgdaOperator{\AgdaInductiveConstructor{∷}}\AgdaSpace{}%
\AgdaOperator{\AgdaFunction{[}}\AgdaSpace{}%
\AgdaInductiveConstructor{set}\AgdaSpace{}%
\AgdaOperator{\AgdaFunction{]}}\AgdaSpace{}%
\AgdaOperator{\AgdaInductiveConstructor{,}}\AgdaSpace{}%
\AgdaInductiveConstructor{set}\AgdaSymbol{)}\<%
\\
\\
\>[0]\AgdaFunction{actMon{-}sig}\AgdaSpace{}%
\AgdaSymbol{:}\AgdaSpace{}%
\AgdaRecord{Signature}\<%
\\
\>[0]\AgdaFunction{actMon{-}sig}\AgdaSpace{}%
\AgdaSymbol{=}\AgdaSpace{}%
\AgdaKeyword{record}\AgdaSpace{}%
\AgdaSymbol{\{}\AgdaSpace{}%
\AgdaField{sorts}\AgdaSpace{}%
\AgdaSymbol{=}\AgdaSpace{}%
\AgdaDatatype{actMonₛ}\AgdaSpace{}%
\AgdaSymbol{;}\AgdaSpace{}%
\AgdaField{ops}\AgdaSpace{}%
\AgdaSymbol{=}\AgdaSpace{}%
\AgdaDatatype{actMonₒ}\AgdaSpace{}%
\AgdaSymbol{\}}\<%

\end{code}

% \noindent Defining operations as a family indexed by arities and
% target sorts carries some benefits in the use of the library: as in
% the above examples, the names of operations are constructors of a
% family of datatypes and so it is possible to perform pattern matching
% on them. Notice also that infinitary signatures can be represented in
% our setting; in fact, all the results are valid for any signature, be
% it finite or infinite.

% We show two examples of signatures with infinite operations, the first
% might be more appealing to computer scientists and the second is more
% mathematical. The abstract syntax of a language for arithmetic expressions
% may have one sort, a constant operation for each natural number and a
% binary operation representing the addition of two expressions.
% \begin{spec}
% data Sortsₑ : Set where E : Sortsₑ
% data Opsₑ : List Sortsₑ × Sortsₑ → Set where
%   val   : (n : ℕ)   → Opsₑ ([] , E)
%   plus  : Opsₑ ( E ∷ [ E ] , E )
% \end{spec}

% \noindent Vector spaces over a field can be seen as a heterogeneous signature
% with two sorts~\cite{birkhoff-70} or as homogeneous signature
% over the field \cite[p. 132]{birkhoff-40}; this latter approach can be
% easily specified in our library, even if the field is infinite:
% \begin{spec}
% data Sorts-v Set where V : Sorts-v
% data Ops-v (F : Set) : Set where
%   _+_ : Ops-v ( V ∷ [ V ] , V )      -- vector addition
%   ν  : (f : F) → Ops-v ( [ V ] , V)  -- scalar multiplication
% vspace-sig : (F : Set) → Signature
% vspace-sig F = record {sorts = Sorts-v ; ops = Ops-v F}
% \end{spec}

% \paragraph{Algebra}
% An \emph{algebra} $\mathcal{A}$ for the signature $\Sigma$ consists of
% a family of sets indexed by the sorts of $\Sigma$ and a family of
% functions indexed by the operations of $\Sigma$. We use
% $\mathcal{A}_s$ for the \emph{interpretation} or the \emph{carrier} of
% the sort $s$; given an operation
% $f \colon [s_1,...,s_n] \Rightarrow s$, the interpretation of $f$ is a
% total function
% $f_{\mathcal{A}}\colon \mathcal{A}_{s_1} \times ... \times
% \mathcal{A}_{s_n} \rightarrow \mathcal{A}_s$. 
% We formalize the product $\mathcal{A}_{s_1} \times ... \times
% \mathcal{A}_{s_n}$ as \emph{heterogeneous vectors}. The
% type of heterogeneous vectors is parameterized by a set |I|
% and a family of sets indexed by |I|; and is indexed over a
% list of |I|:
% \begin{spec}
% data HVec {I : Set}  (A : I → Set) : List I → Set where
%   ⟨⟩    :  HVec A []
%   _▹_   :  ∀  {i is} → A i → HVec A is → HVec A (i ∷ is)
% \end{spec}
% \noindent The first parameter |I| is implicit (written in braces), which means that Agda
% will infer it by unification; notices that the constructor |_▹_| also
% takes two implicit arguments (we use the notation |∀| to skip their
% types). Let |Σ| be a signature and |A : sorts Σ → Set|, then the
% product $\mathcal{A}_{s_1} \times ... \times \mathcal{A}_{s_n}$ is
% formalized as |HVec A [s₁,…,sₙ]|.

% We need one more ingredient to give the formal notion of algebras: the
% mathematical definition of carriers assumes an underlying notion of
% equality.  In type theory one makes it apparent by using setoids (\ie
% sets paired with an equivalence relation), which were thoroughly
% studied by Barthe et al.~\cite{barthe-setoids-2003}. Setoids are
% defined in the standard library \cite{danielsson-agdalib} of
% Agda\footnote{Our formalization is based on several concepts defined
%   in the standard library.} as a record with three
% fields.
% \begin{spec}
% record Setoid : Set₁ where
%   field
%     Carrier       : Set                     
%     _≈_           : Carrier → Carrier → Set 
%     isEquivalence : IsEquivalence _≈_       
% \end{spec}
% \noindent The relation is given as a family of types indexed over a pair
% of elements of the carrier (|a b : Carrier| are related if the type |a
% ≈ b| is inhabited); |IsEquivalence _≈_| is again a record whose fields
% correspond to the proofs of reflexivity, symmetry, and transitivity.

% The finest equivalence relation over any set is given
% by the \emph{propositional equality} which only equates each element with
% itself, thus we can endow any set with a setoid structure with the function
% |setoid : Set → Setoid| of the standard library; vice versa, there
% is a forgetful functor | ∥_∥ : Setoid → Set | which returns the carrier.

% Setoid morphisms are functions which preserve the equality:
% \begin{spec}
% record _⟶_ (A B : Setoid) : Set where
%   field
%   _⟨$⟩_ : ∥ A ∥ → ∥ B ∥
%   cong : ∀ {a a'} → _≈_ A a a' → _≈_ B (_⟨$⟩ a) (_⟨$⟩ a')
% \end{spec}
% \noindent Notice that |_⟶_| is a record parameterized on two setoids.
% The first field is the function, by declaring it mixfix one can
% write |f ⟨$⟩ a| when |f : A ⟶ B| and |a : ∥ A ∥ |; the second field is
% given by a function mapping equivalence proofs on the source setoid to
% equivalence proofs on the target. Setoid morphisms will be used to
% give the interpretation of operations.

% Let |A : I → Set| be a family of sets and |P : {i : I} → A i → Set| a
% family of predicates, we let |P * : ∀ {is} → HVec A is → Set| be the
% point-wise extension of |P| over heterogeneous vectors. We also use
% the point-wise extension to define the setoid of heterogeneous vectors
% given a family of setoids |A : I → Setoid| and write |A ✳ is| for the
% setoid of heterogeneous vectors with index |is|. Algebras are
% formalized as records parameterized on the signature.
% \begin{spec}
% record Algebra (Σ : Signature) : Set₁  where
%   field
%     _⟦_⟧ₛ    : sorts Σ → Setoid
%     _⟦_⟧ₒ    : ∀  {ar s} → (f : ops Σ (ar , s)) → _⟦_⟧ₛ ✳ ar ⟶ _⟦_⟧ₛ s
% \end{spec}
% \noindent If |A| is an algebra for the signature |monoid-sig|, then
% |A ⟦ tt ⟧ₛ| is the carrier, |A ⟦ e ⟧ₒ| and |A ⟦ ∙ ⟧ₒ| are the interpretations
% of the operations. We invite the interested reader to browse the examples to
% see algebras for the signatures we have shown.

% \paragraph{Homomorphism}
% Let $\Sigma$ be a signature and let $\mathcal{A}$ and $\mathcal{B}$ be
% algebras for $\Sigma$. A \emph{homomorphism} $h$ from $\mathcal{A}$ to
% $\mathcal{B}$ is a family of functions indexed by the sorts
% $h_s : \mathcal{A}_s \rightarrow \mathcal{B}_s$, such that for each
% operation $f : [s_1,...,s_n] \Rightarrow s$, the following holds:
% \begin{equation}
%   h_s(f_{\mathcal{A}}(a_1,...,a_n)) = f_{\mathcal{B}}(h_{s_1}\,a_1,...,h_{s_n}\,a_n)\label{eq:homcond}
% \end{equation}
% \noindent Notice that this is a condition over the family of
% functions.

% In order to formalize homomorphisms we first introduce a
% notation for families of setoid morphisms indexed over sorts:
% \begin{spec}
% _⟿_ : ∀ {Σ} → Algebra Σ → Algebra Σ → Set
% _⟿_ {Σ} A B = (s : sorts Σ) → A ⟦ s ⟧ₛ ⟶ B ⟦ s ⟧ₛ
% \end{spec}
% \noindent We make explicit the implicit parameter |Σ| because
% otherwise |sorts Σ| does not make sense.\footnote{In the library we
%   use modules in order to avoid the repetition of the parameters |Σ|,
%   |A|, and |B|.} To enforce \eqref{eq:homcond} we also define a
% predicate over families of setoids morphisms:
% \begin{spec}
% homCond : ∀ {Σ} {A B} → A ⟿ B → Set
% homCond {Σ} {A} {B} h = ∀ {ar s} (f : ops Σ (ar , s)) (as : ∥ A ⟦_⟧ₛ ✳ ar ∥) → 
%          h s ⟨$⟩ (A ⟦ f ⟧ₒ ⟨$⟩ as) ≈ₛ B ⟦ f ⟧ₒ ⟨$⟩ map h as
% \end{spec}
% \noindent where |_≈ₛ_| is the equivalence relation of the setoid
% |B ⟦ s ⟧ₛ| and |map h| is the obvious extension of |h| over vectors.
% A homomorphism is a record parameterized by the source and target algebras
% \begin{spec}
% record Homo {Σ} (A B : Algebra Σ) : Set where
%   field
%     ′_′ : A ⟿ B
%     cond : homCond ′_′
% \end{spec}
% \noindent As expected, we have the identity homomorphism |Idₕ A : Homo A A| and
% the composition |G ∘ₕ F : Homo A C| of homomorphisms |F : Homo A B|
% and |G : Homo B C|. It is also expected that |F ∘ₕ Idₕ A| and |F| are
% equal in some sense. Since Agda is based on an
% intensional type theory, we cannot take the definitional equality
% (which distinguishes |id| from |λ n → n + 0| as functions
% on naturals); instead, we equate setoid morphisms 
% whenever their function parts are extensionally equal:
% \begin{spec}
%   _≈→_ : (f g : A ⟶ B) → Set
%   f ≈→ g  = ∀ (a : ∥ A ∥) → (f ⟨$⟩ a) ≈B (g ⟨$⟩ a)
% \end{spec}
% \noindent Two homomorphisms are equal when their corresponding setoid
% morphisms are extensionally equal:
% \begin{spec}
%   _≈ₕ_  : ∀ {Σ} {A B} → Homo A B → Homo A B → Set
%   F ≈ₕ F' = (s : sorts Σ) → ′ F ′ s ≈→ ′ F' ′ s
% \end{spec}
% \noindent With respect to this equality, it is straightforward to
% prove the associativity of the composition |_∘ₕ_| and that |Idₕ| is
% the identity for the composition.
% % \comment{It is straightforward to define the product |A₁ × A₂| of algebras |A₁|
% % and |A₂| and the projection homomorphisms |Πᵢ : Homo (A₁ × A₂) Aᵢ| where
% % |′ Πᵢ ′ _ ⟨$⟩ p = projᵢ p|.}
% \subsection{Quotient and subalgebras}
% In order to prove the more basic results of universal algebra, we need
% to formalize subalgebras, congruence relations, and quotients.

% \paragraph{Subalgebra}

% A subalgebra $\mathcal{B}$ of an algebra $\mathcal{A}$ consists of a
% family of subsets $\mathcal{B}_s \subseteq \mathcal{A}_s$, that are
% closed under the interpretation of operations; that is, for every
% $ f : [s_1, \ldots,s_n] \Rightarrow s$ the following condition holds
% \begin{equation}
% (a_1,\ldots,a_n) \in \mathcal{B}_{s_1} \times \cdots \times\mathcal{B}_{s_n}   \text{ implies }
%   f_\mathcal{A}(a_1,\ldots,a_n) \in \mathcal{B}_{s} \enspace .
%  \label{eq:opclosed}
% \end{equation} 
% \noindent As shown by Salvesen and Smith \cite{salvesen-subsets},
% subsets cannot be added as a construction in intensional type theory
% because they lack desirable properties. If |A : Set| and |P : A → Set|
% is a predicate over |A|, then one can represent the subset containing the
% elements on |A| that satisfy |P| as the dependent sum\footnote{Do not confuse
% the syntax |Σ[_∈_]_| of dependent sum, with a variable |Σ : Signature|}
% |Σ[ a ∈ A ] P| whose inhabitants are
% pairs |(a , p)| where |a : A| and |p : P a|.\comment{This is not so
%   pleasant as there can be several proofs of |P a|.} Let us consider a
% setoid |A| and a predicate on its carrier |P : ∥ A ∥ → Set|; first
% notice that we can lift the subset construction to setoids, defining
% the equivalence relation |(a , q) ≈ (a' , q')| iff |a ≈ a'|.
% Moreover, we might assume that |P| is \emph{well-defined},
% which means that |a ≈A a'| and |P a| imply
% |P a'|.
% \begin{spec}
%   WellDef : (A : Setoid) → (P : ∥ A ∥ → Set) → Set
%   WellDef A P = ∀ {a a'} → a ≈A a' → P a → P a'
% \end{spec}
% \noindent A family of well-defined predicates will induce a subalgebra;
% but we still need to formalize the condition \eqref{eq:opclosed}.  Let
% |Σ| be a signature and |A| be an algebra for |Σ|.
% \begin{spec}
%     opClosed : (P : (s : sorts Σ) → ∥ A ⟦ s ⟧ₛ∥ → Set) → Set
%     opClosed P = ∀ {ar s} (f : ops Σ (ar , s)) → (P * ⟨→⟩ P s) (A ⟦ f ⟧ₒ ⟨$⟩_)
% \end{spec}
% \noindent |(Q ⟨→⟩ R) f| can be read as the pre-condition |Q| implies
% post-condition |R| after applying |f|; so |opClosed P f| asserts that if a vector |a*|
% satisfies the predicate |P|, then the application of the interpretation |A ⟦ f ⟧ₒ|
% to |a*| satisfies |P|, according to Eq.~\eqref{eq:opclosed}.
% In summary, given an algebra
% |A| for the signature |Σ| and a family |P| of predicates, such that |P
% s| is well-defined for every sort |s| and |P| is |opClosed|, we can
% define the |SubAlgebra A P| 
% \begin{spec}
% SubAlgebra : ∀ {Σ} A P → WellDef P → opClosed P → Algebra Σ
% \end{spec}
% \noindent In the subalgebra, an operation |f| is interpreted by
% applying the interpretation of |f| in |A| to the first components of
% the argument (and use the fact that |P| is op-closed to show that
% the resulting value satisfies the predicate of the target sort).

% \paragraph{Congruence and Quotients}
% A \emph{congruence} on a $\Sigma$-algebra
% $\mathcal{A}$ is a family
% $Q$ of equivalence relations indexed by sorts, and each of them is
% closed under the operations of the algebra. This condition is called
% \emph{substitutivity} and can be formalized using the point-wise
% extension of $Q$ over vectors: for every operation $ f : [s_1,
% \ldots,s_n] \Rightarrow s$
% \begin{equation}
%   (\vec{a},\vec{b}) \in Q_{s_1} \times \cdots \times Q_{s_n} \text{ implies }
%  (f_{\mathcal{A}}(\vec{a}) , f_{\mathcal{A}}(\vec{b})) \in Q_s\label{eq:congcond}
% \end{equation} 

% As with predicates, we say that a binary relation over a setoid is
% well-defined if it is preserved by the setoid equality; this notion
% can be extended over families of relations in the obvious way. In our
% formalization, a congruence on an algebra |A| is a family |Q| of
% well-defined, equivalence relations. The substitutivity condition
% \eqref{eq:congcond} is aptly captured by the generalized containment
% operator |_=[_]⇒_| of the standard library, where |P =[ f ]⇒ Q| if,
% for all |a,b ∈ A|, |(a,b) ∈ P| implies |(f a, f b) ∈ Q|.
% \begin{spec}
% record Congruence (A : Algebra Σ) : Set where
%   field
%     rel : (s : sorts Σ) → (∥ A ⟦ s ⟧ₛ ∥ → ∥ A ⟦ s ⟧ₛ ∥ → Set)
%     welldef : (s : sorts Σ) → WellDefBin (rel s)
%     cequiv : (s : sorts Σ) → IsEquivalence (rel s)
%     csubst : ∀ {ar s} → (f : ops Σ (ar , s)) → rel * =[ A ⟦ f ⟧ₒ ⟨$⟩_  ]⇒ rel s
% \end{spec}

% Given a congruence $Q$ over the algebra $\mathcal{A}$, we can obtain a
% new algebra, the \emph{quotient algebra}, by interpreting the sort $s$
% as the set of equivalence classes $\mathcal{A}_s / Q$; the condition
% \eqref{eq:congcond} ensures that the operation $ f : [s_1, \ldots,s_n]
% \Rightarrow s$ can be interpreted as the function mapping the vector
% $([a_1],\ldots,[a_n])$ of equivalence classes into the class $[
% f_\mathcal{A}(a_1,\ldots,a_n)]$. In Agda, we take the same carriers
% from |A| and use |Q s| as the equivalence relation over |∥ A ⟦ s ⟧ₛ
% ∥|; operations are interpreted just as in |A| and the congruence proof
% is given by |csubst Q|.

% \paragraph{Isomorphism Theorems} The definitions of subalgebras,
% quotients, and epimorphisms (surjective homomorphisms) are related by
% the three isomorphism theorems. Although there is some small overhead
% by the coding of subalgebras, the proofs follow very close what one would
% do in paper. For proving these results we also defined the
% \emph{kernel} and the \emph{homomorphic} image of homomorphisms.

% \begin{theorem}[First isomorphism theorem] If $h : \alg{A} \rightarrow \alg{B}$
% is an epimorphism, then $\alg{A} /\! \mathop{ker} h \simeq \alg{B}$.
% \end{theorem}
% \noindent Remember that the quotient $\alg A /\! \mathop{ker} h$ has
% the same carrier as $\alg A$, so $h$ counts as the underlying function
% and it respects the equivalence relation $\mathop{ker} h$ by
% definition. Clearly $h$ is surjective and its injectivity is obvious.

% \begin{theorem}[Second isomorphism theorem] If $\phi,\psi$ are congruences over $\alg A$,
% such that $\psi \subseteq \phi$, then $(\alg A / \phi) \simeq (\alg A / \psi)/(\phi / \psi)$. 
% \end{theorem}

% \noindent In order to prove this theorem, we first prove that
% $\phi / \psi$ is a congruence over $\alg A / \psi$: it suffices to
% prove the well-definedness of $\phi / \psi$, \ie that
% $(a,c) \in \psi$, $(b,d) \in \psi$, and $(a,b) \in \phi$ imply
% $(c,d) \in \phi$; an obvious consequence of $\psi \subseteq
% \phi$. Notice that the underlying carriers are the same in both cases:
% those of $\alg A$, so the identity function is the mediating
% isomorphism and the proof that it satisfies the homomorphism condition
% is trivial.

% \begin{theorem}[Third isomorphism theorem] Let $\alg B$ be a
% subalgebra of $\alg A$ and $\phi$ be a congruence over $\alg A$. Let
% $[\alg B]^{\phi}=\{K \in A / \phi : K \cap B \not= \emptyset\}$ and
% let $\phi_B$ be the restriction of $\phi$ to $\alg B$, then
% \begin{enumerate*}[label= (\roman*),itemjoin={}]
% \item$\phi_B$ is a congruence over $\alg B$;
% \item$[\alg B]^{\phi}$ is a subalgebra of~$\alg A$; and,
% \item$[\alg B]^{\phi} \simeq \alg B / \phi_B$.
% \end{enumerate*}
% \end{theorem}
% \noindent First we define the \emph{trace} of the congruence $\phi$ on
% the subalgebra $\alg B$ as the restriction of $\phi$ on $\alg B$;
% proving that it is a congruence over $\alg B$ involves some
% bureaucracy (remember that an element of a subalgebra is a pair
% $(a, p)$ such that $a \in A$ and $p$ is the proof that $a$ satisfies
% the predicate defining $B$). For the second item, we model
% $[\alg B]^{\phi}$ as a predicate over $\alg A$; it is satisfied by
% $a \in A$ if there is some $b \in B$ such that $(a,b) \in \phi$. The
% well-definedness of this predicate is easy (assuming $(a,a') \in \phi$
% and $b\in B$ with $(a,b) \in \phi$, one can easily prove that
% $(a',b) \in \phi$, thus $b$ is also the witness for proving that $a'$
% satisfies the predicate). To prove that the predicate is closed under
% the operations we take a vector of triples $(as,bs,ps)$ consisting of
% a vector of elements in $A$, a vector of elements in $B$, and the
% proofs $ps$ proving that $(as_i,bs_i)\in\phi$. Let $f$ be an
% operation, since $B$ is closed we know $f(b_1,\ldots,b_n)\in B$ and
% because $\phi$ is also closed we deduce
% $(f(a_1,\ldots,a_n),f(b_1,\ldots,b_n))\in\phi$. Finally, the
% underlying function witnessing the isomorphism
% $[\alg B]^{\phi} \simeq \alg B / \phi_B$ is given by composing the
% second projection with the first projection, thus getting an element
% in $B$.

% \subsection{The Term Algebra is initial}

% A $\Sigma$-algebra $\mathcal{A}$ is called \emph{initial} if for any
% $\Sigma$-algebra $\mathcal{B}$ there exists exactly one homomorphism
% from $\mathcal{A}$ to $\mathcal{B}$. We give an abstract definition of
% this universal property, existence of a unique element, for any set
% |A| and any relation |R|
% \begin{spec}
% hasUnique {A} _≈_ = A × (∀ a a' → a ≈ a')
% \end{spec}
% \noindent and initiality can be formalized directly:
% \begin{spec}
% Initial : ∀ {Σ} → Algebra Σ → Set
% Initial {Σ} A = ∀ (B : Algebra Σ) → hasUnique (_≈ₕ_ A B)
% \end{spec}
% Given a signature $\Sigma$ we can define the \emph{term algebra}
% $\mathcal{T}$, whose carriers are sets of well-typed words built up
% from the function symbols.  Sometimes this universe is called the
% \emph{Herbrand Universe} and is inductively defined:
% \begin{prooftree}
% \AxiomC{$t_1 \in \mathcal{T}_{s_1}$}
% \AxiomC{$\cdots$}
% \AxiomC{$t_n \in \mathcal{T}_{s_n}$}
% \RightLabel{$f : [s_1,...,s_{n}] \Rightarrow s$}
% \TrinaryInfC{$f\,(t_1,...,t_{n}) \in \mathcal{T}_s$}
% \end{prooftree}
% \noindent This inductive definition can be written directly in Agda:
% \begin{spec}
%   data HU {Σ : Signature} : (s : sorts Σ) → Set where
%     term : ∀  {ar s} → (f : ops Σ (ar ↦ s)) → HVec HU ar → HU s
% \end{spec}
% \noindent We use propositional equality to turn each |HUₛ| into a
% setoid, thus completing the interpretation of sorts. To interpret an
% operation $f \colon [s_1,\ldots,s_n] \Rightarrow s$ we map the vector
% |⟨t₁,…,tₙ⟩ : HVec HU [s₁,…,sₙ]| to |term f ⟨t₁,…,tₙ⟩|; we omit
% the proof of |cong|, which is too long and tedious to be
% shown.
% \begin{spec}
%   ∣T∣ : (Σ : Signature) → Algebra Σ
%   ∣T∣ Σ = record  { _⟦_⟧ₛ = setoid ∘ (HU {Σ}) ; _⟦_⟧ₒ  = ∣_|ₒ }
%     where | f ∣ₒ = record { _⟨$⟩_ = term f ; cong = ... }
% \end{spec}
% \noindent Terms can be interpreted in any algebra
% $\mathcal{A}$, yielding an homomorphism $h_A \colon \mathcal{T}
% \to \mathcal{A}$
% \[
%   h_A (f(t_1,\ldots,t_n)) = f_{\mathcal{A}}\,(h_A\,t_1,...,h_A\,t_n) \enspace .
% \] 
% \noindent We cannot translate this definition directly in Agda, instead
% we have to mutually define | ∣h∣→A | and its extension over vectors
% | ∣h*∣→A| 
% \begin{spec}
%   ∣h∣→A : ∀ {Σ} → (A : Algebra Σ) → {s : sorts Σ} → HU s → ∥ A ⟦ s ⟧ₛ ∥
%   ∣h∣→A A (term f ts) = A ⟦ f ⟧ₒ ⟨$⟩ (∣h*∣→A ts)
% \end{spec}
% \noindent It is straightforward to prove that |∣h∣→A| preserves
% propositional equality and satisfies the homomorphism condition by
% construction. To finish the proof that | ∣T∣ Σ | is initial, we prove,
% by recursion on the structure of terms, that any pair of homomorphisms
% are extensionally equal.

% mscguide.tex
% v1.0, released 11 Nov 2019
% Copyright 2019 Cambridge University Press

\documentclass{msc}

\usepackage{amsmath}
\usepackage{graphicx}


\begin{document}

\lefttitle{LaTeX\ Supplement}
\righttitle{Mathematical Structures in Computer}

\papertitle{Article}

\jnlPage{1}{00}
\jnlDoiYr{2019}
\doival{10.1017/xxxxx}
\title{Formalization of Universal Algebra in Agda}

\begin{authgrp}
  \author{Miguel Pagano}
  \author{Alejandro Gadea}
\author{ Emmanuel Gunther}


\affiliation{Facultad de Matem\'atica, Astronom\'ia y F\'isica,\\
        Universidad Nacional de C\'ordoba\\}
\end{authgrp}

\history{(Received xx xxx xxx; revised xx xxx xxx; accepted xx xxx xxx)}
%\received{20 March 1995; revised 30 September 1998}


\begin{abstract}
In this work we present a novel formalization of
universal algebra in Agda. We show that heterogeneous signatures can
be elegantly modelled in type-theory using sets indexed by arities to
represent operations. We prove elementary results of heterogeneous
algebras, including the proof that the term algebra is initial and the
proofs of the three isomorphism theorems. We further formalize equational theory and
prove soundness and completeness. At the end, we define (derived)
signature morphisms, from which we get the contra-variant functor
between algebras; moreover, we also proved that, under some
restrictions, the translation of a theory induces a contra-variant
functor between models.
\end{abstract}

\begin{keywords}
  universal algebra; formalization of mathematics; equational logic;
\end{keywords}

\maketitle

\section{Introduction}

Universal algebra, initiated by \cite{birkhoff-1935}, is the study of
different types of algebraic structures at an abstract level, thus
revealing common results which are valid for all of them and also
allowing for a unified definition of constructions (for example,
products, sub-algebra, congruences). Universal algebra has played a
relevant role in computer science since its earliest days, in
particular the seminal paper of \cite{birkhoff-70} features regular
languages as a prominent example; shortly before, \cite{burstall69}
had proved properties of programs using structural induction, by
conceiving the language as an initial algebra. The ADJ
group \citep[see][]{goguen-adj} promoted multi-sorted algebras as a key
theoretical tool for specifying data
types~\citep{adj-abstract-data-types}, semantics~\citep{goguen-77}, and
compilers~\citep{thatcher1981more}.  More recently,
institutions~\citep{goguen-92}, a generalization of universal algebra,
has been used as a foundation of methodologies and frameworks for
software specification and development~\citep{sannella2012foundations}.

In spite of the rich mathematical theory of heterogeneous algebras
(mostly inherited from the monosorted setting, but with some nuances
as explained by \cite{tarlecki-nuances} and \cite{adamek-2012}), there are few publicly
available formalizations in type theory (which we discuss in the
conclusion).  This situation is to be contrasted with impressive
advances in mechanization of particular algebraic structures as
witnessed, for example, by the proof of the Feit-Thompson theorem in
Coq by Gonthier and his team~\citeyearpar{gonthier2013machine}.

In this work we present an Agda library of multi-sorted universal
algebra aiming both a reader with a background in the area of
algebraic specifications and also the community of type theory.  For
the former, we try to explain enough Agda in order to keep the paper
self-contained; we will recall the most important definitions of
universal algebra. The main contributions of this paper are:
\begin{itemize}
\item the first formalization of basic universal algebra in Agda;
\item the first, to our knowledge, formalization in type theory of
  derived signature morphisms and the reduct algebras induced by them;
\item a novel representation of heterogeneous signatures in
  type theory, where operations are modelled using sets indexed by
  arities; and
\item an independent library of heterogeneous vectors.
\end{itemize}

We formalized the proof that the term algebra is initial and also the
proofs of the three isomorphism theorems; moreover we also define a
deduction system for conditional equational logic and prove its
soundness and completeness with respect to Goguen and Meseguer
semantics~\citeyearpar{GoguenM82}; moreover we prove that the term model (the
term algebra quotiened by the provable equivalence relation) is
initial among models. Then we show that the class of models of any
conditional equational theory is closed by Isomorphisms, Subalgebras
and (arbitrary) products; if the theory is equational then its models
are also closed by homomorphic images. As a nice corollary we can
prove that fields are not (finitely) axiomatizable.

We also showed that the translations of theories arising from derived
signature morphisms induces a contra-variant functor between models.
In the complete development, which is available at
\url{https://cs.famaf.unc.edu.ar/~mpagano/universal-algebra/}, we
include several examples featuring both the use of equational
reasoning and the preservation of models by signature morphisms.

\textit{Outline.} In Sec.~\ref{sec:univ-alg} we introduce the basic
concepts of Universal Algebra: signature, algebras and homomorphisms,
congruences, quotients and subalgebras, the proofs of three
isomorphisms theorems, and the proof of the initiality of the term
algebra.  In Sec.~\ref{sec:eqlog} we define an equational calculus, introducing
concepts of equations, theories, satisfiability and provability,
ending with the Birkhoff proofs of soundness and completeness.  In
Sec.~\ref{sec:trans} we introduce a new representation of (derived) signature
morphisms and reduct algebras (and homomorphisms), and we explore
translation and implication of theories.
% with the property of preservation of models, under some
%restrictions.
Finally, we conclude in Sec.~\ref{sec:conclusions}, discussing the
work done, and pointing out possible future directions.





\end{document}

% mscguide.tex
% v1.0, released 11 Nov 2019
% Copyright 2019 Cambridge University Press

\documentclass{msc}

\usepackage{amsmath}
\usepackage{graphicx}
\usepackage{amsthm}
 \usepackage{stmaryrd}
 \usepackage{agda}
 \usepackage{bussproofs}

\DeclareUnicodeCharacter{2223}{\ensuremath{|}}
\DeclareUnicodeCharacter{2115}{\ensuremath{\mathds{N}}}
\DeclareUnicodeCharacter{2295}{\ensuremath{\oplus}}
\DeclareUnicodeCharacter{2081}{\ensuremath{_1}}
\DeclareUnicodeCharacter{2082}{\ensuremath{_2}}
\DeclareUnicodeCharacter{2257}{\ensuremath{\circeq}}
\DeclareUnicodeCharacter{22A2}{\ensuremath{\vdash}}
\DeclareUnicodeCharacter{2236}{\ensuremath{:}}
\DeclareUnicodeCharacter{2200}{\ensuremath{\forall}}
\DeclareUnicodeCharacter{27E6}{\ensuremath{\llbracket}}
\DeclareUnicodeCharacter{27E7}{\ensuremath{\rrbracket}}
\DeclareUnicodeCharacter{219D}{\ensuremath{\leadsto}}
\DeclareUnicodeCharacter{2237}{\ensuremath{:\!:}}
\DeclareUnicodeCharacter{3B5}{\ensuremath{\epsilon}}
\DeclareUnicodeCharacter{25B9}{\ensuremath{\triangleright}}
\DeclareUnicodeCharacter{27E8}{\ensuremath{\langle}}
\DeclareUnicodeCharacter{27E9}{\ensuremath{\rangle}}
\DeclareUnicodeCharacter{21D2}{\ensuremath{\Rightarrow}}
\DeclareUnicodeCharacter{209C}{\ensuremath{_t}}
\DeclareUnicodeCharacter{2261}{\ensuremath{\equiv}}
\DeclareUnicodeCharacter{2080}{\ensuremath{_0}}
\DeclareUnicodeCharacter{3A3}{\ensuremath{\Sigma}}
\DeclareUnicodeCharacter{3BB}{\ensuremath{\lambda}}
\DeclareUnicodeCharacter{22A5}{\ensuremath{\bot}}
\DeclareUnicodeCharacter{22A4}{\ensuremath{\top}}
\DeclareUnicodeCharacter{2219}{\ensuremath{\odot}}
\DeclareUnicodeCharacter{209B}{\ensuremath{_s}}
\DeclareUnicodeCharacter{2092}{\ensuremath{_o}}
\DeclareUnicodeCharacter{2098}{\ensuremath{_m}}
\DeclareUnicodeCharacter{2099}{\ensuremath{_n}}
\DeclareUnicodeCharacter{2095}{\ensuremath{_h}}
\DeclareUnicodeCharacter{2091}{\ensuremath{_e}}
\DeclareUnicodeCharacter{2093}{\ensuremath{_x}}
\DeclareUnicodeCharacter{1D62}{\ensuremath{_i}}
\DeclareUnicodeCharacter{2248}{\ensuremath{\approx}}
\DeclareUnicodeCharacter{2225}{\ensuremath{\|}}
\DeclareUnicodeCharacter{27F6}{\ensuremath{\longrightarrow}}
\DeclareUnicodeCharacter{2733}{\ensuremath{\ast}}
\DeclareUnicodeCharacter{27FF}{\ensuremath{\rightsquigarrow}}
\DeclareUnicodeCharacter{2032}{\ensuremath{'}}
\DeclareUnicodeCharacter{2218}{\ensuremath{\circ}}
\DeclareUnicodeCharacter{2208}{\ensuremath{\in}}
\DeclareUnicodeCharacter{3014}{\ensuremath{\llparenthesis}}
\DeclareUnicodeCharacter{3015}{\ensuremath{\rrparenthesis}}
\DeclareUnicodeCharacter{228E}{\ensuremath{\uplus}}
\DeclareUnicodeCharacter{3B8}{\ensuremath{\theta}}
\DeclareUnicodeCharacter{3C3}{\ensuremath{\sigma}}
\DeclareUnicodeCharacter{3C6}{\ensuremath{\phi}}
\DeclareUnicodeCharacter{3C8}{\ensuremath{\psi}}
\DeclareUnicodeCharacter{22C0}{\ensuremath{\bigwedge}}
\DeclareUnicodeCharacter{2227}{\ensuremath{\wedge}}
\DeclareUnicodeCharacter{22A8}{\ensuremath{\models}}
\DeclareUnicodeCharacter{27EA}{\ensuremath{\langle\!\!\langle}}
\DeclareUnicodeCharacter{27EB}{\ensuremath{\rangle\!\!\rangle}}
\DeclareUnicodeCharacter{2228}{\ensuremath{\vee}}
\DeclareUnicodeCharacter{223C}{\ensuremath{~}}
\DeclareUnicodeCharacter{220E}{\ensuremath{\qed}}
\DeclareUnicodeCharacter{22A9}{\ensuremath{\Vdash}}
%%MIGUEL
\DeclareUnicodeCharacter{207D}{\ensuremath{^{(}}}
\DeclareUnicodeCharacter{207E}{\ensuremath{^{)}}}
\DeclareUnicodeCharacter{2004}{\ensuremath{:=}}
\DeclareUnicodeCharacter{2238}{\ensuremath{\dotminus}}
\DeclareUnicodeCharacter{3010}{\ensuremath{(}}
\DeclareUnicodeCharacter{3011}{\ensuremath{)}}
\DeclareUnicodeCharacter{21DD}{\ensuremath{\rightsquigarrow}}
\DeclareUnicodeCharacter{203C}{\ensuremath{!\!!}}
\DeclareUnicodeCharacter{2C7C}{\ensuremath{_{j}}}
\DeclareUnicodeCharacter{3B3}{\ensuremath{\gamma}}
\DeclareUnicodeCharacter{22C3}{\ensuremath{\bigcup}}
\DeclareUnicodeCharacter{2203}{\ensuremath{\exists}}
\DeclareUnicodeCharacter{2209}{\ensuremath{\not\in}}
\DeclareUnicodeCharacter{2250}{\ensuremath{:=}}
\DeclareUnicodeCharacter{D7}{\ensuremath{\times}}
\DeclareUnicodeCharacter{2192}{\ensuremath{\rightarrow}}


\newcommand{\codigo}[1]{}
\usepackage{environ}
\NewEnviron{spec}{}{}


\newenvironment{quote}
               {\list{}{\rightmargin\leftmargin}%
                \item\relax}
               {\endlist}


\newcounter{descriptcount}
\renewcommand*\thedescriptcount{\arabic{descriptcount}}
\newcommand{\setoidMor}{{\xrightarrow{\,\thickapprox\,}}}

\newcommand{\alg}[1]{\mathcal{#1}}

\newcommand{\ie}{i.e. }

\begin{document}

\lefttitle{Universal Algebra in Agda}
\righttitle{Mathematical Structures in Computer}

\papertitle{Article}

\jnlPage{1}{00}
\jnlDoiYr{2019}
\doival{10.1017/xxxxx}
\title{Formalization of Universal Algebra in Agda}

\begin{authgrp}
  \author{Alejandro Gadea}
  \author{Emmanuel Gunther}
  \author{Miguel Pagano}

  \affiliation{Facultad de Matem\'atica, Astronom\'ia y F\'isica,\\
        Universidad Nacional de C\'ordoba\\}
\end{authgrp}

\history{(Received xx xxx xxx; revised xx xxx xxx; accepted xx xxx xxx)}
%\received{20 March 1995; revised 30 September 1998}

\begin{abstract}
  In this work we present a novel formalization of universal algebra
  in Agda. We show that heterogeneous signatures can be elegantly
  modelled in type-theory using sets indexed by arities to represent
  operations. We prove elementary results of heterogeneous algebras,
  including the proof that the term algebra is initial and the proofs
  of the three isomorphism theorems. We further formalize equational
  theory and prove soundness and completeness. Next we prove that
  every equational class is closed under Isomorphism, Products,
  Subalgebras, and Homomorphic images. At the end, we define (derived)
  signature morphisms, from which we get the contra-variant functor
  between algebras; moreover, we also proved that, under some
  restrictions, the translation of a theory induces a contra-variant
  functor between models.
\end{abstract}

\begin{keywords}
  universal algebra; formalization of mathematics; equational logic;
\end{keywords}

\maketitle

\section{Introduction}

Universal algebra, initiated by \cite{birkhoff-1935}, is the study of
different types of algebraic structures at an abstract level, thus
revealing common results which are valid for all of them and also
allowing for a unified definition of constructions (for example,
products, sub-algebra, congruences). Universal algebra has played a
relevant role in computer science since its earliest days, in
particular the seminal paper of \cite{birkhoff-70} features regular
languages as a prominent example; shortly before, \cite{burstall69}
had proved properties of programs using structural induction, by
conceiving the language as an initial algebra. The ADJ
group \citep[see][]{goguen-adj} promoted multi-sorted algebras as a key
theoretical tool for specifying data
types~\citep{adj-abstract-data-types}, semantics~\citep{goguen-77}, and
compilers~\citep{thatcher1981more}.  More recently,
institutions~\citep{goguen-92}, a generalization of universal algebra,
has been used as a foundation of methodologies and frameworks for
software specification and development~\citep{sannella2012foundations}.

In spite of the rich mathematical theory of heterogeneous algebras
(mostly inherited from the monosorted setting, but with some nuances
as explained by \cite{tarlecki-nuances} and \cite{adamek-2012}), there are few publicly
available formalizations in type theory (which we discuss in the
conclusion).  This situation is to be contrasted with impressive
advances in mechanization of particular algebraic structures as
witnessed, for example, by the proof of the Feit-Thompson theorem in
Coq by Gonthier and his team~\citeyearpar{gonthier2013machine}.

In this work we present an Agda library of multi-sorted universal
algebra aiming both a reader with a background in the area of
algebraic specifications and also the community of type theory.  For
the former, we try to explain enough Agda in order to keep the paper
self-contained; we will recall the most important definitions of
universal algebra. The main contributions of this paper are:
\begin{itemize}
\item the first formalization of basic universal algebra in Agda;
\item the first, to our knowledge, formalization in type theory of
  derived signature morphisms and the reduct algebras induced by them;
\item a novel representation of heterogeneous signatures in
  type theory, where operations are modelled using sets indexed by
  arities; and
\item an independent library of heterogeneous vectors.
\end{itemize}

We formalized the proof that the term algebra is initial and also the
proofs of the three isomorphism theorems; moreover we also define a
deduction system for conditional equational logic and prove its
soundness and completeness with respect to Goguen and Meseguer
semantics~\citeyearpar{GoguenM82}; moreover we prove that the term model (the
term algebra quotiened by the provable equivalence relation) is
initial among models. Then we show that the class of models of any
conditional equational theory is closed by Isomorphisms, Subalgebras
and (arbitrary) products; if the theory is equational then its models
are also closed by homomorphic images. As a nice corollary we can
prove that fields are not (finitely) axiomatizable.

We also showed that the translations of theories arising from derived
signature morphisms induces a contra-variant functor between models.
In the complete development, which is available at
\url{https://cs.famaf.unc.edu.ar/~mpagano/universal-algebra/}, we
include several examples featuring both the use of equational
reasoning and the preservation of models by signature morphisms.

\textit{Outline.} In Sec.~\ref{sec:univ-alg} we introduce the basic
concepts of Universal Algebra: signature, algebras and homomorphisms,
congruences, quotients and subalgebras, the proofs of three
isomorphisms theorems, and the proof of the initiality of the term
algebra.  In Sec.~\ref{sec:eqlog} we define an equational calculus, introducing
concepts of equations, theories, satisfiability and provability,
ending with the Birkhoff proofs of soundness and completeness.  In
Sec.~\ref{sec:trans} we introduce a new representation of (derived) signature
morphisms and reduct algebras (and homomorphisms), and we explore
translation and implication of theories.
% with the property of preservation of models, under some
%restrictions.
Finally, we conclude in Sec.~\ref{sec:conclusions}, discussing the
work done, and pointing out possible future directions.

\section{Universal Algebra}
\label{sec:univ-alg}

In this section we present our formalization in Agda of the core
concepts of heterogeneous universal algebra; in the next two sections
we focus respectively on equational logic and signature morphisms.
Meinke' and Tucker's chapter~\cite{meinke-tucker-1992} is our
reference for heterogeneous universal algebra; we will recall some
definitions and state all the results we formalized. Bove et
al.~\cite{agda-intro} offer a gentle introduction to Agda; we expect
the reader to be familiar with Haskell or some other functional
language.

\subsection{Signature, algebra, and homomorphism}

\paragraph*{Signature}

A \emph{signature} is a pair of sets $(S,F)$, called \textit{sorts}
and \textit{operations} (or \textit{function symbols}) respectively;
each operation is a triple $(f,[s_1,\ldots,s_n],s)$ consisting of a
\textit{name}, its \textit{arity}, and the \textit{target sort} (we
also use the notation $f \colon [s_1,...,s_n] \Rightarrow s$).

In Agda we use dependent records to represent signatures; in dependent
records the type of some field may depend on the value of a previous
one or parameters of the record. Type-theoretically one can take
operations (of a signature) as a family of sets indexed by the arity and target sort
(an indexed family of sets can also be thought as predicates over the
index set, an index satisfies the predicate if its family is
inhabited):
\begin{code}
\>[0]\AgdaKeyword{record}\AgdaSpace{}%
\AgdaRecord{Signature}\AgdaSpace{}%
\AgdaSymbol{:}\AgdaSpace{}%
\AgdaPrimitiveType{Set₁}\AgdaSpace{}%
\AgdaKeyword{where}\<%
\\
\>[0][@{}l@{\AgdaIndent{0}}]%
\>[2]\AgdaKeyword{field}\<%
\\
\>[2][@{}l@{\AgdaIndent{0}}]%
\>[4]\AgdaField{sorts}%
\>[11]\AgdaSymbol{:}\AgdaSpace{}%
\AgdaPrimitiveType{Set}\<%
\\
%
\>[4]\AgdaField{ops}%
\>[11]\AgdaSymbol{:}\AgdaSpace{}%
\AgdaSymbol{(}\AgdaDatatype{List}\AgdaSpace{}%
\AgdaField{sorts}\AgdaSymbol{)}\AgdaSpace{}%
\AgdaOperator{\AgdaFunction{×}}\AgdaSpace{}%
\AgdaField{sorts}\AgdaSpace{}%
\AgdaSymbol{→}\AgdaSpace{}%
\AgdaPrimitiveType{Set}\<%
\end{code}

\noindent \AgdaBound{A}\AgdaSymbol{×}\AgdaBound{B} corresponds to the non-dependent cartesian product
of \AgdaBound{A} and \AgdaBound{B}.

In order to declare a concrete signature one first declares
the set of sorts and the set of operations, which are then bundled
together in a record.  For example, the mono-sorted signature of monoids has a
unique sort, so we use the unit type |⊤| with its sole constructor
|tt|. We define a family indexed on |List ⊤ x ⊤|, with two constructors,
corresponding with the operations: a 0-ary operation |e|, and a binary
operation |∙| (note that constructors can
start with a lower-case letter or any symbol):

\begin{code}
\>[0]\AgdaKeyword{data}\AgdaSpace{}%
\AgdaDatatype{monoid{-}op}\AgdaSpace{}%
\AgdaSymbol{:}\AgdaSpace{}%
\AgdaDatatype{List}\AgdaSpace{}%
\AgdaRecord{⊤}\AgdaSpace{}%
\AgdaOperator{\AgdaFunction{×}}\AgdaSpace{}%
\AgdaRecord{⊤}\AgdaSpace{}%
\AgdaSymbol{→}\AgdaSpace{}%
\AgdaPrimitiveType{Set}\AgdaSpace{}%
\AgdaKeyword{where}\<%
\\
\>[0][@{}l@{\AgdaIndent{0}}]%
\>[3]\AgdaInductiveConstructor{e}\AgdaSpace{}%
\AgdaSymbol{:}\AgdaSpace{}%
\AgdaDatatype{monoid{-}op}\AgdaSpace{}%
\AgdaSymbol{(}\AgdaInductiveConstructor{[]}\AgdaSpace{}%
\AgdaOperator{\AgdaInductiveConstructor{,}}\AgdaSpace{}%
\AgdaInductiveConstructor{tt}\AgdaSymbol{)}\<%
\\
%
\>[3]\AgdaInductiveConstructor{∙}\AgdaSpace{}%
\AgdaSymbol{:}\AgdaSpace{}%
\AgdaDatatype{monoid{-}op}\AgdaSpace{}%
\AgdaSymbol{(}\AgdaInductiveConstructor{tt}\AgdaSpace{}%
\AgdaOperator{\AgdaInductiveConstructor{∷}}\AgdaSpace{}%
\AgdaOperator{\AgdaFunction{[}}\AgdaSpace{}%
\AgdaInductiveConstructor{tt}\AgdaSpace{}%
\AgdaOperator{\AgdaFunction{]}}\AgdaSpace{}%
\AgdaOperator{\AgdaInductiveConstructor{,}}\AgdaSpace{}%
\AgdaInductiveConstructor{tt}\AgdaSymbol{)}\<%
\\
\\
\>[0]\AgdaFunction{monoid{-}sig}\AgdaSpace{}%
\AgdaSymbol{:}\AgdaSpace{}%
\AgdaRecord{Signature}\<%
\\
\>[0]\AgdaFunction{monoid{-}sig}\AgdaSpace{}%
\AgdaSymbol{=}\AgdaSpace{}%
\AgdaKeyword{record}\AgdaSpace{}%
\AgdaSymbol{\{}\AgdaSpace{}%
\AgdaField{sorts}\AgdaSpace{}%
\AgdaSymbol{=}\AgdaSpace{}%
\AgdaRecord{⊤}\AgdaSpace{}%
\AgdaSymbol{;}\AgdaSpace{}%
\AgdaField{ops}\AgdaSpace{}%
\AgdaSymbol{=}\AgdaSpace{}%
\AgdaDatatype{monoid{-}op}\AgdaSpace{}%
\AgdaSymbol{\}}\<%
\end{code}

\noindent The signature of monoid actions has two sorts, one for the
monoid and the other for the set on which the monoid acts.

\begin{code}
\\
\>[0]\AgdaKeyword{data}\AgdaSpace{}%
\AgdaDatatype{actMonₛ}\AgdaSpace{}%
\AgdaSymbol{:}\AgdaSpace{}%
\AgdaPrimitiveType{Set}\AgdaSpace{}%
\AgdaKeyword{where}\<%
\\
\>[0][@{}l@{\AgdaIndent{0}}]%
\>[2]\AgdaInductiveConstructor{mon}\AgdaSpace{}%
\AgdaSymbol{:}\AgdaSpace{}%
\AgdaDatatype{actMonₛ}\<%
\\
%
\>[2]\AgdaInductiveConstructor{set}\AgdaSpace{}%
\AgdaSymbol{:}\AgdaSpace{}%
\AgdaDatatype{actMonₛ}\<%
\\
\\
\>[0]\AgdaKeyword{data}\AgdaSpace{}%
\AgdaDatatype{actMonₒ}\AgdaSpace{}%
\AgdaSymbol{:}\AgdaSpace{}%
\AgdaDatatype{List}\AgdaSpace{}%
\AgdaDatatype{actMonₛ}\AgdaSpace{}%
\AgdaOperator{\AgdaFunction{×}}\AgdaSpace{}%
\AgdaDatatype{actMonₛ}\AgdaSpace{}%
\AgdaSymbol{→}\AgdaSpace{}%
\AgdaPrimitiveType{Set}\AgdaSpace{}%
\AgdaKeyword{where}\<%
\\
\>[0][@{}l@{\AgdaIndent{0}}]%
\>[2]\AgdaInductiveConstructor{e}%
\>[5]\AgdaSymbol{:}\AgdaSpace{}%
\AgdaDatatype{actMonₒ}\AgdaSpace{}%
\AgdaSymbol{(}\AgdaInductiveConstructor{[]}\AgdaSpace{}%
\AgdaOperator{\AgdaInductiveConstructor{,}}\AgdaSpace{}%
\AgdaInductiveConstructor{mon}\AgdaSymbol{)}\<%
\\
%
\>[2]\AgdaInductiveConstructor{∙}%
\>[5]\AgdaSymbol{:}\AgdaSpace{}%
\AgdaDatatype{actMonₒ}\AgdaSpace{}%
\AgdaSymbol{(}\AgdaSpace{}%
\AgdaInductiveConstructor{mon}\AgdaSpace{}%
\AgdaOperator{\AgdaInductiveConstructor{∷}}\AgdaSpace{}%
\AgdaOperator{\AgdaFunction{[}}\AgdaSpace{}%
\AgdaInductiveConstructor{mon}\AgdaSpace{}%
\AgdaOperator{\AgdaFunction{]}}\AgdaSpace{}%
\AgdaOperator{\AgdaInductiveConstructor{,}}\AgdaSpace{}%
\AgdaInductiveConstructor{mon}\AgdaSymbol{)}\<%
\\
%
\>[2]\AgdaInductiveConstructor{*}%
\>[5]\AgdaSymbol{:}\AgdaSpace{}%
\AgdaDatatype{actMonₒ}\AgdaSpace{}%
\AgdaSymbol{(}\AgdaSpace{}%
\AgdaInductiveConstructor{mon}\AgdaSpace{}%
\AgdaOperator{\AgdaInductiveConstructor{∷}}\AgdaSpace{}%
\AgdaOperator{\AgdaFunction{[}}\AgdaSpace{}%
\AgdaInductiveConstructor{set}\AgdaSpace{}%
\AgdaOperator{\AgdaFunction{]}}\AgdaSpace{}%
\AgdaOperator{\AgdaInductiveConstructor{,}}\AgdaSpace{}%
\AgdaInductiveConstructor{set}\AgdaSymbol{)}\<%
\\
\\
\>[0]\AgdaFunction{actMon{-}sig}\AgdaSpace{}%
\AgdaSymbol{:}\AgdaSpace{}%
\AgdaRecord{Signature}\<%
\\
\>[0]\AgdaFunction{actMon{-}sig}\AgdaSpace{}%
\AgdaSymbol{=}\AgdaSpace{}%
\AgdaKeyword{record}\AgdaSpace{}%
\AgdaSymbol{\{}\AgdaSpace{}%
\AgdaField{sorts}\AgdaSpace{}%
\AgdaSymbol{=}\AgdaSpace{}%
\AgdaDatatype{actMonₛ}\AgdaSpace{}%
\AgdaSymbol{;}\AgdaSpace{}%
\AgdaField{ops}\AgdaSpace{}%
\AgdaSymbol{=}\AgdaSpace{}%
\AgdaDatatype{actMonₒ}\AgdaSpace{}%
\AgdaSymbol{\}}\<%

\end{code}

\noindent Defining operations as a family indexed by arities and
target sorts carries some benefits in the use of the library: as in
the above examples, the names of operations are constructors of a
family of datatypes and so it is possible to perform pattern matching
on them. Notice also that infinitary signatures can be represented in
our setting; in fact, all the results are valid for any signature, be
it finite or infinite.

\paragraph{Algebra}
An \emph{algebra} $\mathcal{A}$ for the signature $\Sigma$ consists of
a family of sets indexed by the sorts of $\Sigma$ and a family of
functions indexed by the operations of $\Sigma$. We use
$\mathcal{A}_s$ for the \emph{interpretation} or the \emph{carrier} of
the sort $s$; given an operation
$f \colon [s_1,...,s_n] \Rightarrow s$, the interpretation of $f$ is a
total function
$f_{\mathcal{A}}\colon \mathcal{A}_{s_1} \times ... \times
\mathcal{A}_{s_n} \rightarrow \mathcal{A}_s$. 
We formalize the product $\mathcal{A}_{s_1} \times ... \times
\mathcal{A}_{s_n}$ as \emph{heterogeneous vectors}. The
type of heterogeneous vectors is parameterized by a set \AgdaBound{I}
and a family of sets indexed by \AgdaBound{I}; and is indexed over a
list of \AgdaBound{I}:

\begin{code}
\>[0]\AgdaKeyword{data}\AgdaSpace{}%
\AgdaDatatype{HVec}\AgdaSpace{}%
\AgdaSymbol{\{}\AgdaBound{I}\AgdaSpace{}%
\AgdaSymbol{:}\AgdaSpace{}%
\AgdaPrimitiveType{Set}\AgdaSymbol{\}}%
\>[21]\AgdaSymbol{(}\AgdaBound{A}\AgdaSpace{}%
\AgdaSymbol{:}\AgdaSpace{}%
\AgdaBound{I}\AgdaSpace{}%
\AgdaSymbol{→}\AgdaSpace{}%
\AgdaPrimitiveType{Set}\AgdaSymbol{)}\AgdaSpace{}%
\AgdaSymbol{:}\AgdaSpace{}%
\AgdaDatatype{List}\AgdaSpace{}%
\AgdaBound{I}\AgdaSpace{}%
\AgdaSymbol{→}\AgdaSpace{}%
\AgdaPrimitiveType{Set}\AgdaSpace{}%
\AgdaKeyword{where}\<%
\\
\>[0][@{}l@{\AgdaIndent{0}}]%
\>[2]\AgdaInductiveConstructor{⟨⟩}%
\>[8]\AgdaSymbol{:}%
\>[11]\AgdaDatatype{HVec}\AgdaSpace{}%
\AgdaBound{A}\AgdaSpace{}%
\AgdaInductiveConstructor{[]}\<%
\\
%
\>[2]\AgdaOperator{\AgdaInductiveConstructor{\AgdaUnderscore{}▹\AgdaUnderscore{}}}%
\>[8]\AgdaSymbol{:}%
\>[11]\AgdaSymbol{∀}%
\>[14]\AgdaSymbol{\{}\AgdaBound{i}\AgdaSpace{}%
\AgdaBound{is}\AgdaSymbol{\}}\AgdaSpace{}%
\AgdaSymbol{→}\AgdaSpace{}%
\AgdaBound{A}\AgdaSpace{}%
\AgdaBound{i}\AgdaSpace{}%
\AgdaSymbol{→}\AgdaSpace{}%
\AgdaDatatype{HVec}\AgdaSpace{}%
\AgdaBound{A}\AgdaSpace{}%
\AgdaBound{is}\AgdaSpace{}%
\AgdaSymbol{→}\AgdaSpace{}%
\AgdaDatatype{HVec}\AgdaSpace{}%
\AgdaBound{A}\AgdaSpace{}%
\AgdaSymbol{(}\AgdaBound{i}\AgdaSpace{}%
\AgdaOperator{\AgdaInductiveConstructor{∷}}\AgdaSpace{}%
\AgdaBound{is}\AgdaSymbol{)}\<%
\end{code}


\noindent The first parameter \AgdaBound{I} is implicit (written in braces), which means that Agda
will infer it by unification; notices that the constructor \AgdaInductiveConstructor{\AgdaUnderscore{}▹\AgdaUnderscore{}} also
takes two implicit arguments (we use the notation \AgdaSymbol{∀} to skip their
types). Let \AgdaBound{Σ} be a signature and \AgdaBound{A} \AgdaSymbol{:}\AgdaField{sorts} \AgdaBound{Σ} \AgdaSymbol{→} \AgdaPrimitiveType{Set},
then the product $\mathcal{A}_{s_1} \times ... \times \mathcal{A}_{s_n}$ is
formalized as \AgdaDatatype{HVec} \AgdaBound{A} \AgdaSymbol{[} \AgdaBound{s₁},...,\AgdaBound{sₙ}\AgdaSymbol{]}.

We need one more ingredient to give the formal notion of algebras: the
mathematical definition of carriers assumes an underlying notion of
equality.  In type theory one makes it apparent by using setoids (i.e. 
sets paired with an equivalence relation), which were thoroughly
studied by Barthe et al.~\cite{barthe-setoids-2003}. Setoids are
defined in the standard library \cite{danielsson-agdalib} of
Agda\footnote{Our formalization is based on several concepts defined
  in the standard library.} as a record with three
fields.

\begin{code}
\>[0]\AgdaKeyword{record}\AgdaSpace{}%
\AgdaRecord{Setoid}\AgdaSpace{}%
\AgdaSymbol{:}\AgdaSpace{}%
\AgdaPrimitiveType{Set}\AgdaSpace{}%
\AgdaKeyword{where}\<%
\\
\>[0][@{}l@{\AgdaIndent{0}}]%
\>[2]\AgdaKeyword{field}\<%
\\
\>[2][@{}l@{\AgdaIndent{0}}]%
\>[4]\AgdaField{Carrier}%
\>[18]\AgdaSymbol{:}\AgdaSpace{}%
\AgdaPrimitiveType{Set}\<%
\\
%
\>[4]\AgdaOperator{\AgdaField{\AgdaUnderscore{}≈\AgdaUnderscore{}}}%
\>[18]\AgdaSymbol{:}\AgdaSpace{}%
\AgdaField{Carrier}\AgdaSpace{}%
\AgdaSymbol{→}\AgdaSpace{}%
\AgdaField{Carrier}\AgdaSpace{}%
\AgdaSymbol{→}\AgdaSpace{}%
\AgdaPrimitiveType{Set}\<%
\\
%
\>[4]\AgdaField{isEquivalence}\AgdaSpace{}%
\AgdaSymbol{:}\AgdaSpace{}%
\AgdaRecord{IsEquivalence}\AgdaSpace{}%
\AgdaOperator{\AgdaField{\AgdaUnderscore{}≈\AgdaUnderscore{}}}\<%
\end{code}


\noindent The relation is given as a family of types indexed over a pair
of elements of the carrier (\AgdaBound{a} \AgdaBound{b} : \AgdaField{Carrier} are related if the type
\AgdaBound{a}\AgdaField{≈} \AgdaBound{b} is inhabited); \AgdaRecord{IsEquivalence}\AgdaSpace{}%
\AgdaOperator{\AgdaField{\AgdaUnderscore{}≈\AgdaUnderscore{}}} is again a record whose fields
correspond to the proofs of reflexivity, symmetry, and transitivity.

The finest equivalence relation over any set is given
by the \emph{propositional equality} which only equates each element with
itself, thus we can endow any set with a setoid structure with the function
\AgdaFunction{setoid} : \AgdaPrimitiveType{Set} \AgdaSymbol{→} \AgdaDatatype{Setoid} of the standard library; vice versa, there
is a forgetful functor \AgdaSymbol{∥}\AgdaUnderscore{}\AgdaSymbol{∥} : \AgdaDatatype{Setoid} \AgdaSymbol{→} \AgdaPrimitiveType{Set}
which returns the carrier.

Setoid morphisms are functions which preserve the equality:

\begin{code}
\>[0]\AgdaKeyword{record}\AgdaSpace{}%
\AgdaOperator{\AgdaRecord{\AgdaUnderscore{}⟶\AgdaUnderscore{}}}\AgdaSpace{}%
\AgdaSymbol{(}\AgdaBound{A}\AgdaSpace{}%
\AgdaBound{B}\AgdaSpace{}%
\AgdaSymbol{:}\AgdaSpace{}%
\AgdaRecord{Setoid}\AgdaSymbol{)}\AgdaSpace{}%
\AgdaSymbol{:}\AgdaSpace{}%
\AgdaPrimitiveType{Set}\AgdaSpace{}%
\AgdaKeyword{where}\<%
\\
\>[0][@{}l@{\AgdaIndent{0}}]%
\>[2]\AgdaKeyword{field}\<%
\\
\>[2][@{}l@{\AgdaIndent{0}}]%
\>[4]\AgdaOperator{\AgdaField{\AgdaUnderscore{}⟨\$⟩\AgdaUnderscore{}}}\AgdaSpace{}%
\AgdaSymbol{:}\AgdaSpace{}%
\AgdaOperator{\AgdaFunction{∥}}\AgdaSpace{}%
\AgdaBound{A}\AgdaSpace{}%
\AgdaOperator{\AgdaFunction{∥}}\AgdaSpace{}%
\AgdaSymbol{→}\AgdaSpace{}%
\AgdaOperator{\AgdaFunction{∥}}\AgdaSpace{}%
\AgdaBound{B}\AgdaSpace{}%
\AgdaOperator{\AgdaFunction{∥}}\<%
\\
%
\>[4]\AgdaField{cong}\AgdaSpace{}%
\AgdaSymbol{:}\AgdaSpace{}%
\AgdaSymbol{∀}\AgdaSpace{}%
\AgdaSymbol{\{}\AgdaBound{a}\AgdaSpace{}%
\AgdaBound{a'}\AgdaSymbol{\}}\AgdaSpace{}%
\AgdaSymbol{→}\AgdaSpace{}%
\AgdaOperator{\AgdaField{\AgdaUnderscore{}≈\AgdaUnderscore{}}}\AgdaSpace{}%
\AgdaBound{A}\AgdaSpace{}%
\AgdaBound{a}\AgdaSpace{}%
\AgdaBound{a'}\AgdaSpace{}%
\AgdaSymbol{→}\AgdaSpace{}%
\AgdaOperator{\AgdaField{\AgdaUnderscore{}≈\AgdaUnderscore{}}}\AgdaSpace{}%
\AgdaBound{B}\AgdaSpace{}%
\AgdaSymbol{(}\AgdaOperator{\AgdaField{\AgdaUnderscore{}⟨\$⟩\AgdaUnderscore{}}}\AgdaSpace{}%
\AgdaBound{a}\AgdaSymbol{)}\AgdaSpace{}%
\AgdaSymbol{(}\AgdaOperator{\AgdaField{\AgdaUnderscore{}⟨\$⟩\AgdaUnderscore{}}}\AgdaSpace{}%
\AgdaBound{a'}\AgdaSymbol{)}\<%
\end{code}


\noindent Notice that \AgdaOperator{\AgdaRecord{\AgdaUnderscore{}⟶\AgdaUnderscore{}}}\AgdaSpace{} is a record parameterized on two setoids.
The first field is the function, by declaring it mixfix one can
write \AgdaBound{f} \AgdaField{⟨\$⟩} \AgdaBound{a} when \AgdaBound{f} : \AgdaBound{A} \AgdaSymbol{⟶} \AgdaBound{B}
and \AgdaBound{a} : \AgdaSymbol{∥} \AgdaBound{A} \AgdaSymbol{∥}; the second field is
given by a function mapping equivalence proofs on the source setoid to
equivalence proofs on the target. Setoid morphisms will be used to
give the interpretation of operations.

Let \AgdaBound{A} : \AgdaBound{I} \AgdaSymbol{→} \AgdaPrimitiveType{Set} be a family of sets and
\AgdaBound{P} : \AgdaSymbol{\{}\AgdaBound{i} : \AgdaBound{I}\AgdaSymbol{\}} \AgdaSymbol{→} \AgdaBound{A} \AgdaBound{i}
\AgdaSymbol{→} \AgdaPrimitiveType{Set} a
family of predicates, we let \AgdaBound{P}\AgdaSymbol{*} : \AgdaSymbol{∀} \AgdaSymbol{\{}\AgdaBound{is}\AgdaSymbol{\}}
\AgdaSymbol{→} \AgdaDatatype{HVec} \AgdaBound{A} \AgdaBound{is} \AgdaSymbol{→} \AgdaPrimitiveType{Set} be the
point-wise extension of \AgdaBound{P} over heterogeneous vectors. We also use
the point-wise extension to define the setoid of heterogeneous vectors
given a family of setoids \AgdaBound{A} : \AgdaBound{I} \AgdaSymbol{→} \AgdaDatatype{Setoid} and write
\AgdaBound{A} \AgdaSymbol{✳} \AgdaBound{is} for the
setoid of heterogeneous vectors with index \AgdaBound{is}. Algebras are
formalized as records parameterized on the signature.


\begin{code}
\>[0]\AgdaKeyword{record}\AgdaSpace{}%
\AgdaRecord{Algebra}\AgdaSpace{}%
\AgdaSymbol{(}\AgdaBound{Σ}\AgdaSpace{}%
\AgdaSymbol{:}\AgdaSpace{}%
\AgdaRecord{Signature}\AgdaSymbol{)}\AgdaSpace{}%
\AgdaSymbol{:}\AgdaSpace{}%
\AgdaPrimitiveType{Set}\AgdaSpace{}%
\AgdaKeyword{where}\<%
\\
\>[0][@{}l@{\AgdaIndent{0}}]%
\>[2]\AgdaKeyword{field}\<%
\\
\>[2][@{}l@{\AgdaIndent{0}}]%
\>[4]\AgdaOperator{\AgdaField{\AgdaUnderscore{}⟦\AgdaUnderscore{}⟧ₛ}}%
\>[12]\AgdaSymbol{:}\AgdaSpace{}%
\AgdaField{sorts}\AgdaSpace{}%
\AgdaBound{Σ}\AgdaSpace{}%
\AgdaSymbol{→}\AgdaSpace{}%
\AgdaRecord{Setoid}\AgdaSpace{}%
\\
%
\>[4]\AgdaOperator{\AgdaField{\AgdaUnderscore{}⟦\AgdaUnderscore{}⟧ₒ}}%
\>[13]\AgdaSymbol{:}\AgdaSpace{}%
\AgdaSymbol{∀}\AgdaSpace{}%
\AgdaSymbol{\{}\AgdaBound{ar}\AgdaSpace{}%
\AgdaBound{s}\AgdaSymbol{\}}\AgdaSpace{}%
\AgdaSymbol{→}\AgdaSpace{}%
\AgdaField{ops}\AgdaSpace{}%
\AgdaBound{Σ}\AgdaSpace{}%
\AgdaSymbol{(}\AgdaBound{ar}\AgdaSpace{}%
\AgdaOperator{\AgdaInductiveConstructor{,}}\AgdaSpace{}%
\AgdaBound{s}\AgdaSymbol{)}\AgdaSpace{}%
\AgdaSymbol{→}\AgdaSpace{}%
\AgdaOperator{\AgdaField{\AgdaUnderscore{}⟦\AgdaUnderscore{}⟧ₛ}}\AgdaSpace{}%
\AgdaOperator{\AgdaFunction{✳}}\AgdaSpace{}%
\AgdaBound{ar}\AgdaSpace{}%
\AgdaOperator{\AgdaFunction{⟶}}\AgdaSpace{}%
\AgdaOperator{\AgdaField{\AgdaUnderscore{}⟦\AgdaUnderscore{}⟧ₛ}}\AgdaSpace{}%
\AgdaBound{s}\<%
\end{code}



\noindent If \AgdaBound{A} is an algebra for the signature \AgdaFunction{actMon{-}sig}, then
\AgdaBound{A} \AgdaField{⟦ \AgdaInductiveConstructor{tt} ⟧ₛ} is the carrier, \AgdaBound{A}
\AgdaField{⟦ \AgdaInductiveConstructor{e} ⟧ₒ}  and
\AgdaBound{A}
\AgdaField{⟦ \AgdaInductiveConstructor{∙} ⟧ₒ} are the interpretations
of the operations. We invite the interested reader to browse the examples to
see algebras for the signatures we have shown.

\paragraph*{Homomorphism}
Let $\Sigma$ be a signature and let $\mathcal{A}$ and $\mathcal{B}$ be
algebras for $\Sigma$. A \emph{homomorphism} $h$ from $\mathcal{A}$ to
$\mathcal{B}$ is a family of functions indexed by the sorts
$h_s : \mathcal{A}_s \rightarrow \mathcal{B}_s$, such that for each
operation $f : [s_1,...,s_n] \Rightarrow s$, the following holds:
\begin{equation}
  h_s(f_{\mathcal{A}}(a_1,...,a_n)) = f_{\mathcal{B}}(h_{s_1}\,a_1,...,h_{s_n}\,a_n)\label{eq:homcond}
\end{equation}
\noindent Notice that this is a condition over the family of
functions.

In order to formalize homomorphisms we first introduce a
notation for families of setoid morphisms indexed over sorts:


\begin{code}
\>[0]\AgdaOperator{\AgdaFunction{\AgdaUnderscore{}⟿\AgdaUnderscore{}}}\AgdaSpace{}%
\AgdaSymbol{:}\AgdaSpace{}%
\AgdaSymbol{∀}\AgdaSpace{}%
\AgdaSymbol{\{}\AgdaBound{Σ}\AgdaSymbol{\}}\AgdaSpace{}%
\AgdaSymbol{→}\AgdaSpace{}%
\AgdaRecord{Algebra}\AgdaSpace{}%
\AgdaBound{Σ}\AgdaSpace{}%
\AgdaSymbol{→}\AgdaSpace{}%
\AgdaRecord{Algebra}\AgdaSpace{}%
\AgdaBound{Σ}\AgdaSpace{}%
\AgdaSymbol{→}\AgdaSpace{}%
\AgdaPrimitiveType{Set}\<%
\\
\>[0]\AgdaOperator{\AgdaFunction{\AgdaUnderscore{}⟿\AgdaUnderscore{}}}\AgdaSpace{}%
\AgdaSymbol{\{}\AgdaBound{Σ}\AgdaSymbol{\}}\AgdaSpace{}%
\AgdaBound{A}\AgdaSpace{}%
\AgdaBound{B}\AgdaSpace{}%
\AgdaSymbol{=}\AgdaSpace{}%
\AgdaSymbol{(}\AgdaBound{s}\AgdaSpace{}%
\AgdaSymbol{:}\AgdaSpace{}%
\AgdaField{sorts}\AgdaSpace{}%
\AgdaBound{Σ}\AgdaSymbol{)}\AgdaSpace{}%
\AgdaSymbol{→}\AgdaSpace{}%
\AgdaBound{A}\AgdaSpace{}%
\AgdaOperator{\AgdaField{⟦}}\AgdaSpace{}%
\AgdaBound{s}\AgdaSpace{}%
\AgdaOperator{\AgdaField{⟧ₛ}}\AgdaSpace{}%
\AgdaOperator{\AgdaFunction{⟶}}\AgdaSpace{}%
\AgdaBound{B}\AgdaSpace{}%
\AgdaOperator{\AgdaField{⟦}}\AgdaSpace{}%
\AgdaBound{s}\AgdaSpace{}%
\AgdaOperator{\AgdaField{⟧ₛ}}\<%
\end{code}


\noindent We make explicit the implicit parameter \AgdaBound{Σ} because
otherwise \AgdaField{sorts} \AgdaBound{Σ} does not make sense.\footnote{In the library we
  use modules in order to avoid the repetition of the parameters \AgdaBound{Σ},
  \AgdaBound{A}, and \AgdaBound{B}.} To enforce \eqref{eq:homcond} we also define a
predicate over families of setoids morphisms:



\begin{code}
\>[2]\AgdaFunction{homCond}\AgdaSpace{}%
\AgdaSymbol{:}\AgdaSpace{}%
\AgdaSymbol{∀}\AgdaSpace{}%
\AgdaSymbol{\{}\AgdaBound{Σ}\AgdaSymbol{\}}\AgdaSpace{}%
\AgdaSymbol{\{}\AgdaBound{A}\AgdaSpace{}\AgdaBound{B}\AgdaSymbol{\}}\AgdaSpace{}%
\AgdaSymbol{→}\AgdaSpace{}%
\AgdaBound{A}\AgdaSpace{}\AgdaOperator{\AgdaFunction{⟿}}\AgdaSpace{}\AgdaBound{B}\AgdaSpace{}%
\AgdaSymbol{→}\AgdaSpace{}%
\AgdaPrimitiveType{Set}\<%
\\
%
\>[2]\AgdaFunction{homCond}\AgdaSpace{}%
\AgdaSymbol{\{}\AgdaArgument{A}%
\>[142I]\AgdaSymbol{=}\AgdaSpace{}%
\AgdaBound{A}\AgdaSymbol{\}}\AgdaSpace{}%
\AgdaSymbol{\{}\AgdaBound{B}\AgdaSymbol{\}}\AgdaSpace{}%
\AgdaBound{h}\AgdaSpace{}%
\AgdaSymbol{=}\AgdaSpace{}\<%
\\
\>[.]\<[142I]%
\>[14]\AgdaSymbol{∀}\AgdaSpace{}%
\AgdaSymbol{\{}\AgdaBound{ar}\AgdaSpace{}\AgdaBound{s}\AgdaSymbol{\}}\AgdaSpace{}%
\AgdaSymbol{(}\AgdaBound{f}\AgdaSpace{}%
\AgdaSymbol{:}\AgdaSpace{}%
\AgdaField{ops}\AgdaSpace{}%
\AgdaBound{Σ}\AgdaSpace{}%
\AgdaSymbol{(}\AgdaBound{ar}\AgdaSpace{}%
\AgdaOperator{\AgdaInductiveConstructor{,}}\AgdaSpace{}%
\AgdaBound{s}\AgdaSymbol{)}\AgdaSpace{}%
\AgdaSymbol{)}\AgdaSpace{}%
\AgdaSymbol{→}\AgdaSpace{}%
\AgdaSymbol{(}\AgdaBound{as}\AgdaSpace{}%
\AgdaSymbol{:}\AgdaSpace{}%
\AgdaBound{A}\AgdaSpace{}%
\AgdaOperator{\AgdaFunction{⟦}}\AgdaSpace{}%
\AgdaBound{ar}\AgdaSpace{}%
\AgdaOperator{\AgdaFunction{⟧ₛ*}}\AgdaSymbol{)}\AgdaSpace{}%
\AgdaSymbol{→}\<%
\\
\>[.]\<[142I]%
\>[14]\AgdaSymbol{(}\AgdaBound{h}\AgdaSpace{}%
\AgdaBound{s}\AgdaSpace{}%
\AgdaOperator{\AgdaField{⟨\$⟩}}\AgdaSpace{}%
\AgdaSymbol{(}\AgdaBound{A}\AgdaSpace{}%
\AgdaOperator{\AgdaField{⟦}}\AgdaSpace{}%
\AgdaBound{f}\AgdaSpace{}%
\AgdaOperator{\AgdaField{⟧ₒ}}\AgdaSpace{}%
\AgdaOperator{\AgdaField{⟨\$⟩}}\AgdaSpace{}%
\AgdaBound{as}\AgdaSymbol{))}\AgdaSpace{}%
\AgdaOperator{\AgdaFunction{≈ₛ}}\AgdaSpace{}%
\AgdaSymbol{(}\AgdaBound{B}\AgdaSpace{}%
\AgdaOperator{\AgdaField{⟦}}\AgdaSpace{}%
\AgdaBound{f}\AgdaSpace{}%
\AgdaOperator{\AgdaField{⟧ₒ}}\AgdaSpace{}%
\AgdaOperator{\AgdaField{⟨\$⟩}}\AgdaSpace{}%
\AgdaSymbol{(}\AgdaFunction{map}\AgdaSpace{}%
\AgdaBound{h}\AgdaSpace{}%
\AgdaBound{as}\AgdaSymbol{))}\<%
\end{code}


\noindent where \AgdaUnderscore{}\AgdaOperator{\AgdaFunction{≈ₛ}}\AgdaUnderscore{} is the equivalence relation of the setoid
\AgdaBound{B} \AgdaField{⟦ \AgdaBound{s} ⟧ₛ} and \AgdaFunction{map} \AgdaBound{h} is the obvious extension of \AgdaBound{h} over vectors.
A homomorphism is a record parameterized by the source and target algebras

\begin{code}
\>[2]\AgdaKeyword{record}\AgdaSpace{}%
\AgdaRecord{Homo}\AgdaSpace{}%
\AgdaSymbol{\{}\AgdaBound{Σ}\AgdaSymbol{\}}\AgdaSpace{}%
\AgdaSymbol{(}\AgdaBound{A}\AgdaSpace{}\AgdaBound{B}
\AgdaSpace{}\AgdaSymbol{:}\AgdaSpace{}\AgdaRecord{Algebra}\AgdaSpace{}\AgdaBound{Σ}\AgdaSymbol{)}\AgdaSpace{}%
\AgdaSymbol{:}\AgdaSpace{}%
\AgdaPrimitiveType{Set}\AgdaSpace{}%
\AgdaKeyword{where}\<%
\\
\>[2][@{}l@{\AgdaIndent{0}}]%
\>[4]\AgdaKeyword{field}\<%
\\
\>[4][@{}l@{\AgdaIndent{0}}]%
\>[6]\AgdaOperator{\AgdaField{′\AgdaUnderscore{}′}}%
\>[13]\AgdaSymbol{:}\AgdaSpace{}%
\AgdaOperator{\AgdaFunction{\AgdaUnderscore{}⟿\AgdaUnderscore{}}}\<%
\\
%
\>[6]\AgdaField{cond}%
\>[12]\AgdaSymbol{:}\AgdaSpace{}%
\AgdaFunction{homCond}\AgdaSpace{}%
\AgdaOperator{\AgdaField{′\AgdaUnderscore{}′}}\AgdaSpace{}\<%
\end{code}


\noindent As expected, we have the identity homomorphism
\AgdaFunction{Idₕ}\AgdaSpace{}\AgdaBound{A}\AgdaSpace\AgdaSymbol{:}\AgdaRecord{Homo}
\AgdaSpace{}\AgdaBound{A}\AgdaSpace{}\AgdaBound{A} and
the composition \AgdaBound{G}\AgdaSpace{}\AgdaOperator{\AgdaSymbol{∘ₕ}}\AgdaSpace{}\AgdaBound{F}\AgdaSpace{}
\AgdaSymbol{:}\AgdaSpace{}\AgdaRecord{Homo}
\AgdaSpace{}\AgdaBound{A}\AgdaSpace{}\AgdaBound{C} of homomorphisms
\AgdaBound{F}\AgdaSpace{}\AgdaSymbol{:}\AgdaSpace{}
\AgdaRecord{Homo}\AgdaSpace{}\AgdaBound{A}\AgdaSpace{}\AgdaBound{B}
and
\AgdaBound{G}\AgdaSpace{}\AgdaSymbol{:}\AgdaSpace{}
\AgdaRecord{Homo}\AgdaSpace{}\AgdaBound{B}\AgdaSpace{}\AgdaBound{C}.
It is also expected that
\AgdaBound{F}\AgdaSpace{}\AgdaOperator{\AgdaSymbol{∘ₕ}}\AgdaSpace{}
\AgdaFunction{Idₕ}\AgdaSpace{}\AgdaBound{A}\AgdaSpace and
\AgdaBound{F}  are
equal in some sense. Since Agda is based on an
intensional type theory, we cannot take the definitional equality
(which distinguishes \AgdaFunction{id} from \AgdaSymbol{λ} \AgdaBound{n} \AgdaSymbol{→} \AgdaBound{n} \AgdaFunction{+} \AgdaInductiveConstructor{0} as functions
on naturals); instead, we equate setoid morphisms 
whenever their function parts are extensionally equal:


\begin{code}
\>[0]\AgdaOperator{\AgdaFunction{\AgdaUnderscore{}≈ₑₓₜ\AgdaUnderscore{}}}\AgdaSpace{}%
\AgdaSymbol{:}\AgdaSpace{}%
\AgdaSymbol{(}\AgdaBound{f}\AgdaSpace{}\AgdaBound{g}\AgdaSpace{}\AgdaSymbol{:}\AgdaSpace{}
\AgdaBound{A}\AgdaOperator{\AgdaFunction{⟶}}\AgdaBound{B}\AgdaSymbol{)}\AgdaSpace{}
\AgdaSymbol{→}\AgdaSpace{}\AgdaPrimitiveType{Set}\AgdaSpace{}\<%
\\
%
\>[0]\AgdaBound{f}\AgdaSpace{}%
\AgdaOperator{\AgdaFunction{≈ₑₓₜ}}\AgdaSpace{}%
\AgdaBound{g}%
\>[8]\AgdaSymbol{=}\AgdaSpace{}%
\AgdaSymbol{∀}\AgdaSpace{}%
\AgdaSymbol{(}\AgdaBound{a}\AgdaSpace{}%
\AgdaSymbol{:}\AgdaSpace{}%
\AgdaOperator{\AgdaFunction{∥}}\AgdaSpace{}%
\AgdaBound{A}\AgdaSpace{}%
\AgdaOperator{\AgdaFunction{∥}}\AgdaSymbol{)}\AgdaSpace{}%
\AgdaSymbol{→}\AgdaSpace{}%
\AgdaSymbol{(}\AgdaBound{f}\AgdaSpace{}%
\AgdaOperator{\AgdaField{⟨\$⟩}}\AgdaSpace{}%
\AgdaBound{a}\AgdaSymbol{)}\AgdaSpace{}%
\AgdaOperator{\AgdaFunction{≈B}}\AgdaSpace{}%
\AgdaSymbol{(}\AgdaBound{g}\AgdaSpace{}%
\AgdaOperator{\AgdaField{⟨\$⟩}}\AgdaSpace{}%
\AgdaBound{a}\AgdaSymbol{)}\<%
\end{code}


\noindent Two homomorphisms are equal when their corresponding setoid
morphisms are extensionally equal:


\begin{code}
\>[2]\AgdaOperator{\AgdaFunction{\AgdaUnderscore{}≈ₕ\AgdaUnderscore{}}}%
\>[8]\AgdaSymbol{:}\AgdaSpace{}%
\AgdaSymbol{∀}\AgdaSpace{}%
\AgdaSymbol{\{}\AgdaBound{Σ}\AgdaSymbol{\}}\AgdaSpace{}%
\AgdaSymbol{\{}\AgdaBound{A}\AgdaSpace{}\AgdaBound{B}\AgdaSymbol{\}}\AgdaSpace{}%
\AgdaSymbol{→}\AgdaSpace{}%
\AgdaSymbol{(}\AgdaBound{H}\AgdaSpace{}%
\AgdaBound{G}\AgdaSpace{}%
\AgdaSymbol{:}\AgdaSpace{}%
\AgdaRecord{Homo}\AgdaSpace{}\AgdaBound{A}\AgdaSpace{}\AgdaBound{B}\AgdaSymbol{)}\AgdaSpace{}%
\AgdaSymbol{→}\AgdaSpace{}%
\AgdaPrimitiveType{Set}\AgdaSpace{}%
\AgdaSymbol{\AgdaUnderscore{}}\<%
\\
%
\>[2]\AgdaBound{H}\AgdaSpace{}%
\AgdaOperator{\AgdaFunction{≈ₕ}}\AgdaSpace{}%
\AgdaBound{G}\AgdaSpace{}%
\AgdaSymbol{=}\AgdaSpace{}%
\AgdaSymbol{(}\AgdaBound{s}\AgdaSpace{}%
\AgdaSymbol{:}\AgdaSpace{}%
\AgdaField{sorts}\AgdaSpace{}%
\AgdaBound{Σ}\AgdaSymbol{)}\AgdaSpace{}%
\AgdaSymbol{→}\AgdaSpace{}%
\AgdaSymbol{(}\AgdaOperator{\AgdaField{′}}\AgdaSpace{}%
\AgdaBound{H}\AgdaSpace{}%
\AgdaOperator{\AgdaField{′}}\AgdaSpace{}%
\AgdaBound{s}\AgdaSymbol{)}\AgdaSpace{}%
\AgdaOperator{\AgdaFunction{≈ₑₓₜ}}\AgdaSpace{}%
\AgdaSymbol{(}\AgdaOperator{\AgdaField{′}}\AgdaSpace{}%
\AgdaBound{G}\AgdaSpace{}%
\AgdaOperator{\AgdaField{′}}\AgdaSpace{}%
\AgdaBound{s}\AgdaSymbol{)}\<%
\end{code}


\noindent With respect to this equality, it is straightforward to
prove the associativity of the composition \AgdaUnderscore{}\AgdaOperator{\AgdaSymbol{∘ₕ}}\AgdaUnderscore{} and that
\AgdaFunction{Idₕ} is
the identity for the composition.



\subsection{Quotient and subalgebras}
In order to prove the more basic results of universal algebra, we need
to formalize subalgebras, congruence relations, and quotients.

% \paragraph{Subalgebra}

% A subalgebra $\mathcal{B}$ of an algebra $\mathcal{A}$ consists of a
% family of subsets $\mathcal{B}_s \subseteq \mathcal{A}_s$, that are
% closed under the interpretation of operations; that is, for every
% $ f : [s_1, \ldots,s_n] \Rightarrow s$ the following condition holds
% \begin{equation}
% (a_1,\ldots,a_n) \in \mathcal{B}_{s_1} \times \cdots \times\mathcal{B}_{s_n}   \text{ implies }
%   f_\mathcal{A}(a_1,\ldots,a_n) \in \mathcal{B}_{s} \enspace .
%  \label{eq:opclosed}
% \end{equation} 
% \noindent As shown by Salvesen and Smith \cite{salvesen-subsets},
% subsets cannot be added as a construction in intensional type theory
% because they lack desirable properties. If |A : Set| and |P : A → Set|
% is a predicate over |A|, then one can represent the subset containing the
% elements on |A| that satisfy |P| as the dependent sum\footnote{Do not confuse
% the syntax |Σ[_∈_]_| of dependent sum, with a variable |Σ : Signature|}
% |Σ[ a ∈ A ] P| whose inhabitants are
% pairs |(a , p)| where |a : A| and |p : P a|.\comment{This is not so
%   pleasant as there can be several proofs of |P a|.} Let us consider a
% setoid |A| and a predicate on its carrier |P : ∥ A ∥ → Set|; first
% notice that we can lift the subset construction to setoids, defining
% the equivalence relation |(a , q) ≈ (a' , q')| iff |a ≈ a'|.
% Moreover, we might assume that |P| is \emph{well-defined},
% which means that |a ≈A a'| and |P a| imply
% |P a'|.
% \begin{spec}
%   WellDef : (A : Setoid) → (P : ∥ A ∥ → Set) → Set
%   WellDef A P = ∀ {a a'} → a ≈A a' → P a → P a'
% \end{spec}
% \noindent A family of well-defined predicates will induce a subalgebra;
% but we still need to formalize the condition \eqref{eq:opclosed}.  Let
% |Σ| be a signature and |A| be an algebra for |Σ|.
% \begin{spec}
%     opClosed : (P : (s : sorts Σ) → ∥ A ⟦ s ⟧ₛ∥ → Set) → Set
%     opClosed P = ∀ {ar s} (f : ops Σ (ar , s)) → (P * ⟨→⟩ P s) (A ⟦ f ⟧ₒ ⟨$⟩_)
% \end{spec}
% \noindent |(Q ⟨→⟩ R) f| can be read as the pre-condition |Q| implies
% post-condition |R| after applying |f|; so |opClosed P f| asserts that if a vector |a*|
% satisfies the predicate |P|, then the application of the interpretation |A ⟦ f ⟧ₒ|
% to |a*| satisfies |P|, according to Eq.~\eqref{eq:opclosed}.
% In summary, given an algebra
% |A| for the signature |Σ| and a family |P| of predicates, such that |P
% s| is well-defined for every sort |s| and |P| is |opClosed|, we can
% define the |SubAlgebra A P| 
% \begin{spec}
% SubAlgebra : ∀ {Σ} A P → WellDef P → opClosed P → Algebra Σ
% \end{spec}
% \noindent In the subalgebra, an operation |f| is interpreted by
% applying the interpretation of |f| in |A| to the first components of
% the argument (and use the fact that |P| is op-closed to show that
% the resulting value satisfies the predicate of the target sort).

% \paragraph{Congruence and Quotients}
% A \emph{congruence} on a $\Sigma$-algebra
% $\mathcal{A}$ is a family
% $Q$ of equivalence relations indexed by sorts, and each of them is
% closed under the operations of the algebra. This condition is called
% \emph{substitutivity} and can be formalized using the point-wise
% extension of $Q$ over vectors: for every operation $ f : [s_1,
% \ldots,s_n] \Rightarrow s$
% \begin{equation}
%   (\vec{a},\vec{b}) \in Q_{s_1} \times \cdots \times Q_{s_n} \text{ implies }
%  (f_{\mathcal{A}}(\vec{a}) , f_{\mathcal{A}}(\vec{b})) \in Q_s\label{eq:congcond}
% \end{equation} 

% As with predicates, we say that a binary relation over a setoid is
% well-defined if it is preserved by the setoid equality; this notion
% can be extended over families of relations in the obvious way. In our
% formalization, a congruence on an algebra |A| is a family |Q| of
% well-defined, equivalence relations. The substitutivity condition
% \eqref{eq:congcond} is aptly captured by the generalized containment
% operator |_=[_]⇒_| of the standard library, where |P =[ f ]⇒ Q| if,
% for all |a,b ∈ A|, |(a,b) ∈ P| implies |(f a, f b) ∈ Q|.
% \begin{spec}
% record Congruence (A : Algebra Σ) : Set where
%   field
%     rel : (s : sorts Σ) → (∥ A ⟦ s ⟧ₛ ∥ → ∥ A ⟦ s ⟧ₛ ∥ → Set)
%     welldef : (s : sorts Σ) → WellDefBin (rel s)
%     cequiv : (s : sorts Σ) → IsEquivalence (rel s)
%     csubst : ∀ {ar s} → (f : ops Σ (ar , s)) → rel * =[ A ⟦ f ⟧ₒ ⟨$⟩_  ]⇒ rel s
% \end{spec}

% Given a congruence $Q$ over the algebra $\mathcal{A}$, we can obtain a
% new algebra, the \emph{quotient algebra}, by interpreting the sort $s$
% as the set of equivalence classes $\mathcal{A}_s / Q$; the condition
% \eqref{eq:congcond} ensures that the operation $ f : [s_1, \ldots,s_n]
% \Rightarrow s$ can be interpreted as the function mapping the vector
% $([a_1],\ldots,[a_n])$ of equivalence classes into the class $[
% f_\mathcal{A}(a_1,\ldots,a_n)]$. In Agda, we take the same carriers
% from |A| and use |Q s| as the equivalence relation over |∥ A ⟦ s ⟧ₛ
% ∥|; operations are interpreted just as in |A| and the congruence proof
% is given by |csubst Q|.

% \paragraph{Isomorphism Theorems} The definitions of subalgebras,
% quotients, and epimorphisms (surjective homomorphisms) are related by
% the three isomorphism theorems. Although there is some small overhead
% by the coding of subalgebras, the proofs follow very close what one would
% do in paper. For proving these results we also defined the
% \emph{kernel} and the \emph{homomorphic} image of homomorphisms.

% \begin{theorem}[First isomorphism theorem] If $h : \alg{A} \rightarrow \alg{B}$
% is an epimorphism, then $\alg{A} /\! \mathop{ker} h \simeq \alg{B}$.
% \end{theorem}
% \noindent Remember that the quotient $\alg A /\! \mathop{ker} h$ has
% the same carrier as $\alg A$, so $h$ counts as the underlying function
% and it respects the equivalence relation $\mathop{ker} h$ by
% definition. Clearly $h$ is surjective and its injectivity is obvious.

% \begin{theorem}[Second isomorphism theorem] If $\phi,\psi$ are congruences over $\alg A$,
% such that $\psi \subseteq \phi$, then $(\alg A / \phi) \simeq (\alg A / \psi)/(\phi / \psi)$. 
% \end{theorem}

% \noindent In order to prove this theorem, we first prove that
% $\phi / \psi$ is a congruence over $\alg A / \psi$: it suffices to
% prove the well-definedness of $\phi / \psi$, \ie that
% $(a,c) \in \psi$, $(b,d) \in \psi$, and $(a,b) \in \phi$ imply
% $(c,d) \in \phi$; an obvious consequence of $\psi \subseteq
% \phi$. Notice that the underlying carriers are the same in both cases:
% those of $\alg A$, so the identity function is the mediating
% isomorphism and the proof that it satisfies the homomorphism condition
% is trivial.

% \begin{theorem}[Third isomorphism theorem] Let $\alg B$ be a
% subalgebra of $\alg A$ and $\phi$ be a congruence over $\alg A$. Let
% $[\alg B]^{\phi}=\{K \in A / \phi : K \cap B \not= \emptyset\}$ and
% let $\phi_B$ be the restriction of $\phi$ to $\alg B$, then
% \begin{enumerate*}[label= (\roman*),itemjoin={}]
% \item$\phi_B$ is a congruence over $\alg B$;
% \item$[\alg B]^{\phi}$ is a subalgebra of~$\alg A$; and,
% \item$[\alg B]^{\phi} \simeq \alg B / \phi_B$.
% \end{enumerate*}
% \end{theorem}
% \noindent First we define the \emph{trace} of the congruence $\phi$ on
% the subalgebra $\alg B$ as the restriction of $\phi$ on $\alg B$;
% proving that it is a congruence over $\alg B$ involves some
% bureaucracy (remember that an element of a subalgebra is a pair
% $(a, p)$ such that $a \in A$ and $p$ is the proof that $a$ satisfies
% the predicate defining $B$). For the second item, we model
% $[\alg B]^{\phi}$ as a predicate over $\alg A$; it is satisfied by
% $a \in A$ if there is some $b \in B$ such that $(a,b) \in \phi$. The
% well-definedness of this predicate is easy (assuming $(a,a') \in \phi$
% and $b\in B$ with $(a,b) \in \phi$, one can easily prove that
% $(a',b) \in \phi$, thus $b$ is also the witness for proving that $a'$
% satisfies the predicate). To prove that the predicate is closed under
% the operations we take a vector of triples $(as,bs,ps)$ consisting of
% a vector of elements in $A$, a vector of elements in $B$, and the
% proofs $ps$ proving that $(as_i,bs_i)\in\phi$. Let $f$ be an
% operation, since $B$ is closed we know $f(b_1,\ldots,b_n)\in B$ and
% because $\phi$ is also closed we deduce
% $(f(a_1,\ldots,a_n),f(b_1,\ldots,b_n))\in\phi$. Finally, the
% underlying function witnessing the isomorphism
% $[\alg B]^{\phi} \simeq \alg B / \phi_B$ is given by composing the
% second projection with the first projection, thus getting an element
% in $B$.

% \subsection{The Term Algebra is initial}

% A $\Sigma$-algebra $\mathcal{A}$ is called \emph{initial} if for any
% $\Sigma$-algebra $\mathcal{B}$ there exists exactly one homomorphism
% from $\mathcal{A}$ to $\mathcal{B}$. We give an abstract definition of
% this universal property, existence of a unique element, for any set
% |A| and any relation |R|
% \begin{spec}
% hasUnique {A} _≈_ = A × (∀ a a' → a ≈ a')
% \end{spec}
% \noindent and initiality can be formalized directly:
% \begin{spec}
% Initial : ∀ {Σ} → Algebra Σ → Set
% Initial {Σ} A = ∀ (B : Algebra Σ) → hasUnique (_≈ₕ_ A B)
% \end{spec}
% Given a signature $\Sigma$ we can define the \emph{term algebra}
% $\mathcal{T}$, whose carriers are sets of well-typed words built up
% from the function symbols.  Sometimes this universe is called the
% \emph{Herbrand Universe} and is inductively defined:
% \begin{prooftree}
% \AxiomC{$t_1 \in \mathcal{T}_{s_1}$}
% \AxiomC{$\cdots$}
% \AxiomC{$t_n \in \mathcal{T}_{s_n}$}
% \RightLabel{$f : [s_1,...,s_{n}] \Rightarrow s$}
% \TrinaryInfC{$f\,(t_1,...,t_{n}) \in \mathcal{T}_s$}
% \end{prooftree}
% \noindent This inductive definition can be written directly in Agda:
% \begin{spec}
%   data HU {Σ : Signature} : (s : sorts Σ) → Set where
%     term : ∀  {ar s} → (f : ops Σ (ar ↦ s)) → HVec HU ar → HU s
% \end{spec}
% \noindent We use propositional equality to turn each |HUₛ| into a
% setoid, thus completing the interpretation of sorts. To interpret an
% operation $f \colon [s_1,\ldots,s_n] \Rightarrow s$ we map the vector
% |⟨t₁,…,tₙ⟩ : HVec HU [s₁,…,sₙ]| to |term f ⟨t₁,…,tₙ⟩|; we omit
% the proof of |cong|, which is too long and tedious to be
% shown.
% \begin{spec}
%   ∣T∣ : (Σ : Signature) → Algebra Σ
%   ∣T∣ Σ = record  { _⟦_⟧ₛ = setoid ∘ (HU {Σ}) ; _⟦_⟧ₒ  = ∣_|ₒ }
%     where | f ∣ₒ = record { _⟨$⟩_ = term f ; cong = ... }
% \end{spec}
% \noindent Terms can be interpreted in any algebra
% $\mathcal{A}$, yielding an homomorphism $h_A \colon \mathcal{T}
% \to \mathcal{A}$
% \[
%   h_A (f(t_1,\ldots,t_n)) = f_{\mathcal{A}}\,(h_A\,t_1,...,h_A\,t_n) \enspace .
% \] 
% \noindent We cannot translate this definition directly in Agda, instead
% we have to mutually define | ∣h∣→A | and its extension over vectors
% | ∣h*∣→A| 
% \begin{spec}
%   ∣h∣→A : ∀ {Σ} → (A : Algebra Σ) → {s : sorts Σ} → HU s → ∥ A ⟦ s ⟧ₛ ∥
%   ∣h∣→A A (term f ts) = A ⟦ f ⟧ₒ ⟨$⟩ (∣h*∣→A ts)
% \end{spec}
% \noindent It is straightforward to prove that |∣h∣→A| preserves
% propositional equality and satisfies the homomorphism condition by
% construction. To finish the proof that | ∣T∣ Σ | is initial, we prove,
% by recursion on the structure of terms, that any pair of homomorphisms
% are extensionally equal.

\section{Equational Logic}
\label{sec:eqlog}

In this section we introduce the notion of (conditional) equational
theories and the corresponding notion of satisfiability of theories
by algebras. Moreover we formalize (conditional) equational logic as
presented by \cite{goguen2005specifying} and prove that
the deduction system is sound and complete.

\subsection{Satisfiability and provability}

\paragraph*{Equations} In the mono-sorted setting an equation is a pair
of terms where all the variables are assumed to be universally
quantified and an equational theory is a (finite) set of equations.
In a  multi-sorted setting both sides of an equation should be terms
of the same sort. Moreover we allow quasi-identities which we
write as conditional equations:
\[ t = t'\ \mathsf{if} \  t_1 = t'_1, \ldots, t_n = t'_n \enspace . \]

Let signature \AgdaBound{Σ} and \AgdaBound{X} be a family of variables
for \AgdaBound{Σ}. A conditional equation is modelled as a record
with fields for the conclusion and the conditions, consist of an
heterogeneous vector of sorted identities. We declare a constructor to
use the lighter notation
\AgdaInductiveConstructor{⋀}\AgdaSpace{}\AgdaBound{eq}\AgdaSpace{}
\AgdaInductiveConstructor{if}\AgdaSpace{}\AgdaBound{(ar, eqs)} instead of
\AgdaKeyword{record}\AgdaSpace{}%
\AgdaSymbol{\{}\AgdaField{eq}\AgdaSpace{}%
\AgdaSymbol{=}\AgdaSpace{}%
\AgdaBound{eq}\AgdaSpace{}%
\AgdaSymbol{;}\AgdaSpace{}%
\AgdaField{cond}\AgdaSpace{}%
\AgdaSymbol{=}\AgdaSpace{}%
\AgdaBound{\AgdaBound{(ar, eqs)}}
\AgdaSymbol{\}}

\begin{code}
\>[0]\AgdaKeyword{record}\AgdaSpace{}%
\AgdaRecord{Equation}\AgdaSpace{}%
\AgdaSymbol{(}\AgdaBound{X}\AgdaSpace{}%
\AgdaSymbol{:}\AgdaSpace{}%
\AgdaFunction{Vars}\AgdaSpace{}%
\AgdaBound{Σ}\AgdaSymbol{)}\AgdaSpace{}%
\AgdaSymbol{(}\AgdaBound{s}\AgdaSpace{}%
\AgdaSymbol{:}\AgdaSpace{}%
\AgdaField{sorts}\AgdaSpace{}%
\AgdaBound{Σ}\AgdaSymbol{)}\AgdaSpace{}%
\AgdaSymbol{:}\AgdaSpace{}%
\AgdaPrimitiveType{Set}\AgdaSpace{}%
\AgdaKeyword{where}\<%
\\
\>[0][@{}l@{\AgdaIndent{0}}]%
\>[2]\AgdaKeyword{constructor}\AgdaSpace{}%
\AgdaOperator{\AgdaInductiveConstructor{⋀\AgdaUnderscore{}if\AgdaUnderscore{}}}\<%
\\
%
\>[2]\AgdaKeyword{field}\<%
\\
\>[2][@{}l@{\AgdaIndent{0}}]%
\>[4]\AgdaField{eq}%
\>[8]\AgdaSymbol{:}\AgdaSpace{}%
\AgdaRecord{Equ}\AgdaSpace{}%
\AgdaBound{X}\AgdaSpace{}%
\AgdaBound{s}\<%
\\
%
\>[4]\AgdaField{cond}\AgdaSpace{}%
\AgdaSymbol{:}\AgdaSpace{}%
\AgdaFunction{∃}\AgdaSpace{}%
\AgdaSymbol{(}\AgdaDatatype{HVec}\AgdaSpace{}%
\AgdaSymbol{(}\AgdaRecord{Equ}\AgdaSpace{}%
\AgdaBound{X}\AgdaSymbol{))}\<%
\\
%
\end{code}

\noindent A \emph{theory} over the signature $\Sigma$ is given by a
vector of conditional equations.
\begin{code}
  \>[0]\AgdaFunction{Theory}\AgdaSpace{}%
\AgdaSymbol{:}\AgdaSpace{}%
\AgdaSymbol{(}\AgdaBound{X}\AgdaSpace{}%
\AgdaSymbol{:}\AgdaSpace{}%
\AgdaFunction{Vars}\AgdaSpace{}%
\AgdaBound{Σ}\AgdaSymbol{)}\AgdaSpace{}%
\AgdaSymbol{→}\AgdaSpace{}%
\AgdaSymbol{(}\AgdaBound{ar}\AgdaSpace{}%
\AgdaSymbol{:}\AgdaSpace{}%
\AgdaFunction{Arity}\AgdaSpace{}%
\AgdaBound{Σ}\AgdaSymbol{)}\AgdaSpace{}%
\AgdaSymbol{→}\AgdaSpace{}%
\AgdaPrimitiveType{Set}\<%
\\
\>[0]\AgdaFunction{Theory}\AgdaSpace{}%
\AgdaBound{X}\AgdaSpace{}%
\AgdaBound{ar}\AgdaSpace{}%
\AgdaSymbol{=}\AgdaSpace{}%
\AgdaDatatype{HVec}\AgdaSpace{}%
\AgdaSymbol{(}\AgdaRecord{Equation}\AgdaSpace{}%
\AgdaBound{X}\AgdaSymbol{)}\AgdaSpace{}%
\AgdaBound{ar}\<%
\\
%
\end{code}

We deviate from Goguen's and Lin's in that we assume that all the
equations of a theory share the same set of variables, while they
assume that each equation has its own set of quantified
variables. Clearly, this simplification is harmless; if we have a
theory where each equation has its own set of variables, we can take
the union of those sets as the common set. As stressed by
\cite{goguen-remarks-87}, quantifying equations is essential:
\begin{quote}
  [\ldots] the naive unsorted rules of deduction for equational logic
  (namely, reflexivity, symmetry, transitivity and substitutivity) are
  not sound when extended to the many-sorted case in the obvious way;
  [\ldots] adding variable declarations to these rules
  yields a rule set that is sound.
\end{quote}

\noindent Notice that we follow Goguen and Meseguer in that equations
are given explicitly over a set of variables. This, in turn, leads us
to define satisfiability as proposed by Huet and Oppen.

\paragraph*{Satisfiability} Let $\Sigma$ be a signature and
$\mathcal{A}$ be an algebra for $\Sigma$. We say that a conditional equation
$t = t'\ \mathsf{if}\ t_1 = t'_1,\ldots,t_n=t'_n$ is
\emph{satisfied} by $\mathcal{A}$ if for any
environment $\theta : X \to \mathcal{A}$, $⟦ t ⟧θ = ⟦ t' ⟧θ$, whenever
$⟦ t_i ⟧θ = ⟦ t'_i ⟧θ$ for $1 \leqslant i \leqslant n$. In order to
formalize satisfiability we first define when an environment models an
equation.

\begin{code}
  \>[0]\AgdaOperator{\AgdaFunction{\AgdaUnderscore{},\AgdaUnderscore{}⊨ₑ\AgdaUnderscore{}}}\AgdaSpace{}%
\AgdaSymbol{:}\AgdaSpace{}%
\AgdaSymbol{∀}%
\>[12]\AgdaSymbol{\{}\AgdaBound{X}\AgdaSpace{}%
\AgdaSymbol{:}\AgdaSpace{}%
\AgdaFunction{Vars}\AgdaSpace{}%
\AgdaBound{Σ}\AgdaSymbol{\}}\AgdaSpace{}%
\AgdaSymbol{(}\AgdaBound{A}\AgdaSpace{}%
\AgdaSymbol{:}\AgdaSpace{}%
\AgdaRecord{Algebra}\AgdaSpace{}%
\AgdaBound{Σ}\AgdaSymbol{)}\AgdaSpace{}%
\AgdaSymbol{→}\AgdaSpace{}%
\AgdaSymbol{(}\AgdaBound{θ}\AgdaSpace{}%
\AgdaSymbol{:}\AgdaSpace{}%
\AgdaFunction{Env}\AgdaSpace{}%
\AgdaBound{X}\AgdaSpace{}%
\AgdaBound{A}\AgdaSpace{}%
\AgdaSymbol{)}\AgdaSpace{}%
\AgdaSymbol{→}\AgdaSpace{}%
\AgdaSymbol{\{}\AgdaBound{s}\AgdaSpace{}%
\AgdaSymbol{:}\AgdaSpace{}%
\AgdaField{sorts}\AgdaSpace{}%
\AgdaBound{Σ}\AgdaSymbol{\}}\AgdaSpace{}%
\AgdaSymbol{→}\AgdaSpace{}%
\AgdaRecord{Equ}\AgdaSpace{}%
\AgdaBound{X}\AgdaSpace{}%
\AgdaBound{s}\AgdaSpace{}%
\AgdaSymbol{→}\AgdaSpace{}%
\AgdaPrimitiveType{Set}\<%
\\
\>[0]\AgdaOperator{\AgdaFunction{\AgdaUnderscore{},\AgdaUnderscore{}⊨ₑ\AgdaUnderscore{}}}\AgdaSpace{}%
\AgdaSymbol{\{}\AgdaBound{X}\AgdaSymbol{\}}\AgdaSpace{}%
\AgdaBound{A}\AgdaSpace{}%
\AgdaBound{θ}\AgdaSpace{}%
\AgdaSymbol{\{}\AgdaBound{s}\AgdaSymbol{\}}\AgdaSpace{}%
\AgdaSymbol{(}\AgdaBound{t}\AgdaSpace{}%
\AgdaOperator{\AgdaInductiveConstructor{≈ₑ}}\AgdaSpace{}%
\AgdaBound{t'}\AgdaSymbol{)}\AgdaSpace{}%
\AgdaSymbol{=}\AgdaSpace{}%
\AgdaFunction{eval}\AgdaSpace{}%
\AgdaBound{θ}\AgdaSpace{}%
\AgdaBound{t}\AgdaSpace{}%
\AgdaOperator{\AgdaFunction{≈}}\AgdaSpace{}%
\AgdaFunction{eval}\AgdaSpace{}%
\AgdaBound{θ}\AgdaSpace{}%
\AgdaBound{t'}\<%
\end{code}

\noindent Using the point-wise extension of this relation we can write
directly the notion of satisfiability.

\begin{code}
\>[0]\AgdaOperator{\AgdaFunction{\AgdaUnderscore{}⊨\AgdaUnderscore{}}}\AgdaSpace{}%
\AgdaSymbol{:}\AgdaSpace{}%
\AgdaSymbol{∀}\AgdaSpace{}%
\AgdaSymbol{(}\AgdaBound{A}\AgdaSpace{}%
\AgdaSymbol{:}\AgdaSpace{}%
\AgdaRecord{Algebra}\AgdaSpace{}%
\AgdaBound{Σ}\AgdaSymbol{)}\AgdaSpace{}%
\AgdaSymbol{\{}\AgdaBound{X}\AgdaSpace{}%
\AgdaSymbol{:}\AgdaSpace{}%
\AgdaFunction{Vars}\AgdaSpace{}%
\AgdaBound{Σ}\AgdaSymbol{\}}\AgdaSpace{}%
\AgdaSymbol{\{}\AgdaBound{s}\AgdaSpace{}%
\AgdaSymbol{:}\AgdaSpace{}%
\AgdaField{sorts}\AgdaSpace{}%
\AgdaBound{Σ}\AgdaSymbol{\}}\AgdaSpace{}%
\AgdaSymbol{→}\AgdaSpace{}%
\AgdaRecord{Equation}\AgdaSpace{}%
\AgdaBound{X}\AgdaSpace{}%
\AgdaBound{s}\AgdaSpace{}%
\AgdaSymbol{→}\AgdaSpace{}%
\AgdaPrimitiveType{Set}\<%
\\
\>[0]\AgdaOperator{\AgdaFunction{\AgdaUnderscore{}⊨\AgdaUnderscore{}}}%
\>[423I]\AgdaBound{A}\AgdaSpace{}%
\AgdaSymbol{\{}\AgdaArgument{X}\AgdaSpace{}%
\AgdaSymbol{=}\AgdaSpace{}%
\AgdaBound{X}\AgdaSymbol{\}}\AgdaSpace{}%
\AgdaSymbol{\{}\AgdaBound{s}\AgdaSymbol{\}}\AgdaSpace{}%
\AgdaSymbol{(}\AgdaOperator{\AgdaInductiveConstructor{⋀}}\AgdaSpace{}%
\AgdaBound{equ}\AgdaSpace{}%
\AgdaOperator{\AgdaInductiveConstructor{if}}\AgdaSpace{}%
\AgdaSymbol{(}\AgdaBound{ar}\AgdaSpace{}%
\AgdaOperator{\AgdaInductiveConstructor{,}}\AgdaSpace{}%
\AgdaBound{conds}\AgdaSymbol{))}\AgdaSpace{}%
\AgdaSymbol{=}\<%
\\
\>[.][@{}l@{}]\<[423I]%
\>[4]\AgdaSymbol{(}\AgdaBound{θ}\AgdaSpace{}%
\AgdaSymbol{:}\AgdaSpace{}%
\AgdaFunction{Env}\AgdaSpace{}%
\AgdaBound{X}\AgdaSpace{}%
\AgdaBound{A}\AgdaSpace{}%
\AgdaSymbol{)}\AgdaSpace{}%
\AgdaSymbol{→}\AgdaSpace{}%
\AgdaSymbol{(}\AgdaBound{A}\AgdaSpace{}%
\AgdaOperator{\AgdaFunction{,}}\AgdaSpace{}%
\AgdaBound{θ}\AgdaSpace{}%
\AgdaOperator{\AgdaFunction{⊨ₑ\AgdaUnderscore{}}}\AgdaSymbol{)}\AgdaSpace{}%
\AgdaOperator{\AgdaDatatype{⇨v}}\AgdaSpace{}%
\AgdaBound{conds}\AgdaSpace{}%
\AgdaSymbol{→}\AgdaSpace{}%
\AgdaBound{A}\AgdaSpace{}%
\AgdaOperator{\AgdaFunction{,}}\AgdaSpace{}%
\AgdaBound{θ}\AgdaSpace{}%
\AgdaOperator{\AgdaFunction{⊨ₑ}}\AgdaSpace{}%
\AgdaBound{equ}\<%
\\
%
\end{code}

\noindent We say that $\mathcal{A}$ is a \emph{model} of the theory
$E$ if it satisfies each equation in $E$. As usual an equation is a
logical consequence of a theory, if every model of the theory
satisfies the equation.

\begin{code}
  \>[0]\AgdaOperator{\AgdaFunction{\AgdaUnderscore{}⊨T\AgdaUnderscore{}}}\AgdaSpace{}%
  \AgdaSymbol{:}%
\>[454I]\AgdaSymbol{∀}\AgdaSpace{}%
\AgdaSymbol{\{}\AgdaBound{X}\AgdaSpace{}%
\AgdaBound{ar}\AgdaSymbol{\}}\AgdaSpace{}%
\AgdaSymbol{→}\AgdaSpace{}%
\AgdaSymbol{(}\AgdaBound{A}\AgdaSpace{}%
\AgdaSymbol{:}\AgdaSpace{}%
\AgdaRecord{Algebra}\AgdaSpace{}%
\AgdaBound{Σ}\AgdaSymbol{)}\AgdaSpace{}%
\AgdaSymbol{→}\AgdaSymbol{(}\AgdaBound{E}\AgdaSpace{}%
\AgdaSymbol{:}\AgdaSpace{}%
\AgdaFunction{Theory}\AgdaSpace{}%
\AgdaBound{X}\AgdaSpace{}%
\AgdaBound{ar}\AgdaSymbol{)}\AgdaSpace{}%
\AgdaSymbol{→}\AgdaSpace{}%
\AgdaPrimitiveType{Set}\<%
\\
\>[0]\AgdaBound{A}\AgdaSpace{}%
\AgdaOperator{\AgdaFunction{⊨T}}\AgdaSpace{}%
\AgdaBound{E}\AgdaSpace{}%
\AgdaSymbol{=}\AgdaSpace{}%
\AgdaSymbol{∀}\AgdaSpace{}%
\AgdaSymbol{\{}\AgdaBound{s}\AgdaSpace{}%
\AgdaBound{e}\AgdaSymbol{\}}\AgdaSpace{}%
\AgdaSymbol{→}\AgdaSpace{}%
\AgdaOperator{\AgdaDatatype{\AgdaUnderscore{}∈\AgdaUnderscore{}}}\AgdaSpace{}%
\AgdaSymbol{\{}\AgdaArgument{i}\AgdaSpace{}%
\AgdaSymbol{=}\AgdaSpace{}%
\AgdaBound{s}\AgdaSymbol{\}}\AgdaSpace{}%
\AgdaBound{e}\AgdaSpace{}%
\AgdaBound{E}\AgdaSpace{}%
\AgdaSymbol{→}\AgdaSpace{}%
\AgdaBound{A}\AgdaSpace{}%
\AgdaOperator{\AgdaFunction{⊨}}\AgdaSpace{}%
\AgdaBound{e}\<%
\\
%
\\[\AgdaEmptyExtraSkip]%
%
\\[\AgdaEmptyExtraSkip]%
\>[0]\AgdaFunction{⊨All}\AgdaSpace{}%
\AgdaSymbol{:}\AgdaSpace{}%
\AgdaSymbol{∀}\AgdaSpace{}%
\AgdaSymbol{\{}\AgdaSpace{}%
\AgdaBound{X}\AgdaSymbol{\}}\AgdaSpace{}%
\AgdaSymbol{\{}\AgdaBound{ar}\AgdaSpace{}%
\AgdaSymbol{:}\AgdaSpace{}%
\AgdaFunction{Arity}\AgdaSpace{}%
\AgdaBound{Σ}\AgdaSymbol{\}}\AgdaSpace{}%
\AgdaSymbol{\{}\AgdaBound{s}\AgdaSpace{}%
\AgdaSymbol{:}\AgdaSpace{}%
\AgdaField{sorts}\AgdaSpace{}%
\AgdaBound{Σ}\AgdaSymbol{\}}\AgdaSpace{}%
\AgdaSymbol{→}\AgdaSpace{}%
\AgdaSymbol{(}\AgdaBound{E}\AgdaSpace{}%
\AgdaSymbol{:}\AgdaSpace{}%
\AgdaFunction{Theory}\AgdaSpace{}%
\AgdaBound{X}\AgdaSpace{}%
\AgdaBound{ar}\AgdaSymbol{)}\AgdaSpace{}%
\AgdaSymbol{→}\AgdaSymbol{(}\AgdaBound{e}\AgdaSpace{}%
\AgdaSymbol{:}\AgdaSpace{}%
\AgdaRecord{Equation}\AgdaSpace{}%
\AgdaBound{X}\AgdaSpace{}%
\AgdaBound{s}\AgdaSymbol{)}\AgdaSpace{}%
\AgdaSymbol{→}\AgdaSpace{}%
\AgdaPrimitiveType{Set}\<%
\\
\>[0]\AgdaFunction{⊨All}\AgdaSpace{}%
\AgdaBound{E}\AgdaSpace{}%
\AgdaBound{e}\AgdaSpace{}%
\AgdaSymbol{=}\AgdaSpace{}%
\AgdaSymbol{(}\AgdaBound{A}\AgdaSpace{}%
\AgdaSymbol{:}\AgdaSpace{}%
\AgdaRecord{Algebra}\AgdaSpace{}%
\AgdaBound{Σ}\AgdaSymbol{)}\AgdaSpace{}%
\AgdaSymbol{→}\AgdaSpace{}%
\AgdaBound{A}\AgdaSpace{}%
\AgdaOperator{\AgdaFunction{⊨T}}\AgdaSpace{}%
\AgdaBound{E}\AgdaSpace{}%
\AgdaSymbol{→}\AgdaSpace{}%
\AgdaBound{A}\AgdaSpace{}%
\AgdaOperator{\AgdaFunction{⊨}}\AgdaSpace{}%
\AgdaBound{e}\<%
\\
%
\end{code}

\paragraph*{Provability} As noticed by \cite{huet-rewrite}, the
definition of a sound deduction system for multi-sorted equality logic
is more subtle than expected. We formalize the system presented of
\cite{goguen2005specifying}, shown in Fig.~\ref{fig:deduction}. The
first three rules are reflexivity, symmetry and transitivity; the
fourth rule, called substitution, allows to instantiate an axiom with
a substitution \AgdaBound{σ}, provided one has proofs for every condition
of the axiom;\footnote{In our formalization this rule is slightly less
  general because we assume all the equations are quantified over the
  same set of variables.}  finally, the last rule internalizes Leibniz
rule, for replacing equals by equals in subterms.  Notice that we can
only prove identities and not quasi-identities.
\begin{figure}[t]
  \centering
  \bottomAlignProof
  \AxiomC{}
  \UnaryInfC{$E \vdash \forall X,\, t = t$}
  \DisplayProof\hspace{2ex}
%
  \bottomAlignProof
  \AxiomC{$E \vdash \forall X,\, t_0 = t_1$}
  \UnaryInfC{$E \vdash \forall X,\, t_1 = t_0$}
  \DisplayProof \hspace{2ex}
% 
 \bottomAlignProof
 \AxiomC{$E \vdash \forall X,\, t_0 = t_1$}
  \AxiomC{$E \vdash \forall X,\, t_1 = t_2$}
  \BinaryInfC{$E \vdash \forall X,\, t_0 = t_2$}
  \DisplayProof
\\[6pt]
  \AxiomC{$\forall Y,\,t = t' \ \mathsf{if}\
      t_1=t'_1,\ldots, t_n=t'_n \in E$}
  \AxiomC{$E \vdash \forall X,\,\sigma(t_i) = \sigma(t'_i)$}
  \RightLabel{$\sigma \colon Y \rightarrow T_\Sigma(X)$}
  \BinaryInfC{$E \vdash \forall X,\, \sigma(t) = \sigma(t')$}
  \DisplayProof
\\[6pt]
  \AxiomC{$E \vdash \forall X,\, t_1 = t'_1$}
  \AxiomC{$\cdots$}
  \AxiomC{$E \vdash \forall X,\, t_n = t'_n$}
  \RightLabel{$f : [s_1,...,s_{n}] \Rightarrow_{\Sigma} s$}
  \TrinaryInfC{$E \vdash \forall X,\, f\,(t_1,\ldots,t_n) = f\,(t'_1,\ldots,t'_n)$}
  \DisplayProof\\[6pt]
  \caption{Deduction system}
  \label{fig:deduction}
\end{figure}
We define the relation of provability as an inductive type,
parameterized in the theory \AgdaBound{E}, and indexed by the conclusion of the
proof. For conciseness, we only show the constructor for transitivity:

\begin{code}
  \>[2]\AgdaKeyword{data}\AgdaSpace{}%
\AgdaOperator{\AgdaDatatype{\AgdaUnderscore{}⊢\AgdaUnderscore{}}}%
\>[12]\AgdaSymbol{\{}\AgdaBound{ar}\AgdaSpace{}%
\AgdaSymbol{:}\AgdaSpace{}%
\AgdaFunction{Arity}\AgdaSpace{}%
\AgdaBound{Σ}\AgdaSymbol{\}}\AgdaSpace{}%
\AgdaSymbol{(}\AgdaBound{E}\AgdaSpace{}%
\AgdaSymbol{:}\AgdaSpace{}%
\AgdaFunction{Theory}\AgdaSpace{}%
\AgdaBound{X}\AgdaSpace{}%
\AgdaBound{ar}\AgdaSymbol{)}\AgdaSpace{}%
\AgdaSymbol{:}\AgdaSpace{}%
\AgdaSymbol{∀}\AgdaSpace{}%
\AgdaSymbol{\{}\AgdaBound{s}\AgdaSymbol{\}}\AgdaSpace{}%
\AgdaSymbol{→}\AgdaSpace{}%
\AgdaRecord{Equation}\AgdaSpace{}%
\AgdaBound{X}\AgdaSpace{}%
\AgdaBound{s}\AgdaSpace{}%
\AgdaSymbol{→}\AgdaSpace{}%
\AgdaPrimitiveType{Set}\AgdaSpace{}%
\AgdaKeyword{where}\<%
\\
\>[4]\AgdaInductiveConstructor{ptrans}\AgdaSpace{}%
\AgdaSymbol{:}\AgdaSpace{}%
\AgdaSymbol{∀}\AgdaSpace{}%
\AgdaSymbol{\{}\AgdaBound{s}\AgdaSymbol{\}}%
\>[750I]\AgdaSymbol{\{}\AgdaBound{t₀}\AgdaSpace{}%
\AgdaBound{t₁}\AgdaSpace{}%
\AgdaBound{t₂}\AgdaSpace{}%
\AgdaSymbol{:}\AgdaSpace{}%
\AgdaOperator{\AgdaFunction{TΣ〔}}\AgdaSpace{}%
\AgdaBound{X}\AgdaSpace{}%
\AgdaOperator{\AgdaFunction{〕}}\AgdaSpace{}%
\AgdaOperator{\AgdaFunction{∥}}\AgdaSpace{}%
\AgdaBound{s}\AgdaSpace{}%
\AgdaOperator{\AgdaFunction{∥}}\AgdaSymbol{\}}\AgdaSpace{}%
\AgdaSymbol{→}\<%
\\
\>[.][@{}l@{}]\<[750I]%
\>[19]\AgdaBound{E}\AgdaSpace{}%
\AgdaOperator{\AgdaDatatype{⊢}}\AgdaSpace{}%
\AgdaSymbol{(}\AgdaOperator{\AgdaFunction{⋀}}\AgdaSpace{}%
\AgdaBound{t₀}\AgdaSpace{}%
\AgdaOperator{\AgdaFunction{≈}}\AgdaSpace{}%
\AgdaBound{t₁}\AgdaSymbol{)}\AgdaSpace{}%
\AgdaSymbol{→}\AgdaSpace{}%
\AgdaBound{E}\AgdaSpace{}%
\AgdaOperator{\AgdaDatatype{⊢}}\AgdaSpace{}%
\AgdaSymbol{(}\AgdaOperator{\AgdaFunction{⋀}}\AgdaSpace{}%
\AgdaBound{t₁}\AgdaSpace{}%
\AgdaOperator{\AgdaFunction{≈}}\AgdaSpace{}%
\AgdaBound{t₂}\AgdaSymbol{)}\AgdaSpace{}%
\AgdaSymbol{→}\AgdaSpace{}%
\AgdaBound{E}\AgdaSpace{}%
\AgdaOperator{\AgdaDatatype{⊢}}\AgdaSpace{}%
\AgdaSymbol{(}\AgdaOperator{\AgdaFunction{⋀}}\AgdaSpace{}%
\AgdaBound{t₀}\AgdaSpace{}%
\AgdaOperator{\AgdaFunction{≈}}\AgdaSpace{}%
\AgdaBound{t₂}\AgdaSymbol{)}\<%
\\
%
\end{code}

Let \AgdaBound{E} be a theory over a signature \AgdaBound{Σ}. It is
straightforward to define a setoid over
\AgdaFunction{TΣ}
\AgdaFunction{〔}\AgdaSpace{}\AgdaBound{X}\AgdaSpace{}\AgdaFunction{〕}
by letting \AgdaBound{t₁}\AgdaFunction{≈}\AgdaBound{t₂} if
\AgdaBound{E}\AgdaSpace{}\AgdaDatatype{⊢}\AgdaSpace{}\AgdaBound{t₁}\AgdaFunction{≈}\AgdaBound{t₂}; this
equivalence relation (thanks to the first three rules) is a congruence (because
of the last rule) over the term algebra. We can also use the facility provided
by the standard library to write proofs with several transitive steps more
nicely, as can be seen in the next example.

Soundness and completeness are proved as in the
mono-sorted case. For soundness one proceeds by induction on the
derivations; completeness is a consequence of the fact that the quotient of the
term algebra by provable equality is a model.
\begin{theorem}[Soundness and Completeness]
  $E \vdash t ≈ t'$ iff $E \models_{\Sigma} t ≈ t'$.
\end{theorem}
\noindent Let us remark that completeness does not imply that there is a
decidability algorithm for every theory; \ie this result gives no decision
procedure at all.

Let $E$ and $E'$ be two theories over the signature $\Sigma$. We say
that $E$ is \emph{stronger} than $E'$ if every axiom $e \in E'$ can be
deduced from $E$, written $E \vdash_{T}\, E'$.  Obviously if $E$
is stronger than $E'$, then any equation that can be deduced from $E'$
can also be deduced from $E$ and any model of $E$ is also a model of
$E'$.

\subsection{Equational classes are closed under IHSP}

Given a class $\mathcal{C}$ of algebras over some signature $\Sigma$
one interesting question is to know whether there it is equational,
ie.\ if there exists some theory $E$ such that the models of $E$ are
exactly those algebras in $\mathcal{C}$. Birkhoff showed that a
sufficient and necessary condition is to be closed under certain
class-operations.

$\mathcal{C}$ is closed under Isomorphism if
$\mathcal{B}\in \mathcal{C}$ whenever $\mathcal{A}\in \mathcal{C}$ and
$\mathcal{A}\simeq \mathcal{B}$. It is closed under products if for
any (arbitrary) family of algebras $\mathcal{A}_i\in \mathcal{C}$ its
direct product $\Pi_{i\in I} \mathcal{A}_i$ is also in
$\mathcal{C}$. Analogously, $\mathcal{C}$ is closed under subalgebras
if $\mathcal{B}\in \mathcal{C}$ for any subalgebra
$\mathcal{B}\leqslant \mathcal{A} \in \mathcal{C}$. Finally, it is
closed under homomorphic images if for any
$\mathcal{A} \in \mathcal{C}$ and epimorhism
$h : \mathcal{A} \to \mathcal{B}$ the homomorphic image of $A$ under
$h$ is also in $\mathcal{C}$. A class closed by these operations is
called a \emph{variety}. Birkhoff proved that every variety is the
class of models of some equational theory.

In our formalization we proved that the class of models of any
equational theory is closed under IHSP. The proof for I, S and P are
almost the same. In the module \AgdaModule{aux-sem} we proved a result
useful for all of them. From an environment for $\mathcal{B}$ and a
homomorphism from $\mathcal{B}$ to $\mathcal{A}$ we have an
environment for $\mathcal{A}$. Moreover the valuations in
$\mathcal{A}$ via the valuation in $\mathcal{B}$ and the homo, on the
one hand, and via the new environment should coincide by the universal
mapping property of the free algebra.

\newcommand{\sem}[2]{\llbracket #1 \rrbracket_{\mathcal{#2}}} In order
to prove, for example, that $\mathcal{C}$ is closed under subalgebras
we assume that $\mathcal{A}$ satisfies the equation and wants to prove
that $\mathcal{B}$ also satifies it; that is, we have to prove
$\sem{t}{B} = \sem{t'}{B}$. Since the equality in a subalgebra is
given by the equality on the underlying algebra of the first
projections, we need to show that $\pi₁ \sem{t}{A} = \pi_1\sem{t'}{A}$
Then we note that $\pi_1$ is the subembedding morphism from $\mathcal{B}$ to
$\mathcal{A}$, therefore we can use the previous result and the satisfaction
of $e$ by $\mathcal{A}$ to conclude that $\mathcal{B}$ also satisfies it.
\begin{code}
\>[0][@{}l@{\AgdaIndent{0}}]%
\>[2]\AgdaFunction{SubRespects⊨}\AgdaSpace{}%
\AgdaSymbol{:}\AgdaSpace{}%
\AgdaFunction{SubAlgebra}\AgdaSpace{}%
\AgdaBound{B≤A}\AgdaSpace{}%
\AgdaOperator{\AgdaFunction{⊨}}\AgdaSpace{}%
\AgdaBound{e}\<%
\\
%
\>[2]\AgdaFunction{SubRespects⊨}\AgdaSpace{}%
\AgdaBound{θB}\AgdaSpace{}%
\AgdaBound{B⊨conds}\AgdaSpace{}%
\AgdaSymbol{=}\AgdaSpace{}%
\AgdaOperator{\AgdaFunction{begin}}\<%
\\
\>[2][@{}l@{\AgdaIndent{0}}]%
\>[4]\AgdaField{proj₁}\AgdaSpace{}%
\AgdaOperator{\AgdaFunction{⟦}}\AgdaSpace{}%
\AgdaField{left}\AgdaSpace{}%
\AgdaBound{e}\AgdaSpace{}%
\AgdaOperator{\AgdaFunction{⟧B}}%
\>[24]\AgdaOperator{\AgdaFunction{≈⟨}}\AgdaSpace{}%
\AgdaFunction{sym}\AgdaSpace{}%
\AgdaSymbol{(}\AgdaFunction{⟦t⟧A≈H⟦t⟧B}\AgdaSpace{}%
\AgdaSymbol{(}\AgdaField{left}\AgdaSpace{}%
\AgdaBound{e}\AgdaSymbol{))}\AgdaSpace{}%
\AgdaOperator{\AgdaFunction{⟩}}\<%
\\
%
\>[4]\AgdaOperator{\AgdaFunction{⟦}}\AgdaSpace{}%
\AgdaField{left}\AgdaSpace{}%
\AgdaBound{e}\AgdaSpace{}%
\AgdaOperator{\AgdaFunction{⟧A}}%
\>[24]\AgdaOperator{\AgdaFunction{≈⟨}}\AgdaSpace{}%
\AgdaBound{A⊨e}\AgdaSpace{}%
\AgdaFunction{θA}\AgdaSpace{}%
\AgdaSymbol{(}\AgdaFunction{⊨B*→⊨A*}\AgdaSpace{}%
\AgdaBound{B⊨conds}\AgdaSymbol{)}\AgdaSpace{}%
\AgdaOperator{\AgdaFunction{⟩}}\<%
\\
%
\>[4]\AgdaOperator{\AgdaFunction{⟦}}\AgdaSpace{}%
\AgdaField{right}\AgdaSpace{}%
\AgdaBound{e}\AgdaSpace{}%
\AgdaOperator{\AgdaFunction{⟧A}}%
\>[24]\AgdaOperator{\AgdaFunction{≈⟨}}\AgdaSpace{}%
\AgdaFunction{⟦t⟧A≈H⟦t⟧B}\AgdaSpace{}%
\AgdaSymbol{(}\AgdaField{right}\AgdaSpace{}%
\AgdaBound{e}\AgdaSymbol{)}\AgdaSpace{}%
\AgdaOperator{\AgdaFunction{⟩}}\<%
\\
%
\>[4]\AgdaField{proj₁}\AgdaSpace{}%
\AgdaOperator{\AgdaFunction{⟦}}\AgdaSpace{}%
\AgdaField{right}\AgdaSpace{}%
\AgdaBound{e}\AgdaSpace{}%
\AgdaOperator{\AgdaFunction{⟧B}}\AgdaSpace{}%
\AgdaOperator{\AgdaFunction{∎}}\<%
\\
%
\>[4]\AgdaKeyword{where}\<%
\\
%
\>[4]\AgdaKeyword{open}\AgdaSpace{}%
\AgdaModule{aux-sem}\AgdaSpace{}%
\AgdaBound{A}\AgdaSpace{}%
\AgdaSymbol{(}\AgdaFunction{SubAlgebra}\AgdaSpace{}%
\AgdaBound{B≤A}\AgdaSymbol{)}\AgdaSpace{}%
\AgdaBound{θB}\AgdaSpace{}%
\AgdaSymbol{(}\AgdaFunction{sub-embedding}\AgdaSpace{}%
\AgdaBound{A}\AgdaSpace{}%
\AgdaBound{B≤A}\AgdaSymbol{)}\<%
\\
%
\>[4]\AgdaKeyword{open}\AgdaSpace{}%
\AgdaModule{EqR}\AgdaSpace{}%
\AgdaSymbol{(}\AgdaBound{A}\AgdaSpace{}%
\AgdaOperator{\AgdaField{⟦}}\AgdaSpace{}%
\AgdaBound{s}\AgdaSpace{}%
\AgdaOperator{\AgdaField{⟧ₛ}}\AgdaSymbol{)}\<%
\\
%
\>[4]\AgdaKeyword{open}\AgdaSpace{}%
\AgdaModule{Setoid}\AgdaSpace{}%
\AgdaSymbol{(}\AgdaBound{A}\AgdaSpace{}%
\AgdaOperator{\AgdaField{⟦}}\AgdaSpace{}%
\AgdaBound{s}\AgdaSpace{}%
\AgdaOperator{\AgdaField{⟧ₛ}}\AgdaSymbol{)}\<%
\end{code}

The satisfaction of semi-equations are not preserved by homomorphic
images; we proved that the satisfaction of non-conditional equations
are preserved by homomorphic images. This result can be reduced to
proving that satisfaction is preserved under quotients, because the
homomorphic image is isomorphic to the source algebra quotiened by the
kernel of the homomorphism. In the following code fragment,
\AgdaFunction{equ} is the conditional equation corresponding to the
unconditional equation \AgdaBound{e}.

\begin{code}
\>[3]\AgdaFunction{imgH⊨e}\AgdaSpace{}%
\AgdaSymbol{:}\AgdaSpace{}%
\AgdaFunction{homImg}\AgdaSpace{}%
\AgdaBound{A}\AgdaSpace{}%
\AgdaBound{H}\AgdaSpace{}%
\AgdaOperator{\AgdaFunction{⊨}}\AgdaSpace{}%
\AgdaFunction{equ}\<%
\\
%
\>[3]\AgdaFunction{imgH⊨e}\AgdaSpace{}%
\AgdaSymbol{=}\AgdaSpace{}%
\AgdaFunction{IsoRespects⊨}\AgdaSpace{}%
\AgdaSymbol{(}\AgdaFunction{A/φ⊨e}\AgdaSpace{}%
\AgdaBound{A⊨e}\AgdaSymbol{)}\<%
\\
\>[3][@{}l@{\AgdaIndent{0}}]%
\>[5]\AgdaKeyword{where}%
\>[965I]\AgdaKeyword{open}\AgdaSpace{}%
\AgdaModule{IsoRespectSatisfaction}\AgdaSpace{}%
\AgdaFunction{equ}\AgdaSpace{}%
\AgdaSymbol{(}\AgdaFunction{I.iso-A/kerH}\AgdaSpace{}%
\AgdaBound{A}\AgdaSpace{}%
\AgdaBound{B}\AgdaSpace{}%
\AgdaBound{H}\AgdaSymbol{)}\<%
\\
\>[.][@{}l@{}]\<[965I]%
\>[11]\AgdaKeyword{open}\AgdaSpace{}%
\AgdaModule{QuotientPreserveSatisfaction}\AgdaSpace{}%
\AgdaBound{e}\AgdaSpace{}%
\AgdaBound{A}\AgdaSpace{}%
\AgdaSymbol{(}\AgdaFunction{Kernel}\AgdaSpace{}%
\AgdaBound{H}\AgdaSymbol{)}\<%
\end{code}

% This forced us to introduce a
% notion of non-conditional equations and unconditional equational
% theory.

It is our immediate next goal to prove the converse, namely giving an equational
theory for any class closed by HSP. In the development branch of our
formalization we have some preliminaries definitions such as a more general
notion of free algebras, congruences generated by some relation, and more
results about sub-algebras.

One application that models of equational theories are closed under
IHSP is to prove that some theories are not equational
axiomatizable. We proved that fields are not equational because they
are not closed under products (of course, the problem is that an
equational lacks negations). Our approach to proving this, was to
introduce a shallow notion of fields (\ie we do not have a syntactic
representation of fields, but we model them as setoids with operations
satisfying the axioms) and assume that there is an equational theory
modelling them. Thus, we can easily turn $\mathit{Bool}$ into a field with
the xor and conjunction operations that satisfy the equational theory,
then the self-product must satisfy the equational using that it is closed under
IHSP (particularly Products), however as we have noticed the product of fields
is not a field; for this particular field, clearly not all pair
$(b,b') \neq 0_{\mathit{Bool} \times \mathit{Bool}}$
satisfy $(b,b') \wedge (b,b')^{-1} = 1_{\mathit{Bool} \times \mathit{Bool}}$ where
the operations apply component-wise
\footnote{$1_{\mathit{Bool} \times \mathit{Bool}} = (true, true)$,\,
  $0_{\mathit{Bool} \times \mathit{Bool}} = (false, false)$ and $b^{-1} = b$}.
% COMPLETAR.
% Decir que tomar a≠b como ¬(a=b) se llama tight apartness

In the standard library there are shallow definitions of several
algebraic structures (from magmas up to commutative rings) but it
lacks the product bi-functor. We have defined an interface between
models of the deep embeddings of those structures (in our library)
with those of the standard library. In this way we get by closedness
of products a product of shallow monoids.

\subsection{A theory for Boolean Algebras }\label{sec:eqlog-theory-ol}
In this section we outline
how to formalize an equational theory and illustrate each step by
showing snippets of the formalization of a Boolean Theory presented by
\cite{DBLP:conf/RelMiCS/RochaM08}.\footnote{The full code is available
  in the file \nolinkurl{Examples/EqBool.agda} of the repository.}

\paragraph*{Define the signature}
\label{sec:define-signature} describing the language, and choose
  a family of sets for the variables. It helps if one also introduce
  an abbreviation for terms over the signature extended with
  variables.

  \begin{code}
    \>[0]\AgdaKeyword{data}\AgdaSpace{}%
\AgdaDatatype{Σops₁}\AgdaSpace{}%
\AgdaSymbol{:}\AgdaSpace{}%
\AgdaDatatype{List}\AgdaSpace{}%
\AgdaRecord{⊤}\AgdaSpace{}%
\AgdaOperator{\AgdaFunction{×}}\AgdaSpace{}%
\AgdaRecord{⊤}\AgdaSpace{}%
\AgdaSymbol{→}\AgdaSpace{}%
\AgdaPrimitiveType{Set}\AgdaSpace{}%
\AgdaKeyword{where}\<%
\\
\>[0][@{}l@{\AgdaIndent{0}}]%
\>[2]\AgdaInductiveConstructor{t₁} \AgdaInductiveConstructor{f₁}% %
\>[7]\AgdaSymbol{:}\AgdaSpace{}%
\AgdaDatatype{Σops₁}\AgdaSpace{}%
\AgdaSymbol{(}\AgdaInductiveConstructor{[]}\AgdaSpace{}%
\AgdaOperator{\AgdaInductiveConstructor{,}}\AgdaSpace{}%
\AgdaInductiveConstructor{tt}\AgdaSymbol{)}\<%
\\
%
\>[2]\AgdaInductiveConstructor{neg₁}\AgdaSpace{}%
\AgdaSymbol{:}\AgdaSpace{}%
\AgdaDatatype{Σops₁}\AgdaSpace{}%
\AgdaSymbol{(}\AgdaOperator{\AgdaFunction{[}}\AgdaSpace{}%
\AgdaInductiveConstructor{tt}\AgdaSpace{}%
\AgdaOperator{\AgdaFunction{]}}\AgdaSpace{}%
\AgdaOperator{\AgdaInductiveConstructor{,}}\AgdaSpace{}%
\AgdaInductiveConstructor{tt}\AgdaSymbol{)}\<%
\\
%
\>[2]\AgdaInductiveConstructor{and₁} \AgdaInductiveConstructor{or₁}\AgdaSpace{}%
\AgdaSymbol{:}\AgdaSpace{}%
\AgdaDatatype{Σops₁}\AgdaSpace{}%
\AgdaSymbol{((}\AgdaInductiveConstructor{tt}\AgdaSpace{}%
\AgdaOperator{\AgdaInductiveConstructor{∷}}\AgdaSpace{}%
\AgdaOperator{\AgdaFunction{[}}\AgdaSpace{}%
\AgdaInductiveConstructor{tt}\AgdaSpace{}%
\AgdaOperator{\AgdaFunction{]}}\AgdaSymbol{)}\AgdaSpace{}%
\AgdaOperator{\AgdaInductiveConstructor{,}}\AgdaSpace{}%
\AgdaInductiveConstructor{tt}\AgdaSymbol{)}\<%
\\
%
\\[\AgdaEmptyExtraSkip]%
\>[0]\AgdaFunction{Σbool₁}\AgdaSpace{}%
\AgdaSymbol{:}\AgdaSpace{}%
\AgdaRecord{Signature}\<%
\\
\>[0]\AgdaFunction{Σbool₁}\AgdaSpace{}%
\AgdaSymbol{=}\AgdaSpace{}%
\AgdaKeyword{record}\AgdaSpace{}%
\AgdaSymbol{\{}\AgdaSpace{}%
\AgdaField{sorts}\AgdaSpace{}%
\AgdaSymbol{=}\AgdaSpace{}%
\AgdaRecord{⊤}\AgdaSpace{}%
\AgdaSymbol{;}\AgdaSpace{}%
\AgdaField{ops}\AgdaSpace{}%
\AgdaSymbol{=}\AgdaSpace{}%
\AgdaDatatype{Σops₁}\AgdaSpace{}%
\AgdaSymbol{\}}\<%
\\
%
\\[\AgdaEmptyExtraSkip]%
%
\>[0]\AgdaFunction{ΣVars₁}\AgdaSpace{}%
\AgdaSymbol{:}\AgdaSpace{}%
\AgdaFunction{Vars}\AgdaSpace{}%
\AgdaFunction{Σbool₁}\<%
\\
%
\>[0]\AgdaFunction{ΣVars₁}\AgdaSpace{}%
\AgdaBound{s}\AgdaSpace{}%
\AgdaSymbol{=}\AgdaSpace{}%
\AgdaDatatype{ℕ}\<%
\\
%
\\[\AgdaEmptyExtraSkip]%
%
\>[0]\AgdaFunction{Form₁}\AgdaSpace{}%
\AgdaSymbol{:}\AgdaSpace{}%
\AgdaPrimitiveType{Set}\<%
\\
%
\>[0]\AgdaFunction{Form₁}\AgdaSpace{}%
\AgdaSymbol{=}\AgdaSpace{}%
\AgdaFunction{Terms}\AgdaSpace{}%
\AgdaOperator{\AgdaFunction{∥}}\AgdaSpace{}%
\AgdaInductiveConstructor{tt}\AgdaSpace{}%
\AgdaOperator{\AgdaFunction{∥}}\<%
\\
\end{code}
  
\paragraph*{Introduce smart-constructors}
\label{sec:intr-smart-constr}
  for terms of the extended
  signature with variables to ease writing the axioms and proving
  theorems. Usually one has a smart-constructor for each operation and
  one per variable that is used in the axioms or the theorems.

\begin{code}
\>[4]\AgdaOperator{\AgdaFunction{\AgdaUnderscore{}∧\AgdaUnderscore{}}}\AgdaSpace{}%
\AgdaSymbol{:}\AgdaSpace{}%
\AgdaFunction{Form₁}\AgdaSpace{}%
\AgdaSymbol{→}\AgdaSpace{}%
\AgdaFunction{Form₁}\AgdaSpace{}%
\AgdaSymbol{→}\AgdaSpace{}%
\AgdaFunction{Form₁}\<%
\\
%
\>[4]\AgdaBound{φ}\AgdaSpace{}%
\AgdaOperator{\AgdaFunction{∧}}\AgdaSpace{}%
\AgdaBound{ψ}\AgdaSpace{}%
\AgdaSymbol{=}\AgdaSpace{}%
\AgdaInductiveConstructor{term}\AgdaSpace{}%
\AgdaInductiveConstructor{and₁}\AgdaSpace{}%
\AgdaOperator{\AgdaInductiveConstructor{⟨⟨}}\AgdaSpace{}%
\AgdaBound{φ}\AgdaSpace{}%
\AgdaOperator{\AgdaInductiveConstructor{,}}\AgdaSpace{}%
\AgdaBound{ψ}\AgdaSpace{}%
\AgdaOperator{\AgdaInductiveConstructor{⟩⟩}}\<%
\\
%
\\[\AgdaEmptyExtraSkip]%
%
\>[4]\AgdaFunction{¬}\AgdaSpace{}%
\AgdaSymbol{:}\AgdaSpace{}%
\AgdaFunction{Form₁}\AgdaSpace{}%
\AgdaSymbol{→}\AgdaSpace{}%
\AgdaFunction{Form₁}\<%
\\
%
\>[4]\AgdaFunction{¬}\AgdaSpace{}%
\AgdaBound{φ}\AgdaSpace{}%
\AgdaSymbol{=}\AgdaSpace{}%
\AgdaInductiveConstructor{term}\AgdaSpace{}%
\AgdaInductiveConstructor{neg₁}\AgdaSpace{}%
\AgdaOperator{\AgdaInductiveConstructor{⟪}}\AgdaSpace{}%
\AgdaBound{φ}\AgdaSpace{}%
\AgdaOperator{\AgdaInductiveConstructor{⟫}}\<%
\\
%
\\[\AgdaEmptyExtraSkip]%
%
\>[4]\AgdaFunction{p}\AgdaSymbol{,}\AgdaFunction{q}\AgdaSpace{}%
\AgdaSymbol{:}\AgdaSpace{}%
\AgdaFunction{Form₁}\<%
\\
%
\>[4]\AgdaFunction{p}\AgdaSpace{}%
\AgdaSymbol{=}\AgdaSpace{}%
\AgdaInductiveConstructor{term}\AgdaSpace{}%
\AgdaSymbol{(}\AgdaInductiveConstructor{inj₂}\AgdaSpace{}%
\AgdaNumber{0}\AgdaSymbol{)}\AgdaSpace{}%
\AgdaInductiveConstructor{⟨⟩}\<%
\\
\>[4]\AgdaFunction{q}\AgdaSpace{}%
\AgdaSymbol{=}\AgdaSpace{}%
\AgdaInductiveConstructor{term}\AgdaSpace{}%
\AgdaSymbol{(}\AgdaInductiveConstructor{inj₂}\AgdaSpace{}%
\AgdaNumber{1}\AgdaSymbol{)}\AgdaSpace{}%
\AgdaInductiveConstructor{⟨⟩}\<%
\\
%
\\[\AgdaEmptyExtraSkip]%
%
\>[4]\AgdaFunction{true}\AgdaSymbol{,}\AgdaFunction{false}\AgdaSpace{}%
\AgdaSymbol{:}\AgdaSpace{}%
\AgdaFunction{Form₁}\<%
\\
%
\>[4]\AgdaFunction{true}\AgdaSpace{}%
\AgdaSymbol{=}\AgdaSpace{}%
\AgdaInductiveConstructor{term}\AgdaSpace{}%
\AgdaSymbol{(}\AgdaInductiveConstructor{inj₁}\AgdaSpace{}%
\AgdaInductiveConstructor{t₁}\AgdaSymbol{)}\AgdaSpace{}%
\AgdaInductiveConstructor{⟨⟩}\<%
\\
\>[4]\AgdaFunction{false}\AgdaSpace{}%
\AgdaSymbol{=}\AgdaSpace{}%
\AgdaInductiveConstructor{term}\AgdaSpace{}%
\AgdaSymbol{(}\AgdaInductiveConstructor{inj₁}\AgdaSpace{}%
\AgdaInductiveConstructor{f₁}\AgdaSymbol{)}\AgdaSpace{}%
\AgdaInductiveConstructor{⟨⟩}\<%
\\
%
\end{code}

\paragraph*{Define the equational theory}
\label{sec:define-equat-theory}

 by specifying one equation for
  each axiom and collect them in a theory; here one can appreciate the
  convenience of the smart-constructors. Here we only show two of the
  twelve axioms of the theory \AgdaBound{Tbool}. If one will prove theorems
  of the theory, then it is also convenient to define pattern-synonyms
  for the proofs that each axiom is in the theory.

  \begin{code}
    \>[0]\AgdaFunction{commAnd}\AgdaSpace{}%
    \AgdaSymbol{:}\AgdaSpace{}%
    \AgdaFunction{Eq₁}\<%
    \\
    %
    \>[0]\AgdaFunction{commAnd}\AgdaSpace{}%
    \AgdaSymbol{=}\AgdaSpace{}%
    \AgdaOperator{\AgdaFunction{⋀}}\AgdaSpace{}%
    \AgdaFunction{p}\AgdaSpace{}%
    \AgdaOperator{\AgdaFunction{∧}}\AgdaSpace{}%
    \AgdaFunction{q}\AgdaSpace{}%
    \AgdaOperator{\AgdaFunction{≈}}\AgdaSpace{}%
    \AgdaSymbol{(}\AgdaFunction{q}\AgdaSpace{}%
    \AgdaOperator{\AgdaFunction{∧}}\AgdaSpace{}%
    \AgdaFunction{p}\AgdaSymbol{)}\<%
    \\
    %
    \\[\AgdaEmptyExtraSkip]%
    \>[2]\AgdaFunction{defF}\AgdaSpace{}%
    \AgdaSymbol{:}\AgdaSpace{}%
    \AgdaFunction{Eq₁}\<%
    \\
    %
    \>[2]\AgdaFunction{defF}\AgdaSpace{}%
    \AgdaSymbol{=}\AgdaSpace{}%
    \AgdaOperator{\AgdaFunction{⋀}}\AgdaSpace{}%
    \AgdaFunction{p}\AgdaSpace{}%
    \AgdaOperator{\AgdaFunction{∧}}\AgdaSpace{}%
    \AgdaSymbol{(}\AgdaFunction{¬}\AgdaSpace{}%
    \AgdaFunction{p}\AgdaSymbol{)}\AgdaSpace{}%
    \AgdaOperator{\AgdaFunction{≈}}\AgdaSpace{}%
    \AgdaFunction{false}\<%
    \\%
    \\[\AgdaEmptyExtraSkip]%
    \>[2]\AgdaFunction{Tbool₁}\AgdaSpace{}%
    \AgdaSymbol{:}\AgdaSpace{}%
    \AgdaFunction{Theory}\AgdaSpace{}%
    \AgdaFunction{Vars}\AgdaSpace{}%
    \AgdaSymbol{(}\AgdaFunction{replicate}\AgdaSpace{}%
    \AgdaNumber{12}\AgdaSpace{}%
    \AgdaInductiveConstructor{tt}\AgdaSymbol{)}\<%
    \\
    %
    \>[2]\AgdaFunction{Tbool₁}\AgdaSpace{}%
    \AgdaSymbol{=}\AgdaSpace{}%
    \AgdaSymbol{⟨}\AgdaSpace{}\AgdaFunction{commAnd}\AgdaSpace{}%
    \AgdaSymbol{,}\AgdaSpace{}%
    \AgdaFunction{defF}\AgdaSpace{}%
    \AgdaSymbol{,}\AgdaSpace{}%
    $\ldots$
    \AgdaSymbol{⟩}
    \<%
    \\%
    \\[\AgdaEmptyExtraSkip]%
    \>[2]\AgdaKeyword{pattern}\AgdaSpace{}%
    \AgdaInductiveConstructor{commAndAx}\AgdaSpace{}%
    \AgdaSymbol{=}\AgdaSpace{}%
    \AgdaInductiveConstructor{here}\<%
    \\
    %
    \>[2]\AgdaKeyword{pattern}\AgdaSpace{}%
    \AgdaInductiveConstructor{defFAx}\AgdaSpace{}%
    \AgdaSymbol{=}\AgdaSpace{}%
    \AgdaInductiveConstructor{there}\AgdaSpace{}%
    \AgdaInductiveConstructor{here}\<%
  \end{code}

\paragraph*{Prove theorems}
  using the axioms of the theory just defined.
  If a proof uses transitivity, one can use the equational reasoning
  idiom provided by the standard library of Agda:

  \begin{code}
    \>[2]\AgdaFunction{prop}\AgdaSpace{}%
\AgdaSymbol{:}%
\>[8]\AgdaFunction{Tbool₁}\AgdaSpace{}%
\AgdaOperator{\AgdaDatatype{⊢}}\AgdaSpace{}%
\AgdaSymbol{(}\AgdaOperator{\AgdaFunction{⋀}}\AgdaSpace{}%
\AgdaFunction{¬}\AgdaSpace{}%
\AgdaFunction{p}\AgdaSpace{}%
\AgdaOperator{\AgdaFunction{∧}}\AgdaSpace{}%
\AgdaFunction{p}\AgdaSpace{}%
\AgdaOperator{\AgdaFunction{≈}}\AgdaSpace{}%
\AgdaFunction{false}\AgdaSymbol{)}\<%
\\
%
\>[2]\AgdaFunction{prop}\AgdaSpace{}%
\AgdaSymbol{=}%
\>[503I]\AgdaOperator{\AgdaFunction{begin}}\<%
\\
\>[503I][@{}l@{\AgdaIndent{0}}]%
\>[9]\AgdaFunction{¬}\AgdaSpace{}%
\AgdaFunction{p}\AgdaSpace{}%
\AgdaOperator{\AgdaFunction{∧}}\AgdaSpace{}%
\AgdaFunction{p}\<%
\\
%
\>[9]\AgdaOperator{\AgdaFunction{≈⟨}}\AgdaSpace{}%
\AgdaInductiveConstructor{psubst}\AgdaSpace{}%
\AgdaInductiveConstructor{axComm∧}\AgdaSpace{}%
\AgdaFunction{σ}\AgdaSpace{}%
\AgdaInductiveConstructor{⇨v⟨⟩}\AgdaSpace{}%
\AgdaOperator{\AgdaFunction{⟩}}\<%
\\
%
\>[9]\AgdaFunction{p}\AgdaSpace{}%
\AgdaOperator{\AgdaFunction{∧}}\AgdaSpace{}%
\AgdaFunction{¬}\AgdaSpace{}%
\AgdaFunction{p}\<%
\\
%
\>[9]\AgdaOperator{\AgdaFunction{≈⟨}}\AgdaSpace{}%
\AgdaInductiveConstructor{psubst}\AgdaSpace{}%
\AgdaInductiveConstructor{axDefF}\AgdaSpace{}%
\AgdaFunction{idSubst}\AgdaSpace{}%
\AgdaInductiveConstructor{⇨v⟨⟩}\AgdaSpace{}%
\AgdaOperator{\AgdaFunction{⟩}}\<%
\\
%
\>[9]\AgdaFunction{false}\<%
\\
\>[.][@{}l@{}]\<[503I]%
\>[7]\AgdaOperator{\AgdaFunction{∎}}\<%
  \end{code}

\noindent In the justification steps of this proof we use the
substitution rule. The relevant actions of the substitution \AgdaFunction{σ} are
\AgdaFunction{σ}\AgdaSpace{}\AgdaFunction{p}\AgdaSpace{}\AgdaSymbol{=}
\AgdaFunction{¬}\AgdaSpace{}\AgdaFunction{p} and
\AgdaFunction{σ}\AgdaSpace{}\AgdaFunction{q}\AgdaSpace{}\AgdaSymbol{=}
\AgdaFunction{p}.

%%% Local Variables: ***
%%% mode:latex ***
%%% TeX-master: "univ-alg.tex"  ***
%%% End: ***

\section{Morphisms between signatures}
\label{sec:trans}
In this section we explain our formalization of morphisms between
signatures; this notion is interesting because it provides a
conceptual understanding of syntactic translations. After pointing to
some related works, we motivate the usefulness of this notion by
showing a relatively simple example: how to interpret the Boolean
theory of the previous section in the propositional calculus of
Dijkstra and Scholten.\footnote{\cite{rocha-2007}
  study more thoroughly Boolean theories and their morphisms.}

The concept of morphism between signatures is related with the
interpretability of similarity types in universal algebra
(cf.~\cite{garcia-84}), and has an extensive literature:
~\cite{fujiwara-1959} introduced this notion as
\textit{mappings between algebraic systems},
\cite{janssen-98}, following the ADJ group, called it a
\textit{polynomial derivor} and \cite{mossakowski-15}
referred to it as a \textit{derived signature morphism}, a generalization
of the more restricted \textit{signature morphisms} in the theory of
institutions \citep{goguen-92}.

Let us analyze how to translate the Boolean theory of the previous
section to the propositional calculus of \cite{dijkstra-scholten},
whose only non-constant operations are equivalence and disjunction.
\begin{code}
\>[0]\AgdaKeyword{data}\AgdaSpace{}%
\AgdaDatatype{Σops₂}\AgdaSpace{}%
\AgdaSymbol{:}\AgdaSpace{}%
\AgdaDatatype{List}\AgdaSpace{}%
\AgdaRecord{⊤}\AgdaSpace{}%
\AgdaOperator{\AgdaFunction{×}}\AgdaSpace{}%
\AgdaRecord{⊤}\AgdaSpace{}%
\AgdaSymbol{→}\AgdaSpace{}%
\AgdaPrimitiveType{Set}\AgdaSpace{}%
\AgdaKeyword{where}\<%
\\
\>[0][@{}l@{\AgdaIndent{0}}]%
\>[2]\AgdaInductiveConstructor{t₂}\AgdaSpace{}%
\AgdaInductiveConstructor{f₂}%
\>[14]\AgdaSymbol{:}\AgdaSpace{}%
\AgdaDatatype{Σops₂}\AgdaSpace{}%
\AgdaSymbol{(}\AgdaInductiveConstructor{[]}\AgdaSpace{}%
\AgdaOperator{\AgdaInductiveConstructor{↦}}\AgdaSpace{}%
\AgdaInductiveConstructor{tt}\AgdaSymbol{)}\<%
\\
%
\>[2]\AgdaInductiveConstructor{or₂}\AgdaSpace{}%
\AgdaInductiveConstructor{equiv₂}%
\>[14]\AgdaSymbol{:}\AgdaSpace{}%
\AgdaDatatype{Σops₂}\AgdaSpace{}%
\AgdaSymbol{((}\AgdaInductiveConstructor{tt}\AgdaSpace{}%
\AgdaOperator{\AgdaInductiveConstructor{∷}}\AgdaSpace{}%
\AgdaOperator{\AgdaFunction{[}}\AgdaSpace{}%
\AgdaInductiveConstructor{tt}\AgdaSpace{}%
\AgdaOperator{\AgdaFunction{]}}\AgdaSymbol{)}\AgdaSpace{}%
\AgdaOperator{\AgdaInductiveConstructor{↦}}\AgdaSpace{}%
\AgdaInductiveConstructor{tt}\AgdaSymbol{)}\<%
\\
\\[\AgdaEmptyExtraSkip]%
\>[0]\AgdaFunction{Σbool₂}\AgdaSpace{}%
\AgdaSymbol{:}\AgdaSpace{}%
\AgdaRecord{Signature}\<%
\\
\>[0]\AgdaFunction{Σbool₂}\AgdaSpace{}%
\AgdaSymbol{=}\AgdaSpace{}%
\AgdaKeyword{record}\AgdaSpace{}%
\AgdaSymbol{\{}\AgdaSpace{}%
\AgdaField{sorts}\AgdaSpace{}%
\AgdaSymbol{=}\AgdaSpace{}%
\AgdaRecord{⊤}\AgdaSpace{}%
\AgdaSymbol{;}\AgdaSpace{}%
\AgdaField{ops}\AgdaSpace{}%
\AgdaSymbol{=}\AgdaSpace{}%
\AgdaDatatype{Σops₂}\AgdaSpace{}%
\AgdaSymbol{\}}\<%
\end{code}

It is clear that one can translate recursively any term over
\AgdaFunction{Σbool₁} to a term in \AgdaFunction{Σbool₂}
preserving its semantics. % Tiene sentido decir cómo?
An alternative and more general way is to specify how to translate
each operation in \AgdaFunction{Σbool₁} using operations in
\AgdaFunction{Σbool₂}. In this way, any
\AgdaFunction{Σbool₂}-algebra can be seen as a
\AgdaFunction{Σbool₁}-algebra: a
\AgdaFunction{Σbool₁}-operation \AgdaBound{f} is interpreted as
the semantics of the translation of \AgdaBound{f}. In particular,
the translation of formulas is recovered as the initial
homomorphism between
\AgdaFunction{∣T∣}\AgdaSpace{}\AgdaFunction{Σbool₁} and the
transformation of
\AgdaFunction{∣T∣}\AgdaSpace{}\AgdaFunction{Σbool₂}. In this
section we formalize the concepts of \emph{derived signature
  morphism} and \emph{reduct algebra} as introduced, for example,
by \cite{sannella2012foundations}.

\subsection{Derived signature morphism}

Although the disjunction from \AgdaFunction{Σbool₁} can be directly mapped to
its namesake in \AgdaFunction{Σbool₂}, there is no unary operation in
\AgdaFunction{Σbool₂} to translate the negation. In fact, we should be able
to translate an operation as a combination of operations in \AgdaFunction{Σbool₂}
and also refer to the arguments of the original operation.
\newcommand{\sdash}[1]{\vdash}

We introduce the notion of \emph{formal terms} which are formal
composition of projections and operations. We introduce a type
system, shown in Fig.~\ref{fig:formalterms}, ensuring the
well-formedness of these terms: the contexts are arities, \ie lists of
sorts, and identifiers are pointers (like de Bruijn indices).
\begin{figure}[t]
  \centering
    \bottomAlignProof
    \AxiomC{}
    \RightLabel{\textsc{(prj)}}
    \UnaryInfC{$[s_{1},\ldots,s_{n}] \sdash{\Sigma} \sharp i : s_i$}
  \DisplayProof
% 
  \bottomAlignProof
  \insertBetweenHyps{\hskip -4pt}
  \AxiomC{$f : [s_1,...,s_{n}] \Rightarrow_{\Sigma} s$}
  \AxiomC{$\mathit{ar} \sdash{\Sigma} t_1 : s_1$}
  \AxiomC{$\cdots$}
  \AxiomC{$\mathit{ar} \sdash{\Sigma} t_n : s_n$}
  \RightLabel{\textsc{(op)}}
  \QuaternaryInfC{$\mathit{ar} \sdash{\Sigma} f\,(t_1,...,t_{n}) : s$}
  \DisplayProof
  \\[12pt]
\caption{Type system for formal terms}
\label{fig:formalterms}
\end{figure}
It can be formalized as an inductive family
parameterized by arities and indexed by sorts. 
\begin{code}
\>[0][@{}l@{\AgdaIndent{0}}]%
\>[1]\AgdaKeyword{data}\AgdaSpace{}%
\AgdaOperator{\AgdaDatatype{\AgdaUnderscore{}⊢\AgdaUnderscore{}}}%
\>[11]\AgdaSymbol{(}\AgdaBound{ar'}\AgdaSpace{}%
\AgdaSymbol{:}\AgdaSpace{}%
\AgdaFunction{Arity}\AgdaSpace{}%
\AgdaBound{Σ}\AgdaSymbol{)}\AgdaSpace{}%
\AgdaSymbol{:}\AgdaSpace{}%
\AgdaSymbol{(}\AgdaField{sorts}\AgdaSpace{}%
\AgdaBound{Σ}\AgdaSymbol{)}\AgdaSpace{}%
\AgdaSymbol{→}\AgdaSpace{}%
\AgdaPrimitiveType{Set}\AgdaSpace{}%
\AgdaKeyword{where}\<%
\\
\>[1][@{}l@{\AgdaIndent{0}}]%
\>[3]\AgdaInductiveConstructor{\#}%
\>[7]\AgdaSymbol{:}\AgdaSpace{}%
\AgdaSymbol{(}\AgdaBound{n}\AgdaSpace{}%
\AgdaSymbol{:}\AgdaSpace{}%
\AgdaDatatype{Fin}\AgdaSpace{}%
\AgdaSymbol{(}\AgdaFunction{length}\AgdaSpace{}%
\AgdaBound{ar'}\AgdaSymbol{))}\AgdaSpace{}%
\AgdaSymbol{→}\AgdaSpace{}%
\AgdaBound{ar'}\AgdaSpace{}%
\AgdaOperator{\AgdaDatatype{⊢}}\AgdaSpace{}%
\AgdaSymbol{(}\AgdaBound{ar'}\AgdaSpace{}%
\AgdaOperator{\AgdaFunction{‼}}\AgdaSpace{}%
\AgdaBound{n}\AgdaSymbol{)}\<%
\\
%
\>[3]\AgdaOperator{\AgdaInductiveConstructor{\AgdaUnderscore{}∣\$∣\AgdaUnderscore{}}}\AgdaSpace{}%
\AgdaSymbol{:}\AgdaSpace{}%
\AgdaSymbol{∀}%
\>[113I]\AgdaSymbol{\{}\AgdaBound{ar}\AgdaSpace{}%
\AgdaBound{s}\AgdaSymbol{\}}\AgdaSpace{}%
\AgdaSymbol{→}\AgdaSpace{}%
\AgdaField{ops}\AgdaSpace{}%
\AgdaBound{Σ}\AgdaSpace{}%
\AgdaSymbol{(}\AgdaBound{ar}\AgdaSpace{}%
\AgdaOperator{\AgdaInductiveConstructor{,}}\AgdaSpace{}%
\AgdaBound{s}\AgdaSymbol{)}\AgdaSpace{}%
\AgdaSymbol{→}\AgdaSpace{}\AgdaDatatype{HVec}\AgdaSpace{}%
\AgdaSymbol{(}\AgdaBound{ar'}\AgdaSpace{}%
\AgdaOperator{\AgdaDatatype{⊢\AgdaUnderscore{}}}\AgdaSymbol{)}\AgdaSpace{}%
\AgdaBound{ar}\AgdaSpace{}%
\AgdaSymbol{→}\AgdaSpace{}%
\AgdaBound{ar'}\AgdaSpace{}%
\AgdaOperator{\AgdaDatatype{⊢}}\AgdaSpace{}%
\AgdaBound{s}\<%
\end{code}

A formal term specifies how to interpret an operation from the source
signature in the target signature. The arity \AgdaBound{ar'} specifies the sort
of each argument of the original operation. For example, since the
operation \AgdaInductiveConstructor{neg₁} is unary, we can use one identifier
when defining its translation. Notice that \AgdaFunction{Σbool₁} and
\AgdaFunction{Σbool₂} share the sorts; in general, one also considers a
mapping between sorts.

A \emph{derived signature morphism} consists of a mapping between sorts
and a mapping from operations to formal terms:
\begin{code}
\>[0]\AgdaKeyword{record}\AgdaSpace{}%
\AgdaOperator{\AgdaRecord{\AgdaUnderscore{}↝\AgdaUnderscore{}}}\AgdaSpace{}%
\AgdaSymbol{(}\AgdaBound{Σₛ}\AgdaSpace{}%
\AgdaBound{Σₜ}\AgdaSpace{}%
\AgdaSymbol{:}\AgdaSpace{}%
\AgdaRecord{Signature}\AgdaSymbol{)}\AgdaSpace{}%
\AgdaSymbol{:}\AgdaSpace{}%
\AgdaPrimitiveType{Set}\AgdaSpace{}%
\AgdaKeyword{where}\<%
\\
\>[0][@{}l@{\AgdaIndent{0}}]%
\>[1]\AgdaKeyword{open}\AgdaSpace{}%
\AgdaModule{FormalTerm}\AgdaSpace{}%
\AgdaBound{Σₜ}\<%
\\
%
\>[1]\AgdaKeyword{field}\<%
\\
\>[1][@{}l@{\AgdaIndent{0}}]%
\>[2]\AgdaField{↝ₛ}\AgdaSpace{}%
\AgdaSymbol{:}\AgdaSpace{}%
\AgdaField{sorts}\AgdaSpace{}%
\AgdaBound{Σₛ}\AgdaSpace{}%
\AgdaSymbol{→}\AgdaSpace{}%
\AgdaField{sorts}\AgdaSpace{}%
\AgdaBound{Σₜ}\<%
\\
%
\>[2]\AgdaField{↝ₒ}\AgdaSpace{}%
\AgdaSymbol{:}\AgdaSpace{}%
\AgdaSymbol{∀}\AgdaSpace{}%
\AgdaSymbol{\{}\AgdaBound{ar}\AgdaSpace{}%
\AgdaBound{s}\AgdaSymbol{\}}%
\>[17]\AgdaSymbol{→}\AgdaSpace{}%
\AgdaField{ops}\AgdaSpace{}%
\AgdaBound{Σₛ}\AgdaSpace{}%
\AgdaSymbol{(}\AgdaBound{ar}\AgdaSpace{}%
\AgdaOperator{\AgdaInductiveConstructor{,}}\AgdaSpace{}%
\AgdaBound{s}\AgdaSymbol{)}\AgdaSpace{}%
\AgdaSymbol{→}\AgdaSpace{}%
\AgdaFunction{lmap}\AgdaSpace{}%
\AgdaField{↝ₛ}\AgdaSpace{}%
\AgdaBound{ar}\AgdaSpace{}%
\AgdaOperator{\AgdaDatatype{⊢}}\AgdaSpace{}%
\AgdaField{↝ₛ}\AgdaSpace{}%
\AgdaBound{s}\<%
\end{code}
\noindent We show the action of the morphism on the operations
\AgdaInductiveConstructor{or₁} and \AgdaInductiveConstructor{neg₁}
\begin{code}
\>[2]\AgdaFunction{ops↝}\AgdaSpace{}%
\AgdaSymbol{:}\AgdaSpace{}%
\AgdaSymbol{∀}\AgdaSpace{}%
\AgdaSymbol{\{}\AgdaBound{ar}\AgdaSpace{}%
\AgdaBound{s}\AgdaSymbol{\}}\AgdaSpace{}%
\AgdaSymbol{→}\AgdaSpace{}%
\AgdaSymbol{(}\AgdaBound{f}\AgdaSpace{}%
\AgdaSymbol{:}\AgdaSpace{}%
\AgdaDatatype{Σops₁}\AgdaSpace{}%
\AgdaSymbol{(}\AgdaBound{ar}\AgdaSpace{}%
\AgdaOperator{\AgdaInductiveConstructor{,}}\AgdaSpace{}%
\AgdaBound{s}\AgdaSymbol{))}\AgdaSpace{}%
\AgdaSymbol{→}\AgdaSpace{}%
\AgdaFunction{map}\AgdaSpace{}%
\AgdaFunction{id}\AgdaSpace{}%
\AgdaBound{ar}\AgdaSpace{}%
\AgdaOperator{\AgdaDatatype{⊢}}\AgdaSpace{}%
\AgdaBound{s}\<%
\\
%
\>[2]\AgdaFunction{ops↝}\AgdaSpace{}%
\AgdaInductiveConstructor{or₁}\AgdaSpace{}%
\AgdaSymbol{=}\AgdaSpace{}%
\AgdaInductiveConstructor{or₂}\AgdaSpace{}%
\AgdaOperator{\AgdaInductiveConstructor{∣\$∣}}\AgdaSpace{}%
\AgdaOperator{\AgdaInductiveConstructor{⟨⟨}}\AgdaSpace{}%
\AgdaFunction{p}\AgdaSpace{}%
\AgdaOperator{\AgdaInductiveConstructor{,}}\AgdaSpace{}%
\AgdaFunction{q}\AgdaSpace{}%
\AgdaOperator{\AgdaInductiveConstructor{⟩⟩}}\<%
\\
%
\>[2]\AgdaFunction{ops↝}\AgdaSpace{}%
\AgdaInductiveConstructor{neg₁}\AgdaSpace{}%
\AgdaSymbol{=}\AgdaSpace{}%
\AgdaInductiveConstructor{equiv₂}\AgdaSpace{}%
\AgdaOperator{\AgdaInductiveConstructor{∣\$∣}}\AgdaSpace{}%
\AgdaOperator{\AgdaInductiveConstructor{⟨⟨}}\AgdaSpace{}%
\AgdaFunction{p}\AgdaSpace{}%
\AgdaOperator{\AgdaInductiveConstructor{,}}\AgdaSpace{}%
\AgdaInductiveConstructor{f₂}\AgdaSpace{}%
\AgdaOperator{\AgdaInductiveConstructor{∣\$∣}}\AgdaSpace{}%
\AgdaInductiveConstructor{⟨⟩}\AgdaSpace{}%
\AgdaOperator{\AgdaInductiveConstructor{⟩⟩}}\<%
\end{code}
\noindent where \AgdaFunction{p}\AgdaSpace{}\AgdaSymbol{=}\AgdaInductiveConstructor{\#}\AgdaSpace{}\AgdaInductiveConstructor{zero} and \AgdaFunction{q}\AgdaSpace{}\AgdaSymbol{=}\AgdaSpace{}\AgdaInductiveConstructor{\#}\AgdaSpace{}\AgdaSymbol{(}\AgdaInductiveConstructor{suc}\AgdaSpace{}\AgdaInductiveConstructor{zero}\AgdaSymbol{)}.

\subsection{Transformation of Algebras}
\newcommand{\intSign}[2]{#1 ↝ #2}
\newcommand{\intTheo}[1]{\widetilde{\theory{#1}}}
\newcommand{\algTrans}[1]{\langle \mathcal{#1} \rangle}
\newcommand{\mapSort}[2]{#1\,#2}
\newcommand{\mapOp}[2]{#1\,#2}

A signature morphism
$m\colon\intSign{\Sigma_s}{\Sigma_t}$ induces a functor from
$\Sigma_t$-algebras to
$\Sigma_s$-algebras.  Given a $\Sigma_t$-algebra
$\mathcal{A}$, we denote with
$\algTrans{A}$ the corresponding
$\Sigma_s$-algebra, which is known as the \emph{reduct algebra with
  respect to the morphism} $m$. Let us sketch the construction of
the functor on algebras: the interpretation of a $\Sigma_s$-sort $s$ is given by
  $\algTrans{A}_s = \mathcal{A}_{(\mapSort{m}{s})}$ and
for interpreting an operation $f$ in the reduct algebra
$\algTrans A$ we use the interpretation of the formal term $m f$, which
is recursively defined by
\begin{code}
\>[4]\AgdaOperator{\AgdaFunction{⟦\AgdaUnderscore{}⟧⊢}}\AgdaSpace{}%
\AgdaSymbol{:}\AgdaSpace{}%
\AgdaSymbol{∀}\AgdaSpace{}%
\AgdaSymbol{\{}\AgdaBound{ar}\AgdaSpace{}%
\AgdaBound{s}\AgdaSymbol{\}}\AgdaSpace{}%
\AgdaSymbol{→}\AgdaSpace{}%
\AgdaBound{ar}\AgdaSpace{}%
\AgdaOperator{\AgdaDatatype{⊢}}\AgdaSpace{}%
\AgdaBound{s}\AgdaSpace{}%
\AgdaSymbol{→}\AgdaSpace{}%
\AgdaBound{A}\AgdaSpace{}%
\AgdaOperator{\AgdaFunction{∥}}\AgdaSpace{}%
\AgdaBound{ar}\AgdaSpace{}%
\AgdaOperator{\AgdaFunction{∥*}}\AgdaSpace{}%
\AgdaSymbol{→}\AgdaSpace{}%
\AgdaBound{A}\AgdaSpace{}%
\AgdaOperator{\AgdaFunction{∥}}\AgdaSpace{}%
\AgdaBound{s}\AgdaSpace{}%
\AgdaOperator{\AgdaFunction{∥}}\<%
\\
%
\>[4]\AgdaOperator{\AgdaFunction{⟦}}\AgdaSpace{}%
\AgdaInductiveConstructor{\#}\AgdaSpace{}%
\AgdaBound{n}\AgdaSpace{}%
\AgdaOperator{\AgdaFunction{⟧⊢}}%
\>[16]\AgdaBound{as}\AgdaSpace{}%
\AgdaSymbol{=}%
\>[22]\AgdaBound{as}\AgdaSpace{}%
\AgdaOperator{\AgdaFunction{‼v}}\AgdaSpace{}%
\AgdaBound{n}\<%
\\
%
\>[4]\AgdaOperator{\AgdaFunction{⟦}}\AgdaSpace{}%
\AgdaBound{f}\AgdaSpace{}%
\AgdaOperator{\AgdaInductiveConstructor{∣\$∣}}\AgdaSpace{}%
\AgdaBound{ts}\AgdaSpace{}%
\AgdaOperator{\AgdaFunction{⟧⊢}}%
\>[19]\AgdaBound{as}\AgdaSpace{}%
\AgdaSymbol{=}\AgdaSpace{}%
\AgdaBound{A}\AgdaSpace{}%
\AgdaOperator{\AgdaField{⟦}}\AgdaSpace{}%
\AgdaBound{f}\AgdaSpace{}%
\AgdaOperator{\AgdaField{⟧ₒ}}\AgdaSpace{}%
\AgdaOperator{\AgdaField{⟨\$⟩}}\AgdaSpace{}%
\AgdaOperator{\AgdaFunction{⟦}}\AgdaSpace{}%
\AgdaBound{ts}\AgdaSpace{}%
\AgdaOperator{\AgdaFunction{⟧⊢*}}\AgdaSpace{}%
\AgdaBound{as}\<%
\end{code}
\noindent Identifiers denote projections and the application of the
operation \AgdaBound{f} to formal terms \AgdaBound{ts} is interpreted as the interpretation of \AgdaBound{f}
applied to the denotation of each term in \AgdaBound{ts}, the function \AgdaOperator{\AgdaFunction{⟦}}%
\AgdaUnderscore{}%
\AgdaOperator{\AgdaFunction{⟧⊢*}} extends
\AgdaOperator{\AgdaFunction{⟦}}%
\AgdaUnderscore{}%
\AgdaOperator{\AgdaFunction{⟧⊢}} to vectors.

We can formalize the reduct algebra in a direct way, however the
interpretation of operations is a little more complicated, since we
need to convince Agda that any vector
\AgdaBound{vs}\AgdaSpace{}\AgdaSymbol{:}%
\AgdaSpace{}\AgdaDatatype{HVec}\AgdaSpace{}\AgdaSymbol{(}\AgdaBound{A}\AgdaSpace{}%
\AgdaOperator{\AgdaFunction{⟦}}\AgdaUnderscore{}\AgdaOperator{\AgdaFunction{⟧ₛ}}\AgdaSpace{}%
\AgdaOperator{\AgdaFunction{∘↝ₛ}}\AgdaSymbol{)}\AgdaBound{is} has also
the type \AgdaDatatype{HVec}\AgdaSpace{}\AgdaBound{A}\AgdaSymbol{(}\AgdaFunction{map}\AgdaSpace{}%
\AgdaOperator{\AgdaFunction{↝ₛ}}\AgdaSpace{}\AgdaBound{is}\AgdaSymbol{)}, which is accomplished by
\AgdaFunction{reindex}-ing the vector (we omit the proof of \AgdaFunction{cong}):
\begin{code}
\>[0]\AgdaKeyword{module}\AgdaSpace{}%
\AgdaModule{ReductAlgebra}\AgdaSpace{}%
\AgdaSymbol{\{}\AgdaBound{Σₛ}\AgdaSpace{}%
\AgdaBound{Σₜ}\AgdaSymbol{\}}\AgdaSpace{}%
\AgdaSymbol{(}\AgdaBound{t}\AgdaSpace{}%
\AgdaSymbol{:}\AgdaSpace{}%
\AgdaBound{Σₛ}\AgdaSpace{}%
\AgdaOperator{\AgdaRecord{↝}}\AgdaSpace{}%
\AgdaBound{Σₜ}\AgdaSymbol{)}\AgdaSpace{}%
\AgdaKeyword{where}\<%
%
\\
\\[\AgdaEmptyExtraSkip]%
%
\>[2]\AgdaOperator{\AgdaFunction{\AgdaUnderscore{}⟨\AgdaUnderscore{}⟩ₛ}}\AgdaSpace{}%
\AgdaSymbol{:}\AgdaSpace{}%
\AgdaSymbol{∀}%
\>[13]\AgdaSymbol{(}\AgdaBound{A}\AgdaSpace{}%
\AgdaSymbol{:}\AgdaSpace{}%
\AgdaRecord{Algebra}\AgdaSpace{}%
\AgdaBound{Σₜ}\AgdaSymbol{)}\AgdaSpace{}%
\AgdaSymbol{→}\AgdaSpace{}%
\AgdaSymbol{(}\AgdaBound{s}\AgdaSpace{}%
\AgdaSymbol{:}\AgdaSpace{}%
\AgdaField{sorts}\AgdaSpace{}%
\AgdaBound{Σₛ}\AgdaSymbol{)}\AgdaSpace{}%
\AgdaSymbol{→}\AgdaSpace{}%
\AgdaRecord{Setoid}\AgdaSpace{}%
\AgdaSymbol{\AgdaUnderscore{}}\AgdaSpace{}%
\AgdaSymbol{\AgdaUnderscore{}}\<%
\\
%
\>[2]\AgdaBound{A}\AgdaSpace{}%
\AgdaOperator{\AgdaFunction{⟨}}\AgdaSpace{}%
\AgdaBound{s}\AgdaSpace{}%
\AgdaOperator{\AgdaFunction{⟩ₛ}}\AgdaSpace{}%
\AgdaSymbol{=}\AgdaSpace{}%
\AgdaBound{A}\AgdaSpace{}%
\AgdaOperator{\AgdaField{⟦}}\AgdaSpace{}%
\AgdaField{↝ₛ}\AgdaSpace{}%
\AgdaBound{t}\AgdaSpace{}%
\AgdaBound{s}\AgdaSpace{}%
\AgdaOperator{\AgdaField{⟧ₛ}}\<%
%\\
%
\\[\AgdaEmptyExtraSkip]%
%
\>[2]\AgdaOperator{\AgdaFunction{\AgdaUnderscore{}⟨\AgdaUnderscore{}⟩ₒ}}\AgdaSpace{}%
\AgdaSymbol{:}%
\>[11]\AgdaSymbol{(}\AgdaBound{A}\AgdaSpace{}%
\AgdaSymbol{:}\AgdaSpace{}%
\AgdaRecord{Algebra}\AgdaSpace{}%
\AgdaBound{Σₜ}\AgdaSymbol{)}\AgdaSpace{}%
\AgdaSymbol{→}\AgdaSpace{}%
\AgdaField{ops}\AgdaSpace{}%
\AgdaBound{Σₛ}\AgdaSpace{}%
\AgdaSymbol{(}\AgdaBound{ar}\AgdaSpace{}%
\AgdaOperator{\AgdaInductiveConstructor{,}}\AgdaSpace{}%
\AgdaBound{s}\AgdaSymbol{)}\AgdaSpace{}%
\AgdaSymbol{→}\AgdaSpace{}%
\AgdaSymbol{(}\AgdaBound{A}\AgdaSpace{}%
\AgdaOperator{\AgdaFunction{⟨\AgdaUnderscore{}⟩ₛ}}\AgdaSymbol{)}\AgdaSpace{}%
\AgdaOperator{\AgdaFunction{✳}}\AgdaSpace{}%
\AgdaBound{ar}\AgdaSpace{}%
\AgdaOperator{\AgdaFunction{⟶}}%
\>[93]\AgdaBound{A}\AgdaSpace{}%
\AgdaOperator{\AgdaFunction{⟨}}\AgdaSpace{}%
\AgdaBound{s}\AgdaSpace{}%
\AgdaOperator{\AgdaFunction{⟩ₛ}}\<%
\\
%
\>[2]\AgdaBound{A}%
\>[949I]\AgdaOperator{\AgdaFunction{⟨}}\AgdaSpace{}%
\AgdaBound{f}\AgdaSpace{}%
\AgdaOperator{\AgdaFunction{⟩ₒ}}\AgdaSpace{}%
\AgdaSymbol{=}%
\>[953I]\AgdaKeyword{record}%
\>[21]\AgdaSymbol{\{}%
\>[24]\AgdaOperator{\AgdaField{\AgdaUnderscore{}⟨\$⟩\AgdaUnderscore{}}}\AgdaSpace{}%
\AgdaSymbol{=}\AgdaSpace{}%
\AgdaOperator{\AgdaFunction{⟦}}\AgdaSpace{}%
\AgdaField{↝ₒ}\AgdaSpace{}%
\AgdaBound{t}\AgdaSpace{}%
\AgdaBound{f}\AgdaSpace{}%
\AgdaOperator{\AgdaFunction{⟧⊢}}\AgdaSpace{}%
\AgdaOperator{\AgdaFunction{∘}}\AgdaSpace{}%
\AgdaFunction{reindex}\AgdaSpace{}%
\AgdaSymbol{(}\AgdaField{↝ₛ}\AgdaSpace{}%
\AgdaBound{t}\AgdaSymbol{)}\AgdaSpace{}%
\>[20]\AgdaSymbol{;}%
\>[23]\AgdaField{cong}\AgdaSpace{}%
\AgdaSymbol{=}%
\>[31]\ensuremath{\ldots}\AgdaSpace{}\AgdaSymbol{\}}\<%
\\
%
\\[\AgdaEmptyExtraSkip]%
%
\>[2]\AgdaOperator{\AgdaFunction{〈\AgdaUnderscore{}〉}}\AgdaSpace{}%
\AgdaSymbol{:}\AgdaSpace{}%
\AgdaRecord{Algebra}\AgdaSpace{}%
\AgdaBound{Σₜ}\AgdaSpace{}%
\AgdaSymbol{→}\AgdaSpace{}%
\AgdaRecord{Algebra}\AgdaSpace{}%
\AgdaBound{Σₛ}\<%
\\
%
\>[2]\AgdaOperator{\AgdaFunction{〈}}\AgdaSpace{}%
\AgdaBound{A}\AgdaSpace{}%
\AgdaOperator{\AgdaFunction{〉}}\AgdaSpace{}%
\AgdaSymbol{=}\AgdaSpace{}%
\AgdaKeyword{record}\AgdaSpace{}%
\AgdaSymbol{\{}\AgdaSpace{}%
\AgdaOperator{\AgdaField{\AgdaUnderscore{}⟦\AgdaUnderscore{}⟧ₛ}}\AgdaSpace{}%
\AgdaSymbol{=}\AgdaSpace{}%
\AgdaBound{A}\AgdaSpace{}%
\AgdaOperator{\AgdaFunction{⟨\AgdaUnderscore{}⟩ₛ}}\AgdaSpace{}%
\AgdaSymbol{;}\AgdaSpace{}%
\AgdaOperator{\AgdaField{\AgdaUnderscore{}⟦\AgdaUnderscore{}⟧ₒ}}\AgdaSpace{}%
\AgdaSymbol{=}\AgdaSpace{}%
\AgdaBound{A}\AgdaSpace{}%
\AgdaOperator{\AgdaFunction{⟨\AgdaUnderscore{}⟩ₒ}}\AgdaSpace{}%
\AgdaSymbol{\}}\<%
\end{code}
\noindent The action of the functor on homomorphisms is also straightforward.

A more interesting example of signature morphisms and reduct algebras
is the definition of a compiler as presented in
\cite{thatcher1981more}. One defines a signature for the source
language and another one for the target language; these languages are
the term algebras over their respective signatures. A compiler is
specified by a signature morphism from the source signature to the
target signature: indeed the compiler is obtained as the unique
homomorphism from the source algebra to the reduct algebra of the
target algebra. Moreover, one can obtain a correct compiler by
providing semantics of each language as algebras and a morphism
between the source semantics and the reduct of the target
semantics.\footnote{We explored this idea by defining a correct
  compiler for an arithmetic language targeting a stack-based
  language; it can be found at the repository in
  \nolinkurl{Examples/CompilerArith.agda}.}


\newcommand{\theory}[1]{\ensuremath{\mathit{E}_{#1}}}

\subsection{Translation of theories} From a signature morphism
$m : \intSign{\Sigma_s}{\Sigma_t}$ one gets the translation of ground
\AgdaBound{Σₛ} terms as the initial homomorphism from \tsigmaeq{Σₛ} to
\AgdaOperator{\AgdaFunction{⟨}}\AgdaSpace{}\AgdaFunction{∣T∣}\AgdaSpace{}\AgdaBound{Σₜ}\AgdaSpace{}\AgdaOperator{\AgdaFunction{⟩}}. With an appropriate extension to variables, this translation
applied to a theory $\theory{s}$ over $\Sigma_s$ yields the theory
$\intTheo{s}$ over $\Sigma_t$. Moreover if
$\mathcal{A}_t\models\intTheo{s}$, one would think that the reduct
$\langle \mathcal{A}_t \rangle$ is a model of the original theory, \ie
$\langle \mathcal{A}_t \rangle \models \theory{s}$. Even better, if
$\theory{t}$ is a stronger theory than the translated theory
$\intTheo{s}$ and if $\mathcal{A}_t$ is a model for $\theory{t}$, we
would like that the reduct algebra models $\theory{s}$. In Agda such a
result would be realized as a function
\AgdaOperator{\AgdaFunction{\AgdaUnderscore{}↝T}} with
the following type (where \AgdaOperator{\AgdaFunction{↝*}}\AgdaSpace{}\AgdaBound{Eₛ} is the translation of \AgdaBound{Eₛ} and \AgdaFunction{lmap} is \AgdaFunction{map} of lists):

\begin{code}
\>[2]\AgdaOperator{\AgdaFunction{\AgdaUnderscore{}↝T}}\AgdaSpace{}%
\AgdaSymbol{:}\AgdaSpace{}%
\AgdaSymbol{∀}\AgdaSpace{}%
\AgdaSymbol{\{}\AgdaBound{ar}\AgdaSymbol{\}}\AgdaSpace{}%
\AgdaSymbol{→}\AgdaSpace{}%
\AgdaSymbol{(}\AgdaBound{Thₛ}\AgdaSpace{}%
\AgdaSymbol{:}\AgdaSpace{}%
\AgdaFunction{Theory}\AgdaSpace{}%
\AgdaBound{Σₛ}\AgdaSpace{}%
\AgdaBound{Xₛ}\AgdaSpace{}%
\AgdaBound{ar}\AgdaSymbol{)}\AgdaSpace{}%
\AgdaSymbol{→}\AgdaSpace{}%
\AgdaFunction{Theory}\AgdaSpace{}%
\AgdaBound{Σₜ}\AgdaSpace{}%
\AgdaFunction{Xₜ}\AgdaSpace{}%
\AgdaSymbol{(}\AgdaFunction{lmap}\AgdaSpace{}%
\AgdaSymbol{(}\AgdaField{↝ₛ}\AgdaSpace{}%
\AgdaBound{Σ↝}\AgdaSymbol{)}\AgdaSpace{}%
\AgdaBound{ar}\AgdaSymbol{)}\<%
\end{code}
With the morphism $m : \intSign{\Sigma_s}{\Sigma_t}$ and a function
\AgdaOperator{\AgdaFunction{↝ᵥ}}\AgdaSpace{}\AgdaSymbol{:}\AgdaSpace{}%
\AgdaSymbol{\{}\AgdaBound{s}\AgdaSpace{}\AgdaSymbol{:}\AgdaSpace{}%
\AgdaField{sorts}\AgdaSpace{}\AgdaBound{Σₛ}\AgdaSymbol{\}}\AgdaSpace{}\AgdaSymbol{→}\AgdaSpace{}\AgdaBound{Xₛ}\AgdaSpace{}\AgdaBound{s}\AgdaSpace{}\AgdaSymbol{→}\AgdaSpace{}\AgdaBound{Xₜ}\AgdaSpace{}%
\AgdaSymbol{(}\AgdaBound{m}\AgdaSpace{}\AgdaOperator{\AgdaFunction{↝ₛ}}\AgdaSpace{}\AgdaBound{s}%
\AgdaSymbol{)}\AgdaSpace{} to rename variables, one can define the
translation of open terms from \tsigmaeq{Σₛ} to \tsigmaeq{Σₜ} using
initiality.  In general, however, we cannot prove the
\emph{satisfaction property}: if a $\Sigma_t$-algebra models the
translation of an equation, then its reduct models the original
equation. The technical issue is the impossibility of defining a
$\Sigma_t$-environment from a $\Sigma_s$-environment. There is a
well-known solution which consists on restricting the set of variable
of the target signature by letting
$X_t = \bigcup_{s \in \Sigma_s , t = m \hookrightarrow s} X_s$.  Under
this restriction, we can prove the satisfaction property and
furthermore define the function
\AgdaOperator{\AgdaFunction{\AgdaUnderscore{}↝T}}. Such a restriction
over the set of variables seems to us as an impediment, which can be
alleviated if the original variables of $\theory{t}$ are included in
the calculated set of variables.

\section{Monoids and Groups: an example of equational theories}
\label{sec:extheories}

In this section we present a pleasant method, using parameterised modules, for
defining equational theories that depends in a cascade-way in which every layer
increase complexity, how is the case with algebraic structures like monoids,
groups, rings, etc.

\subsection{Monoids}

Previous to use parameterised modules, we can extend the example presented in
the Sec.~\ref{sec:univ-alg} to exemplify the main limitation of directly using a
data type to define a theory. In this prior section we define the data type
\AgdaDatatype{monoid{-}op} as an example to define the monoid signature
\AgdaFunction{monoid-sig}. Now, formalize the theory of monoids; \ie an
algebraic structure with a single associative binary operation and an identity
element. It is straightforward from apply the outline presented in
\ref{sec:eqlog-theory-ol}: first, given some family of sets of variables, say
\AgdaFunction{X}\AgdaSpace{} \AgdaSymbol{:}\AgdaSpace{} \AgdaFunction{Vars}
\AgdaFunction{monoid{-}sig}, we define
\AgdaFunction{Term}\AgdaSpace{} \AgdaSymbol{=}\AgdaSpace{} \AgdaFunction{T}
\AgdaSpace{}\AgdaFunction{monoid{-}sig}\AgdaSpace{}
\AgdaSymbol{〔}\AgdaFunction{X}\AgdaSymbol{〕}, the set of term over the
extended signature; second, we introduce the obvious smart-constructors
\AgdaBound{t}\AgdaSpace{}\AgdaOperator{\AgdaFunction{∘}}\AgdaSpace{}
\AgdaBound{t'} for the binary operation, and \AgdaFunction{u} for the identity
element; finally we write the associativity and identity element axioms.

\begin{code}
\>[4]\AgdaFunction{assocOp}\AgdaSpace{}%
\AgdaSymbol{=}\AgdaSpace{}%
\AgdaOperator{\AgdaFunction{⋀}}\AgdaSpace{}%
\AgdaSymbol{(}\AgdaFunction{x}\AgdaSpace{}%
\AgdaOperator{\AgdaFunction{∘}}\AgdaSpace{}%
\AgdaFunction{y}\AgdaSymbol{)}\AgdaSpace{}%
\AgdaOperator{\AgdaFunction{∘}}\AgdaSpace{}%
\AgdaFunction{z}\AgdaSpace{}%
\AgdaOperator{\AgdaFunction{≈}}\AgdaSpace{}%
\AgdaSymbol{(}\AgdaFunction{x}\AgdaSpace{}%
\AgdaOperator{\AgdaFunction{∘}}\AgdaSpace{}%
\AgdaSymbol{(}\AgdaFunction{y}\AgdaSpace{}%
\AgdaOperator{\AgdaFunction{∘}}\AgdaSpace{}%
\AgdaFunction{z}\AgdaSymbol{))}\<%
\\
%
\\[\AgdaEmptyExtraSkip]%
%
\>[4]\AgdaFunction{unitLeft}\AgdaSpace{}%
\AgdaSymbol{=}\AgdaSpace{}%
\AgdaOperator{\AgdaFunction{⋀}}\AgdaSpace{}%
\AgdaFunction{u}\AgdaSpace{}%
\AgdaOperator{\AgdaFunction{∘}}\AgdaSpace{}%
\AgdaFunction{x}\AgdaSpace{}%
\AgdaOperator{\AgdaFunction{≈}}\AgdaSpace{}%
\AgdaFunction{x}\<%
\\
%
\\[\AgdaEmptyExtraSkip]%
%
\>[4]\AgdaFunction{unitRight}\AgdaSpace{}%
\AgdaSymbol{=}\AgdaSpace{}%
\AgdaOperator{\AgdaFunction{⋀}}\AgdaSpace{}%
\AgdaFunction{x}\AgdaSpace{}%
\AgdaOperator{\AgdaFunction{∘}}\AgdaSpace{}%
\AgdaFunction{u}\AgdaSpace{}%
\AgdaOperator{\AgdaFunction{≈}}\AgdaSpace{}%
\AgdaFunction{x}\<%
\\
%
\\[\AgdaEmptyExtraSkip]%
%
\>[4]\AgdaFunction{MonTheory}%
\>[15]\AgdaSymbol{:}\AgdaSpace{}%
\AgdaFunction{Theory}\AgdaSpace{}%
\AgdaFunction{X}\AgdaSpace{}%
\AgdaSymbol{(}\AgdaInductiveConstructor{tt}\AgdaSpace{}%
\AgdaOperator{\AgdaInductiveConstructor{∷}}\AgdaSpace{}%
\AgdaInductiveConstructor{tt}\AgdaSpace{}%
\AgdaOperator{\AgdaInductiveConstructor{∷}}\AgdaSpace{}%
\AgdaOperator{\AgdaFunction{[}}\AgdaSpace{}%
\AgdaInductiveConstructor{tt}\AgdaSpace{}%
\AgdaOperator{\AgdaFunction{]}}\AgdaSymbol{)}\<%
\\
%
\>[4]\AgdaFunction{MonTheory}\AgdaSpace{}%
\AgdaSymbol{=}\AgdaSpace{}%
\AgdaFunction{assocOp}\AgdaSpace{}%
\AgdaOperator{\AgdaInductiveConstructor{▹}}\AgdaSpace{}%
\AgdaSymbol{(}\AgdaFunction{unitLeft}\AgdaSpace{}%
\AgdaOperator{\AgdaInductiveConstructor{▹}}\AgdaSpace{}%
\AgdaFunction{unitRight}\AgdaSpace{}%
\AgdaOperator{\AgdaInductiveConstructor{▹}}\AgdaSpace{}%
\AgdaInductiveConstructor{⟨⟩}\AgdaSymbol{)}\<%
\end{code}

\noindent
To extend this theory of monoids to the theory of commutative monoids (also
called, abelian monoids) is pretty straightforward. There is no need to extend
the signature with new constructors, neither new smart-constructors. We only
need to define the commutative axiom; thus we simply define the theory of
commutative monoids as the previous theory plus this axiom.

\begin{code}
\noindent \>[0]\AgdaFunction{commOp}\AgdaSpace{}%
\AgdaSymbol{=}\AgdaSpace{}%
\AgdaOperator{\AgdaFunction{⋀}}\AgdaSpace{}%
\AgdaSymbol{(}\AgdaFunction{x}\AgdaSpace{}%
\AgdaOperator{\AgdaFunction{∘}}\AgdaSpace{}%
\AgdaFunction{y}\AgdaSymbol{)}\AgdaSpace{}%
\AgdaOperator{\AgdaFunction{≈}}\AgdaSpace{}%
\AgdaSymbol{(}\AgdaFunction{y}\AgdaSpace{}%
\AgdaOperator{\AgdaFunction{∘}}\AgdaSpace{}%
\AgdaFunction{x}\AgdaSymbol{)}\<%
\\
%
\\[\AgdaEmptyExtraSkip]%
%
\>[0]\AgdaFunction{CommMonTheory}\AgdaSpace{}%
\AgdaSymbol{:}\AgdaSpace{}%
\AgdaFunction{Theory}\AgdaSpace{}%
\AgdaFunction{X}\AgdaSpace{}%
\AgdaSymbol{(}\AgdaInductiveConstructor{tt}\AgdaSpace{}%
\AgdaOperator{\AgdaInductiveConstructor{∷}}\AgdaSpace{}%
\AgdaInductiveConstructor{tt}\AgdaSpace{}%
\AgdaOperator{\AgdaInductiveConstructor{∷}}\AgdaSpace{}%
\AgdaInductiveConstructor{tt}\AgdaSpace{}%
\AgdaOperator{\AgdaInductiveConstructor{∷}}\AgdaSpace{}%
\AgdaInductiveConstructor{tt}\AgdaSpace{}%
\AgdaOperator{\AgdaInductiveConstructor{∷}}\AgdaSpace{}%
\AgdaInductiveConstructor{[]}\AgdaSymbol{)}\<%
\\
%
\>[0]\AgdaFunction{CommMonTheory}\AgdaSpace{}%
\AgdaSymbol{=}\AgdaSpace{}%
\AgdaFunction{commOp}\AgdaSpace{}%
\AgdaOperator{\AgdaInductiveConstructor{▹}}\AgdaSpace{}%
\AgdaFunction{MonTheory}\<%
\\

\end{code}

\subsection{Groups}

\section{Conclusions}
\label{sec:conclusions}

As far as we know, heterogeneous universal algebra has not attracted a
great interest in the academic community of type theory. In this
paper, we have developed in Agda a library with the main concepts of
heterogeneous universal algebra, up to the proof of the three
isomorphisms theorems and the freeness of the term algebra over a set
of variables. In order to define the term algebra we have introduced
heterogeneous vectors, which later turned out to be very useful in
other parts of the library, for example as the set of axioms of finite
theories and as premises of deduction rules. We further introduced a
formal system for conditional equational logic and proved its
soundness and completeness with respect to Goguen and Meseguer
semantics (we refer the reader to \cite{vidal-06} for a deeper
explanation of this result recasting it on a categorical
setting). Finally, we defined a novel representation for (derived)
signature morphisms and its associated contra-variant functor on
algebras. We also showed that, under some restrictions, this functor
also preserves models.

\textit{Related Work.} Let us contrast our work with other
formalizations covering some aspects of universal algebra. As far as
we know, since Capretta's \cite{capretta-99} first mechanization of
universal algebra and its further extension to equational logic in his
thesis, the closest new works are Kahl's \cite{kahl-2011}
formalization of allegories and the development of the algebraic
hierarchy lead by Spitters \cite{spitters-algebraic-11}. Capretta
considered only finitary signatures and his work does not encompass
signature morphisms. Spitters and his co-workers developed some very
preliminary definitions of universal algebra, because their goal is to
use the notion of variety to define the algebraic hierarchy up to the
construction of the reals; in particular they use Coq's typeclasses to
have a cleaner representation of algebraic structures.


\textit{Future Work.} We think that this development opened the path
to several further work, in particular:
\begin{enumerate}
\item a natural step is to formalize institutions;
\item consider algebras of binding structures as proposed by Fiore
\cite{fiore-2010}, Capretta's and Felty's formalization \cite{capretta/felty:2009}
of higher-order algebras might be an interesting starting point;
\item introduce multi-sorted rewriting;
\item formalize more of the mathematical theory behind universal algebra, for
  example Birkhoff's (quasi)-variety characterization; and
\item explore the idea of using completeness and soundness for
  automating the proof of identities in algebraic structures. %braibant phd
\end{enumerate}

%\begin{ack}
  We are grateful to the anonymous referees for their careful reading
  and suggestions.
%\end{ack}


\bibliographystyle{msclike}
\bibliography{biblio}

\end{document}

\section{Monoids and Groups: an example of equational theories}
\label{sec:extheories}

In this section we present our formalization of the algebraic
hierarchy. Using parameterized modules we can build each layer on top
of the previous one; the resulting structure is modular in the sense
that one can decide to parameterize the highest layer while keeping
the rest of the hierarchy intact.

\subsection{Naive formalization}
%\subsection{Monoids}

The naive approach that we used to define the signature of monoids
shows its limitations when we want to reuse it in larger signatures.
In Sec.~\ref{sec:univ-alg} we introduced the data type
\AgdaDatatype{monoid-op} as an example to define the monoid signature
\AgdaFunction{monoid-sig}; following the steps of
Sec.~\ref{sec:eqlog-theory-ol} we choose some set for variables, say
\AgdaFunction{X}\AgdaSpace{} \AgdaSymbol{:}\AgdaSpace{}
\AgdaFunction{Vars} \AgdaFunction{monoid{-}sig}, we introduce some
abbreviations for the set of terms with variables
\AgdaFunction{Term}\AgdaSpace{} \AgdaSymbol{=}\AgdaSpace{}
\AgdaFunction{T} \AgdaSpace{} \AgdaFunction{monoid{-}sig}\AgdaSpace{}
\AgdaSymbol{〔}\AgdaFunction{X}\AgdaSymbol{〕} and also
smart-constructors
\AgdaUnderscore{}\AgdaOperator{\AgdaFunction{∘}}\AgdaUnderscore{} for
the binary operation, and \AgdaFunction{u} for the identity element.
Now we can write the associativity and identity element axioms and
collect them in the theory \AgdaFunction{MonTheory}.

\begin{code}
\>[4]\AgdaFunction{assocOp}\AgdaSpace{}%
\AgdaSymbol{=}\AgdaSpace{}%
\AgdaOperator{\AgdaFunction{⋀}}\AgdaSpace{}%
\AgdaSymbol{(}\AgdaFunction{x}\AgdaSpace{}%
\AgdaOperator{\AgdaFunction{∘}}\AgdaSpace{}%
\AgdaFunction{y}\AgdaSymbol{)}\AgdaSpace{}%
\AgdaOperator{\AgdaFunction{∘}}\AgdaSpace{}%
\AgdaFunction{z}\AgdaSpace{}%
\AgdaOperator{\AgdaFunction{≈}}\AgdaSpace{}%
\AgdaSymbol{(}\AgdaFunction{x}\AgdaSpace{}%
\AgdaOperator{\AgdaFunction{∘}}\AgdaSpace{}%
\AgdaSymbol{(}\AgdaFunction{y}\AgdaSpace{}%
\AgdaOperator{\AgdaFunction{∘}}\AgdaSpace{}%
\AgdaFunction{z}\AgdaSymbol{))}\<%
\\[\AgdaEmptyExtraSkip]%
%
\>[4]\AgdaFunction{unitLeft}\AgdaSpace{}%
\AgdaSymbol{=}\AgdaSpace{}%
\AgdaOperator{\AgdaFunction{⋀}}\AgdaSpace{}%
\AgdaFunction{u}\AgdaSpace{}%
\AgdaOperator{\AgdaFunction{∘}}\AgdaSpace{}%
\AgdaFunction{x}\AgdaSpace{}%
\AgdaOperator{\AgdaFunction{≈}}\AgdaSpace{}%
\AgdaFunction{x}\<%
\\[\AgdaEmptyExtraSkip]%
%
\>[4]\AgdaFunction{unitRight}\AgdaSpace{}%
\AgdaSymbol{=}\AgdaSpace{}%
\AgdaOperator{\AgdaFunction{⋀}}\AgdaSpace{}%
\AgdaFunction{x}\AgdaSpace{}%
\AgdaOperator{\AgdaFunction{∘}}\AgdaSpace{}%
\AgdaFunction{u}\AgdaSpace{}%
\AgdaOperator{\AgdaFunction{≈}}\AgdaSpace{}%
\AgdaFunction{x}\<%
\\
%
\\[\AgdaEmptyExtraSkip]%
%
\>[4]\AgdaFunction{MonTheory}%
\>[15]\AgdaSymbol{:}\AgdaSpace{}%
\AgdaFunction{Theory}\AgdaSpace{}%
\AgdaFunction{X}\AgdaSpace{}%
\AgdaSymbol{(}\AgdaInductiveConstructor{tt}\AgdaSpace{}%
\AgdaOperator{\AgdaInductiveConstructor{∷}}\AgdaSpace{}%
\AgdaInductiveConstructor{tt}\AgdaSpace{}%
\AgdaOperator{\AgdaInductiveConstructor{∷}}\AgdaSpace{}%
\AgdaOperator{\AgdaFunction{[}}\AgdaSpace{}%
\AgdaInductiveConstructor{tt}\AgdaSpace{}%
\AgdaOperator{\AgdaFunction{]}}\AgdaSymbol{)}\<%
\\
%
\>[4]\AgdaFunction{MonTheory}\AgdaSpace{}%
\AgdaSymbol{=}\AgdaSpace{}%
\AgdaFunction{assocOp}\AgdaSpace{}%
\AgdaOperator{\AgdaInductiveConstructor{▹}}\AgdaSpace{}%
\AgdaFunction{unitLeft}\AgdaSpace{}%
\AgdaOperator{\AgdaInductiveConstructor{▹}}\AgdaSpace{}%
\AgdaFunction{unitRight}\AgdaSpace{}%
\AgdaOperator{\AgdaInductiveConstructor{▹}}\AgdaSpace{}%
\AgdaInductiveConstructor{⟨⟩}\<%
\end{code}

\noindent
Extending this theory of monoids to the theory of commutative monoids
is pretty straightforward. There is no need to extend the signature
with new constructors, neither new smart-constructors. We only need to
define the commutative axiom; thus we simply define the theory of
commutative monoids as the previous theory plus this axiom.

\begin{code}
\noindent \>[0]\AgdaFunction{commOp}\AgdaSpace{}%
\AgdaSymbol{=}\AgdaSpace{}%
\AgdaOperator{\AgdaFunction{⋀}}\AgdaSpace{}%
\AgdaSymbol{(}\AgdaFunction{x}\AgdaSpace{}%
\AgdaOperator{\AgdaFunction{∘}}\AgdaSpace{}%
\AgdaFunction{y}\AgdaSymbol{)}\AgdaSpace{}%
\AgdaOperator{\AgdaFunction{≈}}\AgdaSpace{}%
\AgdaSymbol{(}\AgdaFunction{y}\AgdaSpace{}%
\AgdaOperator{\AgdaFunction{∘}}\AgdaSpace{}%
\AgdaFunction{x}\AgdaSymbol{)}\<%
\\%
\\[\AgdaEmptyExtraSkip]%
\>[0]\AgdaFunction{CommMonTheory}\AgdaSpace{}%
\AgdaSymbol{:}\AgdaSpace{}%
\AgdaFunction{Theory}\AgdaSpace{}%
\AgdaFunction{X}\AgdaSpace{}%
\AgdaSymbol{(}\AgdaInductiveConstructor{tt}\AgdaSpace{}%
\AgdaOperator{\AgdaInductiveConstructor{∷}}\AgdaSpace{}%
\AgdaInductiveConstructor{tt}\AgdaSpace{}%
\AgdaOperator{\AgdaInductiveConstructor{∷}}\AgdaSpace{}%
\AgdaInductiveConstructor{tt}\AgdaSpace{}%
\AgdaOperator{\AgdaInductiveConstructor{∷}}\AgdaSpace{}%
\AgdaInductiveConstructor{tt}\AgdaSpace{}%
\AgdaOperator{\AgdaInductiveConstructor{∷}}\AgdaSpace{}%
\AgdaInductiveConstructor{[]}\AgdaSymbol{)}\<%
\\%
\>[0]\AgdaFunction{CommMonTheory}\AgdaSpace{}%
\AgdaSymbol{=}\AgdaSpace{}%
\AgdaFunction{commOp}\AgdaSpace{}%
\AgdaOperator{\AgdaInductiveConstructor{▹}}\AgdaSpace{}%
\AgdaFunction{MonTheory}\<%
\\
\end{code}

Next one might wonder how to to formalize the theory of groups using
our monoid theory. Now we need to extend both the signature and the
theory. The simplest way is to define a data-type with a constructor
to access all the operations of the monoid and another constructor for
the new operation.

\begin{code}
\>[0]\AgdaKeyword{data}\AgdaSpace{}%
\AgdaDatatype{group{-}op}\AgdaSpace{}%
\AgdaSymbol{:}\AgdaSpace{}%
\AgdaDatatype{List}\AgdaSpace{}%
\AgdaRecord{⊤}\AgdaSpace{}%
\AgdaOperator{\AgdaFunction{×}}\AgdaSpace{}%
\AgdaRecord{⊤}\AgdaSpace{}%
\AgdaSymbol{→}\AgdaSpace{}%
\AgdaPrimitiveType{Set}\AgdaSpace{}%
\AgdaKeyword{where}\<%
\\
\>[0][@{}l@{\AgdaIndent{0}}]%
\>[3]\AgdaInductiveConstructor{mops}\AgdaSpace{}%
\AgdaSymbol{:}\AgdaSpace{}%
\AgdaSymbol{∀}\AgdaSpace{}%
\AgdaSymbol{\{}\AgdaBound{ar}\AgdaSpace{}%
\AgdaBound{s}\AgdaSymbol{\}}\AgdaSpace{}
\AgdaSymbol{→}\AgdaSpace{}%
\AgdaSymbol{(}
\AgdaDatatype{monoid{-}op}\AgdaSpace{}%
\AgdaSymbol{(}\AgdaBound{ar}\AgdaSpace{}%
\AgdaOperator{\AgdaInductiveConstructor{,}}\AgdaSpace{}%
\AgdaBound{s}\AgdaSymbol{)}
\AgdaSymbol{)}
\AgdaSymbol{→}\AgdaSpace{}%
\AgdaDatatype{group{-}op}\AgdaSpace{}%
\AgdaSymbol{(}\AgdaBound{ar}\AgdaSpace{}%
\AgdaOperator{\AgdaInductiveConstructor{,}}\AgdaSpace{}%
\AgdaBound{s}\AgdaSymbol{)}
\<%
\\
%
\>[3]\AgdaInductiveConstructor{\AgdaUnderscore{}⁻¹}\AgdaSpace{}%
\AgdaSymbol{:}\AgdaSpace{}%
\AgdaDatatype{group{-}op}\AgdaSpace{}%
\AgdaSymbol{(}\AgdaOperator{\AgdaFunction{[}}\AgdaSpace{}%
\AgdaInductiveConstructor{tt}\AgdaSpace{}%
\AgdaOperator{\AgdaFunction{]}}\AgdaSpace{}%
\AgdaOperator{\AgdaInductiveConstructor{,}}\AgdaSpace{}%
\AgdaInductiveConstructor{tt}\AgdaSymbol{)}\<%
\\
\\
\>[0]\AgdaFunction{group{-}sig}\AgdaSpace{}%
\AgdaSymbol{:}\AgdaSpace{}%
\AgdaRecord{Signature}\<%
\\
\>[0]\AgdaFunction{group{-}sig}\AgdaSpace{}%
\AgdaSymbol{=}\AgdaSpace{}%
\AgdaKeyword{record}\AgdaSpace{}%
\AgdaSymbol{\{}\AgdaSpace{}%
\AgdaField{sorts}\AgdaSpace{}%
\AgdaSymbol{=}\AgdaSpace{}%
\AgdaRecord{⊤}\AgdaSpace{}%
\AgdaSymbol{;}\AgdaSpace{}%
\AgdaField{ops}\AgdaSpace{}%
\AgdaSymbol{=}\AgdaSpace{}%
\AgdaDatatype{group{-}op}\AgdaSpace{}%
\AgdaSymbol{\}}\<%
\end{code}

As before we define new smart-constructor and state the axioms for the inverse:
\begin{code}
\>[0]\AgdaFunction{invElemRight}\AgdaSpace{}%
\AgdaSymbol{=}\AgdaSpace{}%
\AgdaOperator{\AgdaFunction{⋀}}\AgdaSpace{}%
\AgdaSymbol{(}\AgdaFunction{x}\AgdaSpace{}%
\AgdaOperator{\AgdaFunction{∘}}\AgdaSpace{}%
\AgdaSymbol{(}\AgdaFunction{x}\AgdaSpace{}%
\AgdaOperator{\AgdaFunction{⁻}}\AgdaSymbol{))}\AgdaSpace{}%
\AgdaOperator{\AgdaFunction{≈}}\AgdaSpace{}%
\AgdaFunction{u}\<%
%
\\
\>[0]\AgdaFunction{invElemLeft}\AgdaSpace{}%
\AgdaSymbol{=}\AgdaSpace{}%
\AgdaOperator{\AgdaFunction{⋀}}\AgdaSpace{}%
\AgdaSymbol{((}\AgdaFunction{x}\AgdaSpace{}%
\AgdaOperator{\AgdaFunction{⁻}}\AgdaSymbol{)}\AgdaSpace{}%
\AgdaOperator{\AgdaFunction{∘}}\AgdaSpace{}%
\AgdaFunction{x}\AgdaSymbol{)}\AgdaSpace{}%
\AgdaOperator{\AgdaFunction{≈}}\AgdaSpace{}%
\AgdaFunction{u}\<
\end{code}
Since the signature for these two axioms is different from the
signature \AgdaFunction{MonTheory} we cannot construct the theory for
groups, say \AgdaFunction{GrpTheory}, by concatenating these axioms
with \AgdaFunction{MonTheory} (in contrast with
\AgdaFunction{CommMonTheory}). Certainly one could use some signature
morphism, but that would be too cumbersome.

\subsection{Using parameterised modules}
%\subsection{Groups}

The use of parameterised modules helps us to generalize concrete
definitions of signatures, particularly our approach is to generalize
the concrete data type that define the operations field of the
signature. Thus, we refactor the previous formalizations generalizing
the data types that define operations and also the corresponding
constructors of these operations; notice that the type
\AgdaBound{monoid{-}op} is valid only if it has at least two
constructors with the correct type. Also notice that we do not
generalize the set of sorts, but it would be necessary if we pretend
to formalize, for example, Group actions on top of Monoid actions.

\begin{code}
\>[0]\AgdaKeyword{module}%
\>[70I]\AgdaModule{Monoid}%
\>[71I]\AgdaSymbol{\{}\AgdaBound{monoid{-}op}\AgdaSpace{}%
\AgdaSymbol{:}\AgdaSpace{}%
\AgdaDatatype{List}\AgdaSpace{}%
\AgdaRecord{⊤}\AgdaSpace{}%
\AgdaOperator{\AgdaFunction{×}}\AgdaSpace{}%
\AgdaRecord{⊤}\AgdaSpace{}%
\AgdaSymbol{→}\AgdaSpace{}%
\AgdaPrimitiveType{Set}\AgdaSymbol{\}}\<%
\\
\>[.][@{}l@{}]\<[71I]%
\>[14]\AgdaSymbol{(}\AgdaBound{e}%
\>[22]\AgdaSymbol{:}\AgdaSpace{}%
\AgdaBound{monoid{-}op}\AgdaSpace{}%
\AgdaSymbol{(}\AgdaInductiveConstructor{[]}\AgdaSpace{}%
\AgdaOperator{\AgdaInductiveConstructor{,}}\AgdaSpace{}%
\AgdaInductiveConstructor{tt}\AgdaSymbol{))}\<%
\\
%
\>[14]\AgdaSymbol{(}\AgdaBound{∘}%
\>[21]\AgdaSymbol{:}\AgdaSpace{}%
\AgdaBound{monoid{-}op}\AgdaSpace{}%
\AgdaSymbol{((}\AgdaInductiveConstructor{tt}\AgdaSpace{}%
\AgdaOperator{\AgdaInductiveConstructor{∷}}\AgdaSpace{}%
\AgdaOperator{\AgdaFunction{[}}\AgdaSpace{}%
\AgdaInductiveConstructor{tt}\AgdaSpace{}%
\AgdaOperator{\AgdaFunction{]}}\AgdaSymbol{)}\AgdaSpace{}%
\AgdaOperator{\AgdaInductiveConstructor{,}}\AgdaSpace{}%
\AgdaInductiveConstructor{tt}\AgdaSymbol{))}\AgdaSpace{}%
\AgdaKeyword{where}\<%
\\%
\\[\AgdaEmptyExtraSkip]%
\>[0][@{}l@{\AgdaIndent{0}}]%
\>[2]\AgdaFunction{Σ-mon}\AgdaSpace{}%
\AgdaSymbol{:}\AgdaSpace{}%
\AgdaRecord{Signature}\<%
\\%
\>[2]\AgdaFunction{Σ-mon}\AgdaSpace{}%
\AgdaSymbol{=}\AgdaSpace{}%
\AgdaKeyword{record}\AgdaSpace{}%
\AgdaSymbol{\{}\AgdaSpace{}%
\AgdaField{sorts}\AgdaSpace{}%
\AgdaSymbol{=}\AgdaSpace{}%
\AgdaRecord{⊤}\AgdaSpace{}%
\AgdaSymbol{;}\AgdaSpace{}%
\AgdaField{ops}\AgdaSpace{}%
\AgdaSymbol{=}\AgdaSpace{}%
\AgdaBound{monoid{-}op}\AgdaSpace{}%
\AgdaSymbol{\}}\<%
\end{code}

\noindent The rest of the module is straightforward, we define a
submodule \AgdaModule{Theory} where all the definitions related to the
theory (smart-constructors, axioms and the theory) are defined.
\begin{code}
  \>[0][@{}l@{\AgdaIndent{0}}]%
  \>[2]\AgdaKeyword{module}\AgdaSpace{}%
  \AgdaModule{Theory}\AgdaSpace{}%
  \AgdaKeyword{where}
  \\
  \>[2][@{}l@{\AgdaIndent{0}}]%
  \>[4]\AgdaFunction{MonTheory}\AgdaSpace{}%
  \AgdaSymbol{=}\AgdaSpace{}%
  \AgdaFunction{assocOp}\AgdaSpace{}%
  \AgdaOperator{\AgdaInductiveConstructor{▹}}\AgdaSpace{}%
  \AgdaFunction{unitLeft}\AgdaSpace{}%
  \AgdaOperator{\AgdaInductiveConstructor{▹}}\AgdaSpace{}%
  \AgdaFunction{unitRight}\AgdaSpace{}%
  \AgdaOperator{\AgdaInductiveConstructor{▹}}\AgdaSpace{}%
  \AgdaInductiveConstructor{⟨⟩}
  \<%
\end{code}

\noindent
Following the same strategy we define the group theory with a parameterised
modules with the data type that define the operations and the corresponding
constructors for the binary operation, the unary operation and the identity
element. Notice that using the previous module definition we can instantiate
the Monoid module with the data type \AgdaBound{group-op}, then it is a monoid
with the constructors \AgdaBound{e} and \AgdaBound{∘}; thus we have the axioms of
\AgdaFunction{MonTheory} in the correct signature, and we can append the
invertibility axioms.

\begin{code}
  \>[0]\AgdaKeyword{module}%
  \>[64I]\AgdaModule{Group}%
  \>[65I]\AgdaSymbol{\{}\AgdaBound{group-op}\AgdaSpace{}%
  \AgdaSymbol{:}\AgdaSpace{}%
  \AgdaDatatype{List}\AgdaSpace{}%
  \AgdaRecord{⊤}\AgdaSpace{}%
  \AgdaOperator{\AgdaFunction{×}}\AgdaSpace{}%
  \AgdaRecord{⊤}\AgdaSpace{}%
  \AgdaSymbol{→}\AgdaSpace{}%
  \AgdaPrimitiveType{Set}\AgdaSymbol{\}}\<%
  \\
  \>[.][@{}l@{}]\<[65I]%
  \>[13]\AgdaSymbol{(}\AgdaBound{e}%
  \>[21]\AgdaSymbol{:}\AgdaSpace{}%
  \AgdaBound{group-op}\AgdaSpace{}%
  \AgdaSymbol{(}\AgdaInductiveConstructor{[]}\AgdaSpace{}%
  \AgdaOperator{\AgdaInductiveConstructor{,}}\AgdaSpace{}%
  \AgdaInductiveConstructor{tt}\AgdaSymbol{))}\<%
  \\%
  \>[13]\AgdaSymbol{(}\AgdaOperator{\AgdaBound{\AgdaUnderscore{}⁻¹}}%
  \>[21]\AgdaSymbol{:}\AgdaSpace{}%
  \AgdaBound{group-op}\AgdaSpace{}%
  \AgdaSymbol{(}\AgdaOperator{\AgdaFunction{[}}\AgdaSpace{}%
    \AgdaInductiveConstructor{tt}\AgdaSpace{}%
    \AgdaOperator{\AgdaFunction{]}}\AgdaSpace{}%
  \AgdaOperator{\AgdaInductiveConstructor{,}}\AgdaSpace{}%
  \AgdaInductiveConstructor{tt}\AgdaSymbol{))}\<%
  \\%
  \>[13]\AgdaSymbol{(}\AgdaBound{∘}%
  \>[20]\AgdaSymbol{:}\AgdaSpace{}%
  \AgdaBound{group-op}\AgdaSpace{}%
  \AgdaSymbol{((}\AgdaInductiveConstructor{tt}\AgdaSpace{}%
  \AgdaOperator{\AgdaInductiveConstructor{∷}}\AgdaSpace{}%
  \AgdaOperator{\AgdaFunction{[}}\AgdaSpace{}%
    \AgdaInductiveConstructor{tt}\AgdaSpace{}%
    \AgdaOperator{\AgdaFunction{]}}\AgdaSymbol{)}\AgdaSpace{}%
  \AgdaOperator{\AgdaInductiveConstructor{,}}\AgdaSpace{}%
  \AgdaInductiveConstructor{tt}\AgdaSymbol{))}\<%
  \\
  \>[.][@{}l@{}]\<[64I]%
  \>[7]\AgdaKeyword{where}\<%
  \\
  \>[0][@{}l@{\AgdaIndent{0}}]%
  \>[2]\AgdaKeyword{open}\AgdaSpace{}%
  \AgdaKeyword{module}\AgdaSpace{}%
  \AgdaModule{M}\AgdaSpace{}%
  \AgdaSymbol{=}\AgdaSpace{}%
  \AgdaModule{Monoid}\AgdaSpace{}%
  \AgdaSymbol{\{}\AgdaBound{group-op}\AgdaSymbol{\}}\AgdaSpace{}%
  \AgdaBound{e}\AgdaSpace{}%
  \AgdaBound{∘}\<%
  \\
  %
  \\[\AgdaEmptyExtraSkip]%
  %
  \>[2]\AgdaFunction{Σ-grp}\AgdaSpace{}%
  \AgdaSymbol{:}\AgdaSpace{}%
  \AgdaRecord{Signature}\<%
  \\
  %
  \>[2]\AgdaFunction{Σ-grp}\AgdaSpace{}%
  \AgdaSymbol{=}\AgdaSpace{}%
  \AgdaKeyword{record}\AgdaSpace{}%
  \AgdaSymbol{\{}\AgdaSpace{}%
  \AgdaField{sorts}\AgdaSpace{}%
  \AgdaSymbol{=}\AgdaSpace{}%
  \AgdaRecord{⊤}\AgdaSpace{}%
  \AgdaSymbol{;}\AgdaSpace{}%
  \AgdaField{ops}\AgdaSpace{}%
  \AgdaSymbol{=}\AgdaSpace{}%
  \AgdaBound{group-op}\AgdaSpace{}%
  \AgdaSymbol{\}}\<%
  \\%
  \\[\AgdaEmptyExtraSkip]%
  %
  \>[2]\AgdaKeyword{module}\AgdaSpace{}%
  \AgdaModule{GrpTheory}\AgdaSpace{}%
  \AgdaKeyword{where}\<%
  \\
  \>[2][@{}l@{\AgdaIndent{0}}]%
  \>[4]\AgdaKeyword{open}\AgdaSpace{}%
  \AgdaModule{M.Theory}\AgdaSpace{}%
  \\
  \>[4]$\vdots$
  \\
  \>[4]\AgdaFunction{GrpTheory}\AgdaSpace{}%
  \AgdaSymbol{=}\AgdaSpace{}%
  \AgdaFunction{invElemRight}\AgdaSpace{}%
  \AgdaOperator{\AgdaInductiveConstructor{▹}}\AgdaSpace{}%
  \AgdaFunction{invElemLeft}\AgdaSpace{}%
  \AgdaOperator{\AgdaInductiveConstructor{▹}}\AgdaSpace{}%
  \AgdaFunction{MonTheory}\<%
  \<%
\end{code}


\noindent
Using this approach we formalized Monoids, Commutative Monoids, Groups,
Commutative Groups, Semirings, Ring and Commutative Rings.

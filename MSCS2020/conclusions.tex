\section{Conclusions}
\label{sec:conclusions}

As far as we know, heterogeneous universal algebra has not attracted a
great interest in the academic community of type theory. In this
paper, we have developed in Agda a library with the main concepts of
heterogeneous universal algebra, up to the proof of the three
isomorphisms theorems and the freeness of the term algebra over a set
of variables. In order to define the term algebra we have introduced
heterogeneous vectors, which later turned out to be very useful in
other parts of the library, for example as the set of axioms of finite
theories and as premises of deduction rules. We further introduced a
formal system for conditional equational logic and proved its
soundness and completeness with respect to Goguen and Meseguer
semantics (we refer the reader to \cite{vidal-06} for a deeper
explanation of this result recasting it on a categorical
setting). Finally, we defined a novel representation for (derived)
signature morphisms and its associated contra-variant functor on
algebras. We also showed that, under some restrictions, this functor
also preserves models.

\textit{Related Work.} Let us contrast our work with other
formalizations covering some aspects of universal algebra. As far as
we know, since Capretta's \citeyearpar{capretta-99} first mechanization of
universal algebra and its further extension to equational logic in his
thesis, the closest new works are Kahl's \citeyearpar{kahl-2011}
formalization of allegories and the development of the algebraic
hierarchy by \cite{spitters-algebraic-11}. Capretta
considered only finitary signatures and his work does not encompass
signature morphisms. Spitters and his co-workers developed some very
preliminary definitions of universal algebra, because their goal is to
use the notion of variety to define the algebraic hierarchy up to the
construction of the reals; in particular they use Coq's typeclasses to
have a cleaner representation of algebraic structures.


\textit{Future Work.} We plan to extend the library to cover more
material of the standard references on universal algebra. This has
several independent lines of development. First we want to finish the
proof of Birkhoff's (quasi)-variety characterization theorem; this
implies to introduce new concepts (like subdirect products) and review
some of our current design choices (should we allow for infinite
theories). Congruences play an import rôle in universal algebra and we
lack important notions (permutable congruence) and results (for
example that congruences form a complete lattice). With regard to
universal algebraist, it would also be nice to have an interface for
monosorted algebras so to get rid of the nuances of the multisorted
setting.

Our interest in including translations of signatures arose as we tried
to understand compilation in an algebraic setting (we include an
example of a correct-by-construction compiler in our library). The
results in Sec.~\ref{sec:trans} almost show that equational logic can
be seen as an institution\footnote{The only gap are the proofs that
  composition of signature morphisms is associative and that the
  identity is neutral.} It might be interesting (but this is not our
priority) to explore the use specification-building operations to
build equational theories from smaller ones; for example, abelian
groups can be obtained as the union of two theories
\citep[see][p. 232]{sannella2012foundations}.

Somewhat related to our work are the algebras of binding structures as
proposed by \cite{fiore-2010}. Capretta's and Felty's
\citeyearpar{capretta/felty:2009} formalization of higher-order
algebras might be an interesting starting point.

% We think that this development opened the path
% to several further work, in particular:
% \begin{enumerate}
% \item 
% \item introduce multi-sorted rewriting; and
% \item explore the idea of using completeness and soundness for
%   automating the proof of identities in algebraic structures. %braibant phd
% \end{enumerate}

%\begin{ack}

\paragraph*{Acknowledgments}
  We are grateful to the anonymous referees for their careful reading
  and suggestions. We are grateful to Pedro Sánchez Terraf for discussions
  about universal algebra.
%\end{ack}

%%% Local Variables: ***
%%% mode:latex ***
%%% TeX-master: "univ-alg.tex"  ***
%%% End: ***

\documentclass[11pt]{beamer}
%\documentclass[11pt,trans]{beamer}
\usetheme{Darmstadt}
\newcommand{\beginbackup}{
   \newcounter{framenumbervorappendix}
   \setcounter{framenumbervorappendix}{\value{framenumber}}
}
\newcommand{\backupend}{
   \addtocounter{framenumbervorappendix}{-\value{framenumber}}
   \addtocounter{framenumber}{\value{framenumbervorappendix}} 
}

\setbeamertemplate{navigation symbols}{}
\setbeamertemplate{footline}{ 
  \hbox{%
  \begin{beamercolorbox}[wd=.50\paperwidth,ht=2.25ex,dp=1ex,center]{white}%
  \end{beamercolorbox}%
  \begin{beamercolorbox}[wd=.20\paperwidth,ht=2.25ex,dp=1ex,center]{white}%
  \end{beamercolorbox}%
  \begin{beamercolorbox}[wd=.30\paperwidth,ht=2.25ex,dp=1ex,left]{title in head/foot}%
    \usebeamerfont{title in head/foot}\hspace{1em}\textbf{\insertsection} \hspace*{2em}
    \textbf{\insertframenumber{} / \inserttotalframenumber}\hspace*{0.5em}
  \end{beamercolorbox}
  }%
  \vskip0pt%
}
\setbeamertemplate{footline}{}
\setbeamertemplate{headline}{
\hbox{
  \begin{beamercolorbox}[wd=\paperwidth,ht=2.25ex,dp=1ex,right]{white}%
    \color{black}\textbf{\insertframenumber{} /\inserttotalframenumber}\hspace*{2em}
  \end{beamercolorbox}}
  \vspace{-.4cm}
}

\setbeamertemplate{section in toc}{\leftskip=2ex%
    \usebeamerfont*{section number projected}%
    \usebeamercolor{section number projected}%
    \begin{pgfpicture}{-1ex}{2pt}{1ex}{2.2ex}
      \color{bg}
      \pgfpathcircle{\pgfpoint{0pt}{1.5ex}}{1.8ex}
      \pgfusepath{fill}
      \pgftext[bottom,y=3pt]{\color{fg}\inserttocsectionnumber}
    \end{pgfpicture}\kern1em%
  \inserttocsection\par}

\usepackage{amsmath}
\usepackage{amssymb}
\usepackage{mathpartir}
\usepackage{minted}
\usepackage{bussproofs}


\usepackage{fontspec}
\setmonofont{DejaVu Sans Mono}

\let\emph\relax
\DeclareTextFontCommand{\emph}{\bfseries\em}

\title{Formalization of Universal Algebra in Agda}
\author{\underline{Emmanuel Gunther}, Alejandro Gadea and Miguel Pagano}
\institute{FAMAF, Universidad Nacional de Cordoba, Cordoba, Argentina}
\date{}
\newcommand{\titlePoint}[1]{#1}
\newcommand{\point}[1]{\textcolor{gray}{#1}}

\begin{document}
\begin{frame}
  \titlepage
\end{frame}

\begin{frame}
  \tableofcontents
\end{frame}

\section{Why this work?}

\begin{frame}{Why this work?}
  \begin{itemize}[<+->]
  \item \titlePoint{What is Universal Algebra?}
    \point{A common and abstract theory for different algebraic structures.}

  \item Importance of Universal Algebra in Computer Science.
    \point{Specification of datatypes, semantics and compilers. To conceive
      a language as an initial algebra of a signature.}

  \item Our first motivation in Universal Algebra: To develop an algebraic correct
    compiler certified in Type Theory.
    \point{It involves the notions of \emph{derived signature morphisms} and
      \emph{reduct algebras}.}

  \item Available formalizations were not suitable.
  \end{itemize}
\end{frame}

\section{What we did?} 

\begin{frame}{Formalization of Universal Algebra in Type Theory}
  
  \begin{block}{Previous Works}
      Out-dated or restricted to some particular algebras.
    \point{
      \begin{itemize}
      \item Capretta: Formalization in Coq (1999). 
      \item Kahl: Formalization of allegories (too broad).  
      \item Spitters: Development of the algebraic hierarchy.
    \end{itemize}
    }
   \end{block}

   \pause
  \begin{block}{Our paper}
  \begin{itemize}
     \item Main definitions of Heterogeneous Universal Algebra in Agda.
     \item Some basic results: for example, the three isomorphism theorems.
     \item Conditional equational calculus, soundness and completeness.
     \item First formalization of derived signature morphism and
          reduct algebras.
  \end{itemize}
\end{block}
\end{frame}


\section{Heterogeneous Algebras}

\begin{frame}{Heterogeneous Signature}
  \begin{itemize}[<+->]
    \setlength\itemsep{1em}
  \item An heterogeneous signature consists of a set of \emph{sorts} and a
    set of \emph{operations symbols} $(f,[s_1,...,s_n],s)$

  \item Operation symbols are formalized as a family indexed on sorts.

  \item This allows to define predicates and functions over operations
    in a type-theoretical way.
  \end{itemize}

\end{frame}

\begin{frame}[fragile]{Heterogeneous Signature, example}
  \begin{block}{}
    \begin{minted}{agda}
record Signature : Set where 
  field
    sorts  : Set
    ops    : (List sorts) × sorts → Set
    \end{minted}
  \end{block}
  \pause

  \begin{block}{Multi-sorted Signature}
    \begin{minted}[fontsize=\footnotesize]{agda}
data ESorts : Set where
  N : ESorts
  B : ESorts
  
data EOps : List ESorts × ESorts → Set where
  consN  : (n : ℕ) → EOps ([] , N)
  plus   : EOps (N ∷ N ∷ [] , N)
  eqN    : EOps (N ∷ N ∷ [] , B)

ExprSig : Signature
ExprSig = record  { sorts  = ESorts
                  ; ops    = EOps }
    \end{minted}

  \end{block}
\end{frame}

\begin{frame}[fragile]{Heterogeneous Signature}
  \begin{block}{}
\begin{minted}{agda}
record Signature : Set where 
  field
    sorts  : Set
    ops    : (List sorts) × sorts → Set
    \end{minted}
  \end{block}


  \begin{block}{A Monosorted Signature}
    \begin{minted}{agda}
data op-mon : List ⊤ × ⊤ → Set where
  e    : op-mon ([] , tt)
  op   : op-mon ([ tt , tt ] , tt)

Σ-mon : Signature
Σ-mon = record { sorts = ⊤ ; ops = op-mon }
    \end{minted}

  \end{block}
\end{frame}


\begin{frame}[fragile]{Heterogeneous Algebra}
  \begin{itemize}[<+->]
    \setlength\itemsep{1em}
  \item An heterogeneous algebra consists of the interpretation of sorts (called carriers) and operations of
    a signature.

  \item Constructively, one does not deal with sets for representing carriers, but with setoids.
    \point{Setoids make explicit the equality of its underlying set.
      Quotient sets are easily defined.}
    
  \item Each operation is interpreted as a function from
    the product of the interpretation of its arity to the
    interpretation of its target.
    
  \item We formalized the product of carriers with
    \emph{heterogeneous vectors}: collections of elements indexed by lists
    of sorts.
  \end{itemize}
\end{frame}

\begin{frame}[fragile]{Heterogeneous Algebra}
  \begin{block}{Setoids, from Agda's stdlib}
    \begin{minted}{agda}
record Setoid : Set where
  field
    Carrier       : Set 
    _≈_           : Carrier → Carrier → Set
    isEquivalence : IsEquivalence _≈_
      \end{minted}
  \end{block}
  \pause
  \begin{block}{Heterogeneous Vectors}
    \begin{minted}[fontsize=\footnotesize]{agda}
data HVec {I : Set} (A : I -> Set) : List I → Set where
  ⟨⟩  : HVec A []
  _▹_ : ∀ {i is} → A i → HVec A is → HVec A (i ∷ is)
       \end{minted}
     \end{block}
     Heterogeneous vectors turned to be useful in other contexts of
     our development (theories, deduction rules).
\end{frame}

\begin{frame}[fragile]{Basic Constructions and Some Results}
  \begin{itemize}[<+->]
  \setlength\itemsep{1em}
\item Important algebraic concepts:
\emph{Subalgebras}, \emph{Homomorphism}, \emph{Congruences}.
      
\item Congruences give rise to \emph{Quotient algebras}.
  
\item Quotient are easily defined thanks to setoids:
keep the carriers and use the congruence as the new equality.

\item These constructions are related by the three \emph{isomorphism
    theorems}.

\item We also defined the \emph{term algebra} and proved it \emph{initial}.
\end{itemize}
\end{frame}


\section{Equational Logic}

\begin{frame}[fragile]{Conditional Equational Logic}

\begin{itemize}[<+->]
\item The meta-theory of multisorted equational logic presents some
  differences with respect to the single-sorted case.

\item We formalized the deductive system for multisorted conditional
  equational logic as presented in the work of Goguen and Lin.

\item We proved soundness and completeness of this calculus.

% \begin{figure}[t]
%   \centering
%   \bottomAlignProof
%   \AxiomC{}
%   \UnaryInfC{$ {\scriptstyle E \vdash \forall X,\, t = t } $}
%   \DisplayProof\hspace{2ex}
% %
%   \bottomAlignProof
%   \AxiomC{$ {\scriptstyle E \vdash \forall X,\, t_0 = t_1 } $}
%   \UnaryInfC{$ {\scriptstyle E \vdash \forall X,\, t_1 = t_0 } $}
%   \DisplayProof \hspace{2ex}
% % 
%  \bottomAlignProof
%  \AxiomC{$ {\scriptstyle E \vdash \forall X,\, t_0 = t_1 }$}
%   \AxiomC{$ {\scriptstyle E \vdash \forall X,\, t_1 = t_2} $}
%   \BinaryInfC{$ {\scriptstyle E \vdash \forall X,\, t_0 = t_2} $}
%   \DisplayProof
% \\[6pt]
%   \AxiomC{$ {\scriptstyle \forall X,\,t = t' \ \mathsf{if}\
%       t_1=t'_1,\ldots, t_n=t'_n \in E} $}
%   \AxiomC{$ {\scriptstyle E \vdash \forall X,\,\sigma(t_i) = \sigma(t'_i)} $}
%   \RightLabel{$ {\scriptstyle \sigma \colon X \rightarrow E_\Sigma(X)} $}
%   \BinaryInfC{$ {\scriptstyle E \vdash \forall X,\, \sigma(t) = \sigma(t')} $}
%   \DisplayProof
% \\[6pt]
%   \AxiomC{$ {\scriptstyle E \vdash \forall X,\, t_1 = t'_1} $}
%   \AxiomC{$ {\scriptstyle \cdots} $}
%   \AxiomC{$ {\scriptstyle E \vdash \forall X,\, t_n = t'_n} $}
%   \RightLabel{$ {\scriptstyle f : [s_1,...,s_{n}] \Rightarrow_{\Sigma} s} $}
%   \TrinaryInfC{$ {\scriptstyle E \vdash \forall X,\, f\,(t_1,\ldots,t_n) = f\,(t'_1,\ldots,t'_n)} $}
%   \DisplayProof
%   \label{fig:deduction}
% \end{figure}
\end{itemize}
  
\end{frame}

\begin{frame}[fragile]{Conditional Equational Logic: an Example}

  \begin{block}{Propositional Logic}
    \begin{minted}[fontsize=\scriptsize]{agda}
data Σops : List ⊤ × ⊤ → Set where
  t        : Σops₁ ([] ↦ tt)
  f        : Σops₁ ([] ↦ tt)
  neg      : Σops₁ ([ tt ] ↦ tt)
  and , or : Σops₁ ((tt ∷ [ tt ]) ↦ tt)

Σbool : Signature
Σbool = record { sorts = ⊤ ; ops = Σops }
\end{minted}
%
\pause
%    
\begin{minted}[fontsize=\scriptsize]{agda}
_∧_ , _∨_ : BTerm → BTerm → BTerm   -- and, or operators
φ ∧ ψ = term and ⟨⟨ φ , ψ ⟩⟩
  
¬ : BTerm → BTerm                   -- neg operator
¬ φ = term neg₁ ⟪ φ ⟫   

p , q : BTerm                        -- variables
\end{minted}
\pause
\begin{minted}[fontsize=\scriptsize]{agda}
axAbs₁ , ax3excl, axAssoc∧ , ... : EqBool
axAbs₁ = ⋀ p ∧ (p ∨ q) ≈ p

Tbool : Theory ΣBool X (tt ∷ tt ∷ ... ∷ [ tt ])
Tbool = axAbs₁ ▹ (ax3excl ▹ axAssoc∧ ▹ ... ▹ ⟨⟩)
\end{minted}
    
\end{block}
\end{frame}

\begin{frame}[fragile]{Conditional Equational Logic: Example}

\begin{block}{Neutral element of conjunction}
\begin{minted}[fontsize=\footnotesize]{agda}
Neu∧ : Tbool ⊢ ⋀ p ∧ true ≈ p
Neu∧ =
begin
  p ∧ true

≈⟨ preemp ∼⟨⟨ prefl , psym (ax3excl ∣ idSubst) ⟩⟩∼ ⟩

  p ∧ (p ∨ (¬ p))

≈⟨ axAbs₁ ∣ (λ q ⟶ ¬ p) ) ⟩

  p

∎
\end{minted}
\end{block}
\end{frame}



\section{Derived Signature Morphisms}

\begin{frame}{Derived Signature Morphisms}
  \begin{itemize}[<+->]
  \item When we wanted to define a correct compiler via universal
    algebra we rediscovered the concept of \emph{derived signature
      morphism}.

  \item Derived signature morphisms allows to define translations in a
    generic way.

     \begin{block}{Example}
       $\Sigma_{BOOL} = \{ T,F,\neg,\wedge,\vee \}$ $\;\;\;\;\Sigma_{DS} = \{ T,F,\vee,\equiv \}$

       \medskip
       $\Sigma_{BOOL} \leadsto \Sigma_{DS}$

       $\;\;\;T \;\;\;\;\;\;\;\;\leadsto T$

       $\;\;\;F \;\;\;\;\;\;\;\;\leadsto F$
       
       $\;\;\;\neg x \;\;\;\;\;\;\leadsto x \equiv F$
       
       $\;\;\;(x \wedge y) \leadsto (x \equiv y) \equiv (x \vee y)$
     \end{block}
  
  \end{itemize}
\end{frame}

\begin{frame}{Translations}
  \begin{itemize}[<+->]
  \item A derived signature morphism maps sorts and operations from
    one signature to another. Each operation is mapped to a
    combination of operations preserving arities.


  \item From a derived signature morphism from $\Sigma_1$ to
    $\Sigma_2$, it is possible to define a $\Sigma_2$-algebra from a
    $\Sigma_1$-algebra. This concept is known as \emph{reduct
      algebra}.

    
  \item Translation of operations are formalized by defining
    meta-terms over signatures.
    \point{Having the operations defined
      as indexed families makes the preservation of arities trivial.}

  \item This is the first formalization in Type Theory of derived
    signature morphisms and reduct algebras.

  \end{itemize}
\end{frame}

\section{Conclusion and future work}

\begin{frame}{Conclusions}
  \begin{itemize}[<+->]
    
  \item Universal Algebra plays an important role in Computer Science
    and there is not much work done in Type Theory.
    
  \item We have developed in Agda a library with the main concepts of
    Heterogeneous Universal Algebra and some basic results.

  \item We presented a novel representation of heterogeneous signatures
    and an independent library of heterogeneous vectors.  

  \item We formalized a deduction system for conditional equational
    logic and proved its soundness and completeness.

  \item Finally we defined the first representation of derived
    signature morphisms and reduct algebras in Type Theory.
        
  \end{itemize}
\end{frame}

\begin{frame}{Future work}
  % From this formalization we can continue in several directions.

  \begin{itemize}[<+->]
    \setlength\itemsep{1em}
  \item Formalize more results of Heterogeneous Universal Algebra, like Birkhoff's
    (quasy)-variety characterization.
    
  \item Formalize different logics and the translation between them: Institutions.
    
  \item Formalize Rewriting Systems Modulo axioms and Decision procedures.

  \end{itemize}
\end{frame}


\begin{frame}{Thank you}

  Questions?
  
\end{frame}



\end{document}





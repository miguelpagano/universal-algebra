\section{Monoids and Groups: an example of equational theories}
\label{sec:extheories}

In this section we present a pleasant method, using parameterised modules, for
defining equational theories that depends in a cascade-way in which every layer
increase complexity, how is the case with algebraic structures like monoids,
groups, rings, etc.

\subsection{Monoids}

Previous to use parameterised modules, we can extend the example presented in
the Sec.~\ref{sec:univ-alg} to exemplify the main limitation of directly using a
data type to define a theory. In this prior section we define the data type
\AgdaDatatype{monoid-op} as an example to define the monoid signature
\AgdaFunction{monoid-sig}. It is straightforward to formalize the theory of
monoids (\ie an algebraic structure with a single associative binary operation
and an identity element) by applying the outline presented in
\ref{sec:eqlog-theory-ol}: first, given some family of sets of variables, say
\AgdaFunction{X}\AgdaSpace{} \AgdaSymbol{:}\AgdaSpace{} \AgdaFunction{Vars}
\AgdaFunction{monoid{-}sig}, we define \AgdaFunction{Term}\AgdaSpace{}
\AgdaSymbol{=}\AgdaSpace{} \AgdaFunction{T} \AgdaSpace{}
\AgdaFunction{monoid{-}sig}\AgdaSpace{}
\AgdaSymbol{〔}\AgdaFunction{X}\AgdaSymbol{〕}, the set of terms over the
extended signature; second, we introduce the obvious smart-constructors
\AgdaUnderscore{}\AgdaOperator{\AgdaFunction{∘}}\AgdaUnderscore{} for the
binary operation, and \AgdaFunction{u} for the identity element; finally we
write the associativity and identity element axioms.

\begin{code}
\>[4]\AgdaFunction{assocOp}\AgdaSpace{}%
\AgdaSymbol{=}\AgdaSpace{}%
\AgdaOperator{\AgdaFunction{⋀}}\AgdaSpace{}%
\AgdaSymbol{(}\AgdaFunction{x}\AgdaSpace{}%
\AgdaOperator{\AgdaFunction{∘}}\AgdaSpace{}%
\AgdaFunction{y}\AgdaSymbol{)}\AgdaSpace{}%
\AgdaOperator{\AgdaFunction{∘}}\AgdaSpace{}%
\AgdaFunction{z}\AgdaSpace{}%
\AgdaOperator{\AgdaFunction{≈}}\AgdaSpace{}%
\AgdaSymbol{(}\AgdaFunction{x}\AgdaSpace{}%
\AgdaOperator{\AgdaFunction{∘}}\AgdaSpace{}%
\AgdaSymbol{(}\AgdaFunction{y}\AgdaSpace{}%
\AgdaOperator{\AgdaFunction{∘}}\AgdaSpace{}%
\AgdaFunction{z}\AgdaSymbol{))}\<%
\\
%
\\[\AgdaEmptyExtraSkip]%
%
\>[4]\AgdaFunction{unitLeft}\AgdaSpace{}%
\AgdaSymbol{=}\AgdaSpace{}%
\AgdaOperator{\AgdaFunction{⋀}}\AgdaSpace{}%
\AgdaFunction{u}\AgdaSpace{}%
\AgdaOperator{\AgdaFunction{∘}}\AgdaSpace{}%
\AgdaFunction{x}\AgdaSpace{}%
\AgdaOperator{\AgdaFunction{≈}}\AgdaSpace{}%
\AgdaFunction{x}\<%
\\
%
\\[\AgdaEmptyExtraSkip]%
%
\>[4]\AgdaFunction{unitRight}\AgdaSpace{}%
\AgdaSymbol{=}\AgdaSpace{}%
\AgdaOperator{\AgdaFunction{⋀}}\AgdaSpace{}%
\AgdaFunction{x}\AgdaSpace{}%
\AgdaOperator{\AgdaFunction{∘}}\AgdaSpace{}%
\AgdaFunction{u}\AgdaSpace{}%
\AgdaOperator{\AgdaFunction{≈}}\AgdaSpace{}%
\AgdaFunction{x}\<%
\\
%
\\[\AgdaEmptyExtraSkip]%
%
\>[4]\AgdaFunction{MonTheory}%
\>[15]\AgdaSymbol{:}\AgdaSpace{}%
\AgdaFunction{Theory}\AgdaSpace{}%
\AgdaFunction{X}\AgdaSpace{}%
\AgdaSymbol{(}\AgdaInductiveConstructor{tt}\AgdaSpace{}%
\AgdaOperator{\AgdaInductiveConstructor{∷}}\AgdaSpace{}%
\AgdaInductiveConstructor{tt}\AgdaSpace{}%
\AgdaOperator{\AgdaInductiveConstructor{∷}}\AgdaSpace{}%
\AgdaOperator{\AgdaFunction{[}}\AgdaSpace{}%
\AgdaInductiveConstructor{tt}\AgdaSpace{}%
\AgdaOperator{\AgdaFunction{]}}\AgdaSymbol{)}\<%
\\
%
\>[4]\AgdaFunction{MonTheory}\AgdaSpace{}%
\AgdaSymbol{=}\AgdaSpace{}%
\AgdaFunction{assocOp}\AgdaSpace{}%
\AgdaOperator{\AgdaInductiveConstructor{▹}}\AgdaSpace{}%
\AgdaSymbol{(}\AgdaFunction{unitLeft}\AgdaSpace{}%
\AgdaOperator{\AgdaInductiveConstructor{▹}}\AgdaSpace{}%
\AgdaFunction{unitRight}\AgdaSpace{}%
\AgdaOperator{\AgdaInductiveConstructor{▹}}\AgdaSpace{}%
\AgdaInductiveConstructor{⟨⟩}\AgdaSymbol{)}\<%
\end{code}

\noindent
Extending this theory of monoids to the theory of commutative monoids (also
called, abelian monoids) is pretty straightforward. There is no need to extend
the signature with new constructors, neither new smart-constructors. We only
need to define the commutative axiom; thus we simply define the theory of
commutative monoids as the previous theory plus this axiom.

\begin{code}
\noindent \>[0]\AgdaFunction{commOp}\AgdaSpace{}%
\AgdaSymbol{=}\AgdaSpace{}%
\AgdaOperator{\AgdaFunction{⋀}}\AgdaSpace{}%
\AgdaSymbol{(}\AgdaFunction{x}\AgdaSpace{}%
\AgdaOperator{\AgdaFunction{∘}}\AgdaSpace{}%
\AgdaFunction{y}\AgdaSymbol{)}\AgdaSpace{}%
\AgdaOperator{\AgdaFunction{≈}}\AgdaSpace{}%
\AgdaSymbol{(}\AgdaFunction{y}\AgdaSpace{}%
\AgdaOperator{\AgdaFunction{∘}}\AgdaSpace{}%
\AgdaFunction{x}\AgdaSymbol{)}\<%
\\
%
\\[\AgdaEmptyExtraSkip]%
%
\>[0]\AgdaFunction{CommMonTheory}\AgdaSpace{}%
\AgdaSymbol{:}\AgdaSpace{}%
\AgdaFunction{Theory}\AgdaSpace{}%
\AgdaFunction{X}\AgdaSpace{}%
\AgdaSymbol{(}\AgdaInductiveConstructor{tt}\AgdaSpace{}%
\AgdaOperator{\AgdaInductiveConstructor{∷}}\AgdaSpace{}%
\AgdaInductiveConstructor{tt}\AgdaSpace{}%
\AgdaOperator{\AgdaInductiveConstructor{∷}}\AgdaSpace{}%
\AgdaInductiveConstructor{tt}\AgdaSpace{}%
\AgdaOperator{\AgdaInductiveConstructor{∷}}\AgdaSpace{}%
\AgdaInductiveConstructor{tt}\AgdaSpace{}%
\AgdaOperator{\AgdaInductiveConstructor{∷}}\AgdaSpace{}%
\AgdaInductiveConstructor{[]}\AgdaSymbol{)}\<%
\\
%
\>[0]\AgdaFunction{CommMonTheory}\AgdaSpace{}%
\AgdaSymbol{=}\AgdaSpace{}%
\AgdaFunction{commOp}\AgdaSpace{}%
\AgdaOperator{\AgdaInductiveConstructor{▹}}\AgdaSpace{}%
\AgdaFunction{MonTheory}\<%
\\
\end{code}

\subsection{Groups}

\section{Equational Logic}
\label{sec:eqlog}

In this section we introduce the notion of (conditional) equational
theories and the corresponding notion of satisfiability of theories
by algebras. Moreover we formalize (conditional) equational logic as
presented by Goguen and Lin \cite{goguen2005specifying} and prove that
the deduction system is sound and complete.

\subsection{Free algebra with variables}
The term algebra we have just defined contained only \emph{ground}
terms, \ie terms without variables. Given a signature \codigo{Σ| and }X :
sorts Σ → Set| a family of variables, we define a new signature
extending \codigo{Σ| with |X} by taking the variables as new constants
(i.e. , operations with arity []).
\begin{spec}
  _〔_〕 : (Σ : Signature) → (X : sorts Σ → Set) → Signature
  Σ 〔 X 〕 = record  { sorts = sorts Σ ; ops =  ops' }
     where   ops' ([] , s)   = ops Σ ([] , s) ⊎ X s
             ops' (ar , s)   = ops Σ (ar , s)
\end{spec}%
\noindent Note that it is easy to refer to constant operations and
extend them, because we indexed the set of operations on their arity
and target sort.

It is easy to turn the term algebra of the extended signature
into an algebra for the original signature:
\begin{spec}
∣T∣_〔_〕 : (Σ : Signature) → (X : sorts Σ → Set) → Algebra Σ
∣T∣ Σ 〔 X 〕  = record { _⟦_⟧ₛ = \codigo{T} (Σ 〔 X 〕) ⟦_⟧ₛ , _⟦_⟧ₒ = io }
  where  io {[]}  f  = \codigo{T} (Σ 〔 X 〕) ⟦ inj₁ f ⟧ₒ
         io {ar}  f  = \codigo{T} (Σ 〔 X 〕) ⟦ f ⟧ₒ
\end{spec}
\noindent The only difference with the algebra of ground terms is that
we inject constants from \codigo{Σ} to distinguish them from variables. In
order to interpret terms with variables we need \emph{environments} to
give meaning to variables.

Let \codigo{Env X A = ∀ {s} → X s → ∥ A ⟦ s ⟧ₛ ∥ } be the set of
environments from \codigo{X| to |A|.  The free algebra | ∣T∣ Σ 〔 X 〕} has the
universal \emph{freeness} property: given \codigo{A : Algebra Σ} and an
environment \codigo{θ : Env X A|, there exists an unique homomorphism ⟦_⟧θ} :
\codigo{Homo (∣T∣ Σ 〔 X 〕) A}\codigo{ such that |⟦ x ⟧θ = θ(x)| for |x ∈ X}.

\subsection{Satisfiability and provability}

\paragraph{Equations} In the mono-sorted setting an equation is a pair
of terms where all the variables are assumed to be universally
quantified and an equational theory is a (finite) set of equations.
In a  multi-sorted setting both sides of an equation should be terms
of the same sort. Moreover we allow quasi-identities which we
write as conditional equations:
\[ t = t'\ \mathsf{if} \  t_1 = t'_1, \ldots, t_n = t'_n \enspace . \] 

Let \codigo{Σ| be a signature and |X : sorts Σ → Set} be a family of
variables for \codigo{Σ|. An identity |e : Eq Σ X s} is a pair of (open)
terms with sort \codigo{s}. A conditional equation is modelled as record with
fields for the conclusion and the conditions, modelled as an
heterogeneous vector of sorted identities. We declare a constructor
to use the lighter notation \codigo{⋀ eq if (ar , eqs)| instead of }record {
  eq = e ; cond = (ar , eqs )}|. 
\begin{spec}
record Equation (Σ : Signature) (X : sorts Σ → Set) (s : sorts Σ) : Set where
  constructor ⋀_if_
  field
    eq     :   Eq Σ X s
    cond   :   Σ[ ar ∈ List (sorts Σ) ] (HVec (Eq Σ X) ar)
\end{spec}
\noindent A \emph{theory} over the signature $\Sigma$ is given by a
vector of conditional equations.
\begin{spec}
Theory : (Σ : Signature) → (X : sorts Σ → Set) → (ar : List (sorts Σ)) → Set
Theory Σ X ar = HVec (Equation Σ X) ar
\end{spec}
We deviate from Goguen's and Lin's in that we assume that all the
equations of a theory share the same set of variables, while they
assume that each equation has its own set of quantified
variables. Clearly, this simplification is harmless; if we have a
theory where each equation has its own set of variables, we can take
the union of those sets as the common set. As stressed by Goguen and
Meseguer \cite{goguen-remarks-87}, quantifying equations is essential:
\begin{quote}
  [\ldots] the naive unsorted rules of deduction for equational logic
  (namely, reflexivity, symmetry, transitivity and substitutivity) are
  not sound when extended to the many-sorted case in the obvious way;
  [\ldots] adding variable declarations to these rules
  yields a rule set that is sound.
\end{quote}

\comment{\noindent Notice that we follow Goguen and Meseguer in that equations
are given explicitly over a set of variables. This, in turn, leads us
to define satisfiability as proposed by Huet and Oppen.}

\paragraph{Satisfiability} Let $\Sigma$ be a signature and
$\mathcal{A}$ be an algebra for $\Sigma$. We say that a conditional equation
$t = t'\ \mathsf{if}\ t_1 = t'_1,\ldots,t_n=t'_n$ is
\emph{satisfied} by $\mathcal{A}$ if for any
environment $\theta : X \to \mathcal{A}$, $⟦ t ⟧θ = ⟦ t' ⟧θ$, whenever
$⟦ t_i ⟧θ = ⟦ t'_i ⟧θ$ for $1 \leqslant i \leqslant n$. In order to
formalize satisfiability we first define when an environment models an
equation.
\begin{spec}
_⊨ₑ_ : ∀ {Σ X A} → (θ : Env X A) → {s : sorts Σ} → Eq Σ X s → Set
_⊨ₑ_ θ {s} (t , t') = _≈_ (A ⟦ s ⟧ₛ) (⟦ t ⟧ θ) (⟦ t' ⟧ θ)
\end{spec}
\noindent Using the point-wise extension of this relation we can write
directly the notion of satisfiability.
\begin{spec} 
_⊨_ : ∀ {Σ X} (A : Algebra Σ) → {s : sorts Σ} → Equation Σ X s → Set
A ⊨ (⋀ eq if (_ , eqs)) = ∀ θ → ((θ ⊨ₑ_)* eqs) → θ ⊨ₑ eq
\end{spec}

\noindent We say that $\mathcal{A}$ is a \emph{model} of the theory
$E$ if it satisfies each equation in $E$. As usual an equation is a
logical consequence of a theory, if every model of the theory
satisfies the equation.
\begin{spec}
_⊨ₘ_ : ∀ {Σ X ar} → (A : Algebra Σ) → (E : Theory Σ X ar) → Set
A ⊨ₘ E = (A ⊨_)* E

_⊨Σ_ : ∀ {Σ X ar s} → (E : Theory Σ X ar) → (e : Equation Σ X s) → Set
_⊨Σ_ {Σ} E e = (A : Algebra Σ) → A ⊨ₘ E → A ⊨ e
\end{spec}%

\paragraph{Provability} As noticed by Huet and Oppen
\cite{huet-rewrite}, the definition of a sound deduction system for
multi-sorted equality logic is more subtle than expected. We formalize
the system presented in \cite{goguen2005specifying}, shown in
Fig.~\ref{fig:deduction}. The first three rules are reflexivity,
symmetry and transitivity; the fourth rule, called substitution,
allows to instantiate an axiom with a substitution \codigo{σ}, provided one
has proofs for every condition of the axiom;\footnote{In our
  formalization this rule is slightly less general because we assume
  all the equations are quantified over the same set of variables.}
finally, the last rule internalizes Leibniz rule, for replacing equals
by equals in subterms.  Notice that we can only prove identities and
not quasi-identities.
\begin{figure}[t]
  \centering
  \bottomAlignProof
  \AxiomC{}
  \UnaryInfC{$E \vdash \forall X,\, t = t$}
  \DisplayProof\hspace{2ex}
%
  \bottomAlignProof
  \AxiomC{$E \vdash \forall X,\, t_0 = t_1$}
  \UnaryInfC{$E \vdash \forall X,\, t_1 = t_0$}
  \DisplayProof \hspace{2ex}
% 
 \bottomAlignProof
 \AxiomC{$E \vdash \forall X,\, t_0 = t_1$}
  \AxiomC{$E \vdash \forall X,\, t_1 = t_2$}
  \BinaryInfC{$E \vdash \forall X,\, t_0 = t_2$}
  \DisplayProof
\\[6pt]
  \AxiomC{$\forall Y,\,t = t' \ \mathsf{if}\
      t_1=t'_1,\ldots, t_n=t'_n \in E$}
  \AxiomC{$E \vdash \forall X,\,\sigma(t_i) = \sigma(t'_i)$}
  \RightLabel{$\sigma \colon Y \rightarrow T_\Sigma(X)$}
  \BinaryInfC{$E \vdash \forall X,\, \sigma(t) = \sigma(t')$}
  \DisplayProof
\\[6pt]
  \AxiomC{$E \vdash \forall X,\, t_1 = t'_1$}
  \AxiomC{$\cdots$}
  \AxiomC{$E \vdash \forall X,\, t_n = t'_n$}
  \RightLabel{$f : [s_1,...,s_{n}] \Rightarrow_{\Sigma} s$}
  \TrinaryInfC{$E \vdash \forall X,\, f\,(t_1,\ldots,t_n) = f\,(t'_1,\ldots,t'_n)$}
  \DisplayProof
  \caption{Deduction system}
  \label{fig:deduction}
\end{figure}
We define the relation of provability as an inductive type,
parameterized in the theory \codigo{E}, and indexed by the conclusion of the
proof. For conciseness, we only show the constructor for transitivity:
\begin{spec}
data _⊢_ {Σ X ar} (E : Theory Σ X ar) : ∀ {s} → Eq Σ X s → Set where
    ptrans : ∀    {s} {t₀ t₁ t₂} →
                  E ⊢ (t₀ , t₁) → E ⊢ (t₁ , t₂) → E ⊢ (t₀ , t₂)
\end{spec}

Let \codigo{E| be a theory over a signature |Σ}. It is straightforward to
define a setoid over \codigo{∣T∣ Σ 〔 X 〕| by letting |t₁ ≈ t₂| if }E ⊢ t₁ ≈
t₂|; this equivalence relation (thanks to the first three rules) is a
congruence (because of the last rule) over the term algebra. We can
also use the facility provided by the standard library to write
proofs with several transitive steps more nicely, as can be seen
in the next example.

Soundness and completeness are proved as in the
mono-sorted case. For soundness one proceeds by induction on the
derivations; completeness is a consequence of the fact that the quotient of the
term algebra by provable equality is a model.
\begin{theorem}[Soundness and Completeness]
  $E \vdash t ≈ t'$ iff $E \models_{\Sigma} t ≈ t'$.
\end{theorem}
\noindent Let us remark that completeness does not imply that there is a
decidability algorithm for every theory; \ie this result gives no decision
procedure at all.

Let $E$ and $E'$ be two theories over the signature $\Sigma$. We say
that $E$ is \emph{stronger} than $E'$ if every axiom $e \in E'$ can be
deduced from $E$, written $E \vdash_{T}\, E'$.  Obviously if $E$
is stronger than $E'$, then any equation that can be deduced from $E'$
can also be deduced from $E$ and any model of $E$ is also a model of
$E'$. 

\subsection{A theory for Boolean Algebras } In this section we outline
how to formalize an equational theory and illustrate each step by
showing snippets of the formalization of a Boolean Theory presented by
Rocha and Meseguer~\cite{DBLP:conf/RelMiCS/RochaM08}.\footnote{The
  full code is available in the file \nolinkurl{Examples/EqBool.agda}
  of the repository.}


% \begin{description}[%
%   before={\setcounter{descriptcount}{0}},%
%   ,font=\it \stepcounter{descriptcount}\thedescriptcount.~]

\begin{itemize}
\item[Define the signature] describing the language, and choose a
  family of sets for the variables. It helps if one also introduce an
  abbreviation for terms over the signature extended with variables.
\begin{spec}
data bool-ops : List ⊤ × ⊤ → Set where
  f t    : bool-ops ([] ↦ tt)
  neg  : bool-ops ([ tt ] ↦ tt)
  and or  : bool-ops (([ tt , tt ]) ↦ tt)

bool-sig : Signature
bool-sig = record { sorts = ⊤ ; ops = bool-ops }
vars : sorts bool-sig → Set
vars tt = ℕ

Form : Set
Form = HU bool-sig 〔 vars 〕
\end{spec}

\item[Introduce smart-constructors] for terms of the extended
  signature with variables to ease writing the axioms and proving
  theorems. Usually one has a smart-constructor for each operation and
  one per variable that is used in the axioms or the theorems.
\begin{spec}
true false : Form
true = term (inj₁ t) ⟨⟩
false = term (inj₁ f) ⟨⟩

p q  : Form
p = term (inj₂ 0) ⟨⟩
q = term (inj₂ 1) ⟨⟩

_∧_ : Form → Form → Form
φ ∧ ψ = term and ⟨⟨ φ , ψ ⟩⟩

¬ : Form → Form
¬ φ = term neg ⟨⟨ φ ⟩⟩
\end{spec}

\item[Define the equational theory] by specifying one equation for
  each axiom and collect them in a theory; here one can appreciate the
  convenience of the smart-constructors. Here we only show two of the
  twelve axioms of the theory \codigo{bool-theory}. If one will prove theorems
  of the theory, then it is also convenient to define pattern-synonyms
  for the proofs that each axiom is in the theory.
\begin{spec}
commAnd leastDef : Equation bool-sig vars tt
commAnd = ⋀ (p ∧ q) ≈ (q ∧ p) if ([] , ⟨⟩)
leastDef = ⋀ (p ∧ (¬ p)) ≈ false  if ([] , ⟨⟩)

bool-theory : Theory bool-sig vars [ tt , tt , … ]
bool-theory = ⟨ commAnd , leastDef , … ⟩

pattern commAndAx = here
pattern leastDefAx = there here
\end{spec}

\item [Prove theorems] using the axioms of the theory just defined.
  If a proof uses transitivity, one can use the equational reasoning
  idiom provided by the standard library of Agda:
\begin{spec}
  p₁ : bool-theory ⊢ (⋀ ¬ p ∧ p ≈ false)
  p₁ = begin
         ¬ p ∧ p
         ≈⟨ psubst commAndAx σ₁ ∼⟨⟩ ⟩
         p ∧ ¬ p
         ≈⟨ psubst leastDefAx idSubst ∼⟨⟩ ⟩
         false
       ∎
\end{spec}
\noindent In the justification steps of this proof we use the
substitution rule. The relevant actions of the substitution \codigo{σ₁} are
\codigo{σ₁ p = ¬ p| and |σ₁ q = p}.
\end{itemize}
